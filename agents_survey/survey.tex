\documentclass{article}
\usepackage{xcolor}

\def\fb#1{\textbf{#1}}

\newcommand{\todo}[1]{[\textcolor{red}{TODO}: #1]}

\title{Roles in Agent Oriented Programming\\\normalsize{a survey}}
\author{Claudia Grigore, Rem Collier, Radu Grigore}

\begin{document}
\maketitle

% {{{ abstract
\begin{abstract}
\end{abstract}
% abstract }}}
%{{{ introduction
\section{Introduction}

AOP (\fb agent \fb oriented \fb programming) is a research area
whose goal is to find better ways to design, analyse, and develop
programs by using concepts imported from sociology. The products
of researchers working in this area range from methodologies that
are useful in designing programs to programming languages. The
type of applications that benefit the most from these results
are those that are inherently distributed, an example being,
unsurprisingly, simulators of societies.

It is fairly clear now how certain social concepts,
such as belief and behaviour, apply to programming
(Section~\ref{sec:background}). Much recent work focuses on
the assimilation from sociology of the concepts of role and
organisation, which are closely related.

\todo{Continue here.}
Clearly, AOP is inspired by sociology. Recently there has been
much interest in importing other concepts from sociology, such
as roles and organisations, in order to make it easier to write
large scale programs. In this paper we briefly review the concept
of roles as it is used in sociology and then survey attempts by
computer scientists to integrate those concepts in AOP languages
and frameworks. 
%}}} introduction
%{{{ background
\section{Background}

Compared to OOP (object oriented programming), AOP (agent oriented
programming) is done at a higher level of abstraction. The state of
an agent is a set of beliefs, which are described in a logic-like
language. The actions of the agent are dictated by declarative rules.
In AOP style it is easy to simulate societies and to code distributed
algorithms.

\todo{Briefly explain concepts related to agents that are used later,
  \emph{except} roles and organisations. Use examples.}
\todo{Talk about FIPA specs.}
\todo{Talk about JADE.}
\todo{Talk about other AOP languages and frameworks that do \emph{not}
  have roles, but are referred to later, perhaps because they were
  extended with roles.}

The Java Agent DEvelopment framework complies with the FIPA
specifications and is implemented in Java. Several systems
including roles are built on top of it. \todo{Say exactly which
are those systems.}
%}}} background
%{{{ roles in sociology
\section{Roles in Sociology}

Roles are sets of connected behaviours, rights, and obligations
that correspond to the social position of a person (agent) in a
group (organisation). This definition is used both by functionalists
(for example, Auguste Comte) and by modern and contemporary
interactionists.
%}}} roles in sociology
%{{{ roles in software engineering
\section{Roles in Software Engineering}

\todo{Explain the typical relation between agents, roles, and
  organisations.}
\todo{Explain the difference between a (conceptual) model and an
  implementation. Make it clear what is the difference between a
  model and a meta-model.}
AOP (agent oriented programming) is supported by languages and by
frameworks.  \todo{Explain the difference between languages and
frameworks.}


\todo{More intro.}

\todo{For the following, make it clear which are conceptual models and
  which are implementations.}
\todo{Decide when you say `multi-agent system' and when `agent system'.
  It's probably OK to stick to just one.}

\subsection{AGENT0}
\subsection{AFAPL2}
\subsection{PLACA}
\subsection{SPLAW}
\subsection{SLABSp}
\subsection{Jason}
\subsection{3APL, 2APL}
\subsection{GOAL}
\subsection{MetateM}
\subsection{J-Moise+}
\subsection{Normative multi-agent programming language}

\subsection{AALAADIN} 
\label{sec:aalaadin}

AGR, AALAADIN's meta-model, is a model for organisation-centered
multi-agent systems. AGR is based on three main concepts: agents,
groups and roles.

An organisation is constituted of agents that manifest a behaviour.
The overall organisation can be partitioned into groups that may
overlap. 

Agents play roles within groups. A group can be viewed as a
context for activities. Two agents can interact only if they are
members of the same group, but usually an agent is member of
several groups. Groups are sets of agents that either have some
similar characteristics or work together toward a common goal.

A role describes an abstract behaviour of agents and represents the
abstraction of a functional position of an agent in a group.

\subsection{BRAIN} 

The BRAIN Framework supports the development of interactions
in agent-based applications, using the concept of role. It
consists of (1)~an interaction model, (2)~an XML-based notation
to express roles (XRole), and (3)~an interaction infrastructure
implementation (Rolesystem).

A role is a set of capabilities and has an expected behaviour.
Capabilities are activities that an agent can perform. Behaviours
are reactions to incoming events. \todo{What is a `reaction'?
Does a role have one or multiple behaviours? Does a behaviour
prescribe one action for one event or multiple actions for a
combination of events?}

Rolesystem is built on top of Jade. It translates actions of
agents into event notifications to other agents. An event may,
in turn, trigger actions, according to the roles assumed by
the notified agent. In this way, Rolesystem, facilitates the
interaction between agents.

\subsection{Jade extension}

\todo{Does this have a better name than `Jade extension'?}

\todo{This paragraph seems to serve as a good general description
of the relation between agents, roles, and organizations.}
Baldoni et al. extended the Jade framework to offer primitives
for constructing organisations. The underlying conceptual model
is based on the concept of role. Roles are used to structure the
organisation, to distribute responsibilities among agents and as
a means of coordination. A role exists only in the context of an
organisation, working as an interface between the organisation and
agents. Roles offer powers to the agents to operate inside the
organisation, and in the same time request a set of requirements
from them. The expected behaviour is described in terms of goals and
beliefs attributed to a role. An agent can play multiple roles in an
organisation but not simultaneously, only one role at a time can be
active. \todo{addDROPS} \todo{Explain the previous TODO.}

\subsection{JadeOrgs}

JadeOrgs extends Jade. Unlike a Jade agent, a JadeOrgs
agent manages roles and organisation membership. A JadeOrgs
organisation is a Jade agent that manages a group of JadeOrg
agents. \todo{`Manages' sounds a bit like `leads', when what I
mean is that it bookkeeps membership. Rephrase.} Roles consist
of (1)~required behaviours, (2)~required ontologies, and
(3)~templates for messages. \todo{Huh? Why would a role have
templates for messages? Explain.}

\subsection{Moise+} 

The Moise+ organisational model is the first one to join roles
with plans. It consists of three main dimensions: structural,
functional and deontic. Roles represent the structural, plans
are the functional, and the relations between roles and
plans represent the deontic dimension. \todo{Define the word
`deontic'.}

\todo{Decide on some shape for the following three paragraphs
and use the same shape for all three.}

The structural dimension has three levels. The individual level
is represented by roles, which are viewed as sets of constraints
that agents ought to follow. The social level is represented
by links, which are relations between roles, and constrain the
agents' behaviour in relation to other agents. The collective
level is represented by groups, which introduce compatibility
constraints between roles, which specify what roles an agent is
allowed to assume, depending on the roles that it is currently
playing. \todo{Perhaps this paragraph could be made to flow
better.}

The functional dimension is based on the concepts of missions
(sets of coherent goals that an agent can commit to), global
plans (goals in a structure) and social scheme (the goal
decomposition tree). 

The deontic dimension defines at the individual level the set of
obligations and permissions of a role on a mission.

Saci-Moise+, Jason-Moise+ and Moise+JavaAPI frameworks are all
implementations of the Moise+ meta-model. \todo{More detail.}

\subsection{MadKit} 

MadKit is a multi-agent platform written in Java. It is build upon
the AGR organisational model (Section~\ref{sec:aalaadin}).

MadKit supports heterogeneity in agent architectures and
communication languages. Agents can be programmed in Java,
Scheme, Jess and BeanShell.

\subsection{GAIA}
\subsection{MaSE}
\subsection{OMNI}
\subsection{SODA}
\subsection{Styx}
\subsection{Tropos}
%}}} roles in software engineering
%{{{ conclusion
\section{Conclusion}
%}}} conclusion

\bibliographystyle{plain}
\bibliography{survey}

\end{document}

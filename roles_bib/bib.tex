% header {{{1
\documentclass{article}
% packages {{{2
\usepackage[annote]{babelbib}
\usepackage[english]{babel}
\usepackage{microtype}

\usepackage{hyperref}
% macros {{{2
\newcommand{\fb}[1]{\textbf{#1}}
\newenvironment{notes}{\medskip\hrule\smallskip\narrower}{\smallskip\hrule\medskip}
\renewcommand{\btxannotation}[1]{\par\noindent\textit{#1}}
\renewcommand{\sectionautorefname}{Section}

% title, author, ... {{{2
\title{An Annotated Bibliography of Roles in Agent-Oriented Programming}
\begin{document}
\maketitle

% abstract {{{1
\begin{abstract}
We survey the literature and the tools related to roles as used in agent-oriented programming.
The focus is on practical advances, which may be of interest to software developers.
\end{abstract}

\section{Introduction}\label{sec:introduction} %{{{1

AOP (\fb agent-\fb oriented \fb programming) is an alternative to OOP (\fb object-\fb oriented \fb programming).
Very roughly, agents replace objects, logic formulas replace primitive values, messages replace method calls, and roles replace classes and interfaces.
Just as interfaces are very useful for developing large OO programs, so are roles very useful for developing large agent systems.

We are not aware of any large ($>100{,}000$ lines of code) agent system.
This may be because agent-oriented development tools, in general, and tools that support roles and organizations, in particular, are just beginning to mature.
There exist however a few agent systems, some of them fun, some of them useful, and some of them even using roles.

\begin{notes}
\emph{Examples of agent systems}

\paragraph{RoboCup}
According to \url{http://www.robocup.org/}, the first Soccer Server was developed by Itsuki Noda at the ElectroTechnical Laboratory, while doing research on multi-agent systems.
Today's version is used for a yearly competition.
The RoboCup competition has various tracks, some focused on soccer and some on other tasks, some using real robots and some focusing on simulation.
We are here talking about simulation.
The server itself is written in C++.
The clients discuss with the server using UDP (messages).
The simulation is real-time.
The server keeps track of the environment and of client resources, such as stamina.
The environment may be 2D or 3D\null.
A team has 10 field players, 1 goalie, and 1 coach.
A human referee may interrupt the game and award a free kick, for example when he judges players intention as being malicious.

\end{notes}

Next (\autoref{sec:implementations}), we list the main development environments (interpreters, libraries, etc.) that support roles and compare them with regards to various characteristics.
Then (\autoref{sec:general_nonsense}), we present the main methodologies and conceptual models  that complement and motivate the existing implementations.

\section{Implementations}\label{sec:implementations} %{{{1

The following implementations are of high-quality and are actively developed.
\begin{itemize}
\item Jason~\cite{books/sp/map2005/BordiniHV05}
\item 3APL~\cite{books/sp/map2005/DastaniRM05}
\item GOAL~\cite{hindriks2009programmingrationalagents}
\item Jadex~\cite{todo}
\item Aglet~\cite{todo}
\item The RoboCup soccer simulator
\item AgentFactory
\end{itemize}

\section{Methodologies and Conceptual Models}\label{sec:general_nonsense} %{{{1

\section{Conclusions}\label{sec:conclusions} %{{{1

\section{Working Notes} %{{{1
Methodologies that emphasize social aspects:
\begin{itemize}
\item GAIA~\cite{journals/aamas/WooldridgeJK00}
\item INGENIAS
\item MESSAGE
\item SODA
\end{itemize}
Metamodels that emphasize social aspects:
\begin{itemize}
\item AGR~\cite{conf/aose/FerberGM03}
\item RIO
\end{itemize}

TODO: Perl has roles! That \emph{must} be mentioned.

NOTE: There's a slew of abstract work that we ignore, because we focus on those research contributions that resulted in executable artefacts.

Test text:
AOP~\cite{journals/ai/Shoham93} is a fairly new programming paradigm.
Also, Gaia~\cite{journals/aamas/WooldridgeJK00} is so cool.
Software engineering is better ever since~\cite{conf/aose/WooldridgeC00}.

% footer {{{1
\bibliographystyle{babplain-fl}
\bibliography{bib}
\end{document}

% see http://citeseerx.ist.psu.edu/viewdoc/download?doi=10.1.1.33.8207&rep=rep1&type=pdf


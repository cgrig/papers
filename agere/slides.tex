\documentclass{beamer}
\usepackage{fancyvrb}
\usepackage[T1]{fontenc}
\usepackage[english]{babel}
\usepackage{amsmath}
\usepackage{concmath}
\usepackage{graphics}
\usepackage{listings}
\usepackage{microtype}

\usefonttheme{serif}
\usetheme{CambridgeUS}
\setbeamertemplate{navigation symbols}{}
\setbeamertemplate{background}{\includegraphics[width=\paperwidth]{back1.jpg}}
\setbeamertemplate{footline}{}
\setbeamerfont{note page}{size*={8}{10pt}}
%\setbeameroption{show notes}

\lstset{language=Haskell}
\lstset{columns=flexible}
\lstset{keywordstyle=\rmfamily\underbar}
\lstset{morekeywords={role,session}}
\lstset{deletekeywords={Eq,Num}}

\title{AF-Raf: An Agent-Oriented Programming Language with Algebraic Data Types}
\author{Claudia Grigore \and Rem W.~Collier}
\institute{
  School of Computer Science\\
  University College Dublin\\
  IRCSET}
\date[AGERE~2011]{Actors and aGEnts REloaded, 2011}

\begin{document}

\begin{frame}
  \titlepage
\end{frame}

\begin{frame}
\frametitle{Agent Factory and AF-Raf}
\begin{itemize}
\item
  AF is a Java-based development framework for agent-oriented applications.
  AF facilitates integration with diverse programming languages.
\item
  AF-Raf is a new programming language.
\end{itemize}
\end{frame}

\begin{frame}[fragile]
\frametitle{AF-Raf}
Fundamental components: a belief base, and a rule base.
\medskip
\begin{Verbatim}
include stdio
rule State(initialized()) & Name(n) {
    println("hello from " + n);
}
rule Monitoring(name, addr) {
    println("ask " + name + " for status");
    send(agentID(name, addr), request(status()));
}
rule Message(other, status(alive())) {
    println("OK, ask again.");
    send(other, request(status()));
}
\end{Verbatim}

\note[item]{Beliefs are terms.}
\note[item]{Rules are represented by: a name, a query, and an action.}
\note[item]{Rules are evaluated at each time-step.}
\note[item]{Evaluate the query on the current belief base}
\note[item]{Evaluate the action on every query result: substitute the
free variables, then execute the action.}
\end{frame}

\begin{frame}[fragile]
\frametitle{Multi-Sorted Predicate Logic}
\begin{align*}
\mathit{Term}\quad\tau &::= \omega \mid \phi \\
\mathit{Variable}\quad\omega &::= \nu \\
\mathit{Function}\quad\phi &::= \nu(\tau_1,\ldots,\tau_n) \\
\mathit{Name}\quad\nu
\end{align*}

A term not containing variables is said to be
\emph{ground}.

\begin{block}{Sorts}
\begin{align*}
\sigma, \sigma_1, \sigma_2, \sigma_3, \ldots &\in S
\end{align*}

Each function~$\nu$ has a signature
$(\sigma_1\times\cdots\times\sigma_n)\to\sigma$.  
\end{block}

\note[item]{Function~$\nu$ is always applied to $n$~terms whose sorts must be, respectively, $\sigma_1$,~$\sigma_2$, \dots,~$\sigma_n$, and the resulting term has}
\note[item]{A multi-sorted logic has a set~$S$ of sorts.}
\end{frame}


\begin{frame}[fragile]
\frametitle{Algebraic Data Types}
\begin{itemize}
\item
  Types = sets of ground terms.
\item
  A type is defined by a sequence of patterns.
\end{itemize}
\begin{align}
&\mathbf{type}\,\mathit{nat} =
      \mathit{zero}()
  \mid\mathit{succ}(\mathit{nat})
  \mid\mathit{add}(\mathit{nat},\mathit{nat}) \\
&\mathbf{type}\,e =
      \mathit{zero}()
  \mid\mathit{succ}(o)
  \mid\mathit{add}(e,e)
  \mid\mathit{add}(o,o) \\
&\mathbf{type}\,o =
      \mathit{succ}(e)
  \mid\mathit{add}(o,e)
  \mid\mathit{add}(e,o)
\end{align}
\end{frame}

\begin{frame}[fragile]
\frametitle{AF-Raf with Types}
\begin{itemize}
\item
  Primitives and binary operators.
\item
  Using formulas to define subtypes:
 $\mathit{integer}[\mathbf{this}\%2==0]$ 
\item
  Using Algebraic Data Types for defining non-primitive types.
\item 
  The type Any.
\item 
  Type aliases: $\mathbf{type}\,\delta=\delta'$
\end{itemize}
\note[item]{}
\end{frame}

\begin{frame}[fragile]
\frametitle{Type Checking}
\begin{verbatim}
    type term = 
      | A of string * term list 
      | I of int 
      | S of string
    type type_ =
      | AT of (string * type_ list) list
      | IT of (int -> bool)
      | ST of (string -> bool)

    let rec check v ts = match (v, ts) with
      | A (v, vs), AT ts -> let f (t, ts) =
            v = t && all2 check vs ts in List.exists f ts
      | I v, IT ts -> ts v
      | S v, ST ts -> ts v
      | _ -> false
\end{verbatim}
\note[item]{}
\end{frame}

\begin{frame}[fragile]
\frametitle{Types, Debugging, and Autonomy}
\begin{itemize}
\item
  The agent has two types attached: the type of it's beliefs, and the type of
   messages that the agent can process.
\item
  Types are included in the agent identifiers.
\item
  Debugging by preventing the illegal operations and by handling the 
  exceptional circumstances.
\item
  Agent's autonomy: illegal operations are caught and turned into beliefs.
\end{itemize}
\note[item]{}
\end{frame}

\begin{frame}
\centerline{Thank you!}
\note[item]{}
\end{frame}

\end{document}

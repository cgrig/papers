\documentclass[preprint]{sigplanconf} % <<<
\usepackage{amsmath}
\usepackage{amssymb}
\usepackage[british]{babel}
\usepackage{listings}
\usepackage{microtype}
\usepackage{xcolor}

\usepackage{hyperref}

\newcommand{\todo}[1]{{\small \textcolor{gray}{[\textcolor{red}{TODO}: #1]}}}
\newenvironment{notes}{\medskip\hrule\nobreak\smallskip\narrower}{\smallskip\hrule\medskip}

\lstset{columns=fullflexible}
\lstset{identifierstyle=\itshape,commentstyle=\rm}
\lstset{literate={->}{{$\to\;$}}1 {<-}{{$\gets\;$}}1 {=>}{{$\Rightarrow\;$}}1}
\lstdefinestyle{hs}{language=haskell,deletekeywords={elem,all,Eq,Num,Int}}
\lstdefinestyle{me}{language=haskell,keywords={role,agent,plays,true,this,PG,R,rules,send,session,repeat}}

\title{Type-checking in Agent Oriented Programming}
\authorinfo
  {Claudia Grigore and Rem W.~Collier}
  {University College Dublin}
  {claudia.grigore@ucdconnect.ie}

\begin{document} % <<<
\maketitle
\begin{abstract} % <<<
There is virtually no type-checking in Agent Oriented Programming at the moment. Type errors can lead to erroneous calculations. We propose typing message content using abstract data types and typing agent interactions using roles and sessions. We provide a type-checking implementation.

\end{abstract} % >>>
\category{I.2.11}{Artificial Intelligence}{Distributed Artificial
Intelligence}[Multiagent Systems]
\terms Languages
\keywords Types, Messages, Beliefs

\section{Introduction} % <<<
The intuition that a function is like a pair of messages, one carrying the
arguments and one carrying the result, led us to two interesting
observations. First, algebraic data types are convenient for describing the
content of messages. We expect to see fewer runtime errors once messages
are typed. Second, we developed a notion of role in the context of the 2APL
language. These roles have certain similarities with existing 2APL modules
and with Java interfaces, but are nevertheless distinct concepts.

We want types~\cite{DBLP:conf/ctcs/Hagino87}.

% >>>
% ending <<<
\acks

The authors thank the Irish Research Council for Science, Engineering and
Technology for funding.

\bibliographystyle{abbrvnat}
\bibliography{types}

% >>>
\end{document} % >>>

% vim:tw=75:fmr=<<<,>>>:fo+=t:

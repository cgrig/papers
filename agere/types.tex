\documentclass[preprint]{sigplanconf} % <<<
\usepackage{amsmath}
\usepackage{amssymb}
\usepackage{amsthm}
\usepackage[british]{babel}
\usepackage{listings}
\usepackage{microtype}
\usepackage{xcolor}

\usepackage[colorlinks]{hyperref}

\newcommand{\B}{\mathbb{B}}

\renewcommand{\sectionautorefname}{Section}
\renewcommand{\subsectionautorefname}{\sectionautorefname}

\newtheorem{proposition}{Proposition}
%\theoremstyle{definition}
%\newtheorem{example}{Example}
\theoremstyle{remark}
\newtheorem{remark}{Remark}
%\newtheorem{notation}{Notation}

\definecolor{darkblue}{rgb}{0,0,0.4}
\hypersetup{colorlinks,linkcolor=darkblue,citecolor=darkblue,urlcolor=darkblue}

\newcommand{\todo}[1]{{\small \textcolor{gray}{[\textcolor{red}{TODO}: #1]}}}
\newenvironment{notes}{\medskip\hrule\nobreak\smallskip\narrower}{\smallskip\hrule\medskip}

\lstset{columns=fullflexible}
\lstset{identifierstyle=\itshape,commentstyle=\rm}
\lstset{literate={->}{{$\to\;$}}1 {<-}{{$\gets\;$}}1 {=>}{{$\Rightarrow\;$}}1}
\lstdefinestyle{hs}{language=haskell,deletekeywords={elem,all,Eq,Num,Int}}
\lstdefinestyle{me}{language=haskell,keywords={role,agent,plays,true,this,PG,R,rules,send,session,repeat}}
\lstdefinestyle{ocaml}{language=caml}

\title{
  AF-Raf: An Agent-Oriented Programming Language with Algebraic Data Types}
\authorinfo
  {Claudia Grigore and Rem W.~Collier}
  {University College Dublin}
  {claudia.grigore@ucdconnect.ie}

% >>>
\begin{document} % <<<
\maketitle
\begin{abstract} % <<<
Agent-oriented programming languages used in practice do not have good type
systems. It is generally accepted in the programming languages community
that types help developers to write correct programs. We present an
agent-oriented programming language that uses algebraic data types for
dynamically checking beliefs and messages.
\end{abstract} % >>>
\category{I.2.11}{Artificial Intelligence}{Distributed Artificial
Intelligence}[Multiagent Systems]
\terms Languages
\keywords Types, Messages, Beliefs

\section{Introduction} % <<<

In previous work~\cite{grigore2011}, we argue for incorporating ideas from
functional programming into agent-oriented programming. Specifically, we
argue (1)~for importing algebraic data types, (2)~for making roles similar
to type classes, and (3)~for orchestrating global interactions using
session types. In this article we explain how algebraic data types fit in
an agent-oriented programming language. Using the Agent~Factory
framework~\cite{russell2011af} (\autoref{sec:af}), we developed a
programming language called AF-Raf that has beliefs, (behaviour) rules,
(FIPA~\cite{poslad2000fipa}) messages, and algebraic data
types~(\autoref{sec:af-raf} and \autoref{sec:af-raf.types}).

The state of an agent is a belief base, and agents communicate through
messages. Both beliefs and messages are typically formulas in some
predicate logic. Our proposal to add algebraic data types to the
programming language is similar to using a \emph{multi-sorted} predicate
logic for beliefs and for messages~(\autoref{sec:multi-sorted}). The main
benefit is that bugs are easier to identify, both by developers and by
agents~(\autoref{sec:bugs}).

Our main contribution is AF-Raf, a simple agent-oriented programming
language that has algebraic data types~\cite{site:af-raf}.

% >>>
\section{The Agent Factory Framework} \label{sec:af} % <<<

Agent Factory~\cite{collier2002agent} is an open-source Java-based
development framework that provides support for the development and
deployment of agent-oriented applications.

Agent Factory (AF) provides a generic run-time environment for deploying
agent-based systems that is based on the FIPA standards~\cite{poslad2000fipa}.
Central to this environment is a configurable agent platform that supports the
concurrent deployment of heterogeneous agent types employing a range of agent
architectures and interpreters. AF also supports the deployment of
platform-level resources in the form of platform services that are shared
amongst agents, along with monitoring and inspection tools that aid the
developer in debugging their implementations. 

The Common Language Framework (CLF) is a collection of components that
facilitates the design and implementation of diverse Agent Programming
Languages in AF. CLF includes a generic logic framework, a planning and
plan execution framework, a common API model based on sensors, actions and
modules, an outline Grammar and template compiler implementation based on
JavaCC, and a configurable debugging tool.

Currently there are four Agent Programming Languages that have been built
using the CLF: 
\begin{enumerate}

\item \textit{AFAPL}, a reimplementation of the original Agent Factory
agent programming language that is based on commitment rules;

\item \textit{AF-AgentSpeak}, an implementation of the AgentSpeak language based on Jason;

\item \textit{AF-TeleoReactive}, an implementation of Nilsson's teleo-reactive
programming model; 

\item and our new \textit{AF-Raf}, described in the next section.
\end{enumerate}

Agent Factory is fully integrated with Eclipse in a way that simplifies
the task of providing support for new languages and architectures.

For further details on Agent Factory the reader is directed
to~\cite{collier2009modeling}. The Common Language Framework is described
in~\cite{russell2011af}. Also, a discussion on the evolution of Agent Factory since it
was created in the early 1990s can be found in~\cite{muldoon2009towards}.

% >>>
\section{AF-Raf without Types} \label{sec:af-raf} % <<<

AF-Raf is a new programming language we created in the context of Agent
Factory framework. The AF-Raf agent has two fundamental components a
\textit{belief base} (to model the agent's view of the current state of its
environment) and a \textit{rule base} (to model the agent's behaviour).

Beliefs are terms, where a term denotes an expression that can be obtained
from either a constant symbol, a variable or a function symbol. Rules are
represented by a name (optional), a query and an action.  The code in
Figure~\ref{fig:AF-Raf} illustrates an AF-Raf agent.


\begin{figure}\footnotesize % <<<
\begin{center}
\begin{tabular}{c}
\begin{lstlisting}[style=hs]
include stdio

rule State(initialized()) & Name(n) {
    println("hello from " + n);
}

rule Monitoring(name, addr) {
    println("ask " + name + " for status");
    send(agentID(name, addr), request(status()));
}

rule Message(other, status(alive())) {
    println("OK, ask again.");
    send(other, request(status()));
}

rule Message(other, status()) {
    println("Oh, someone wants me alive!");
    send(other, inform(status(alive())));
}
\end{lstlisting}
\end{tabular}
\end{center}
\caption{The code of an AF-Raf agent}
\label{fig:AF-Raf}
\end{figure} % >>>

The rules are evaluated at each time-step. The first stage of evaluating a
rule is to evaluate the query on the current belief base. The result is a
set of query results. If the set is not empty, the action is then evaluated
for every query result. A particular query result says with what term to
substitute each free variable in the action. In other words, a query result
is a set of bindings that covers all the free variable in the action (of
the rule). The first step of evaluating the action is to apply these
substitutions. The next step is to execute it.

An action is either a simple action or a composed action. Executing a
simple action means executing a piece of associated Java code. Examples of
simple actions include sending a message (\textit{send}), logging a string
(\textit{println}), and adopting a belief (\textit{adopt}).  Composed
actions, also referred to as plans, include sequence and while loops.

Sensors are pieces of Java code that are run at each time-step. They are
typically used to update the belief base according to the changes in the
environment. For example, there is a standard sensor (defined in
\texttt{stdio}) that adds a belief $\mathit{Message}(s,c)$ when a message
with content~$c$ is received from sender~$s$.

% >>>
\section{Multi-Sorted Predicate Logic} \label{sec:multi-sorted} % <<<

Agent oriented programming languages use first order logic, more
specifically predicate logic, to represent beliefs and messages. As a
consequence we look at how the introduction of types is managed in
mathematical logic, and recall basic notions necessary to our
understanding. We also introduce notational conventions used in AF-Raf.
Moreover, although these notions are simple, their definitions in
literature tend to have subtle but differences.

A \emph{term} is a variable or a function applied to other terms.
\begin{align}
\mathit{Term}\quad\tau &::= \omega \mid \phi \\
\mathit{Variable}\quad\omega &::= \nu \\
\mathit{Function}\quad\phi &::= \nu(\tau_1,\ldots,\tau_n) \\
\mathit{Name}\quad\nu
\end{align}
In AF-Raf, term names are strings.  Because term names uniquely identify a
variable or a function we will say ``the function~$\nu$'' rather than ``the
function with name~$\nu$.'' A term not containing variables is said to be
\emph{ground}.

\begin{remark}
Ground terms are essentially trees of strings.
\end{remark}

A multi-sorted logic has a set~$S$ of sorts.  We use the letter~$\sigma$ to
denote sorts.
\begin{align}
\sigma, \sigma_1, \sigma_2, \sigma_3, \ldots &\in S
\end{align}
Each function~$\nu$ has a signature
$(\sigma_1\times\cdots\times\sigma_n)\to\sigma$.  Function~$\nu$ is always
applied to $n$~terms whose sorts must be, respectively,
$\sigma_1$,~$\sigma_2$, \dots,~$\sigma_n$, and the resulting term has
sort~$\sigma$. We say that $\nu$~has \emph{arity}~$n$. \emph{Constants} are
functions with arity~$0$.

Typically, in mathematical logic, the set~$S$ of sorts and the function
signatures are required to satisfy further constraints. \textit{Bool} must
be one of the sorts. \emph{Formulas} are terms with sort \textit{Bool}.
The argument sorts ($\sigma_1$,~$\sigma_2$, \dots,~$\sigma_n$) either are
all \textit{Bool}, or none is \textit{Bool}.  If the argument sorts are
\textit{Bool}, then the result sort~$\sigma$ must also be \textit{Bool},
and $\nu$ is said to be a \emph{boolean connective}.  If the argument sorts
are not \textit{Bool} and the result sort~$\sigma$ is \textit{Bool}, then
$\nu$ is said to be a \emph{predicate}.  Agent~Factory has these
constraints and AF-Raf inherits them. In addition, AF-Raf uses an infix
notation (described later) for boolean connectives, letter strings starting
with uppercase for predicate names, and letter strings starting with
lowercase for other function names and for variable names.

\begin{remark}
The definitions given here are in-between what applied computer scientists
tend to prefer and what pure logicians tend to prefer.  Computer scientists
work with \emph{expressions} (rather than terms and formulas) and do not
have constraints on function signatures with respect to booleans.
Logicians, on the other hand, do have these constraints and, moreover,
define terms in such a way that formulas are not terms, and predicates are
not functions. Moreover, logicians single out equality between terms as
being a special predicate.
\end{remark}

In AF-Raf, a \emph{belief} is a boolean ground term; in AF-Raf, a
\emph{message} is a non-boolean ground term.

Algebraic data types offer an alternative way of defining sets of ground
terms. In multi-sorted predicate logic, the sort~$\sigma$ determines a set
of ground terms, namely those of the form $\nu(*)$, where $\nu$ has a
signature of the form $*\to\sigma$.  For example, the sort \textit{Bool}
determines the set of terms that are formulas.  With algebraic data types,
a type~$\delta$ is defined by a sequence of patterns, each of the form
$\nu(\delta_1,\ldots,\delta_n)$. For example, one could define the types
\textit{nat}, $e$, and~$o$ as follows.
\begin{align}
&\mathbf{type}\,\mathit{nat} =
      \mathit{zero}()
  \mid\mathit{succ}(\mathit{nat})
  \mid\mathit{add}(\mathit{nat},\mathit{nat}) \\
&\mathbf{type}\,e =
      \mathit{zero}()
  \mid\mathit{succ}(o)
  \mid\mathit{add}(e,e)
  \mid\mathit{add}(o,o) \\
&\mathbf{type}\,o =
      \mathit{succ}(e)
  \mid\mathit{add}(o,e)
  \mid\mathit{add}(e,o)
\end{align}
We make the following observations.
\begin{enumerate}
\item
  Function names may appear in multiple definitions. For example,
  \textit{zero} appears in the definition of~$\mathit{nat}$ and also in the
  definition of~$e$. In particular, the term $\mathit{zero}()$ belongs both
  to the set of ground terms defined by the type~$\mathit{nat}$ and to the
  one defined by the type~$e$.
\item
  Function names may appear multiple times in the same definition. For
  example, \textit{add} appears twice in the definition of~$e$. (This
  property distinguishes our types from polymorphic
  variants~\cite{garrigue1998}.)
\item
  Type definitions may be recursive or mutually recursive. For example,
  $\mathit{nat}$~appears within the definition of~$\mathit{nat}$.
\end{enumerate}
The set of ground terms corresponding to type~$\mathit{nat}$ is the following.
\begin{equation}
\begin{aligned}
\{\,&\mathit{zero}(), \\
    &\mathit{succ}(\mathit{zero}()),
        \mathit{succ}(\mathit{succ}(\mathit{zero}())), \ldots \\
    &\mathit{add}(\mathit{zero}(), \mathit{zero}()),
        \mathit{add}(\mathit{zero}(), \mathit{succ}(\mathit{zero}())),
        \ldots \\
    &\mathit{succ}(\mathit{add}(\mathit{zero}(), \mathit{zero}())),
        \ldots \\
    &\ldots\, \}
\end{aligned}
\end{equation}
Moreover, $e$~and~$o$ partition~$\mathit{nat}$.

\begin{proposition}
Algebraic data types (as defined above) are strictly more expressive than
sorts. More precisely, (1)~all sets of ground terms that can be defined by
sorts can also be defined by algebraic data types, and (2)~there are pairs
of sets of ground terms that can be defined by algebraic data types but
cannot be defined by sorts.
\end{proposition}

\begin{proof}
For~(1), we define a type~$\delta(\sigma)$ for each sort~$\sigma$ as
follows: For each signature $\nu:(\sigma_1,\ldots,\sigma_n)\to\sigma$ we
add a pattern $\nu(\delta(\sigma_1),\ldots,\delta(\sigma_n))$. It is easy
to see, by structural induction, that the sort~$\sigma$ and the
type~$\delta(\sigma)$ define the same sets of ground terms.

For~(2), note that, given a fixed set of function signatures, the sets of
ground terms defined by distinct sorts are disjoint. In the previous
example, however, the sets of ground terms defined by the types
$\mathit{nat}$~and~$e$ have (at least) a common element, namely
$\mathit{zero}()$.
\end{proof}

Finally, let us note that there are sets of ground terms for which it is
undecidable whether a given term is an element.  (A classic result of
computability theory is that there are undecidable sets of bit-strings, and
sorted ground terms can encode bit-strings.) In particular, there exist
sets of ground terms that cannot be defined using algebraic data types.

% >>>
\section{AF-Raf with Types} \label{sec:af-raf.types} % <<<

In this section we describe how algebraic data types (as described in
\autoref{sec:multi-sorted}) are integrated in AF-Raf.

\subsection{Primitive Types and Aliases}

\autoref{sec:multi-sorted} represents the number~$2$ by the term
$\mathit{succ}(\mathit{succ}(\mathit{zero}()))$, which would be cumbersome
in practice. In Agent~Factory, string literals and integer literals are
also terms, and AF-Raf inherits this decision. AF-Raf has
predefined corresponding primitive types \textit{string} and
\textit{integer}. Moreover, any Java type is suitable to be an AF-Raf
primitive type.

Binary operators such as $+$ are also inherited from Agent~Factory. From
the point of view of type-checking, an expressions like $2+3$ is equivalent
to $+(2,3)$. Standard conventions on precedence and associativity are
obeyed.

More interestingly, AF-Raf has built-in the type $\mathit{integer}[\tau]$,
where $\tau$ is a formula that may use the special variable \textbf{this}.
For example, the type $\mathit{integer}[\mathbf{this}\%2==0]$ defines the
set of ground terms $\{\,\ldots,-4,-2,0,2,4,\ldots\,\}$. Similarly, AF-Raf
has built-in the type $\mathit{string}[\tau]$. For example, the type
\[\mathit{string}[\mathbf{this}\;\mathbf{matches}\;\verb|"[a-z]+"|]\]
defines the set of string literals that match the given regular expression.

Enhanced with algebraic data types, the AF-Raf programming language has
virtually endless possibilities to define new types. The primitive types
are combined using discriminated union and Cartesian product in the manner
described in \autoref{sec:multi-sorted}.

To support the interaction with untyped languages, AF-Raf also has built-in
the type \textit{Any}, which defines the set of all ground terms.

Finally, one may define type aliases.
\begin{align}
\mathbf{type}\,\delta=\delta'
\end{align}
Such aliases are especially useful when $\delta'$ is of the form
$\mathit{integer}[\tau]$ or $\mathit{string}[\tau]$, and $\tau$~is long.

\subsection{Types and Agents}

More interesting is the relation between types, messages, beliefs, and
agents. From a practical point of view, our effort tries to improve upon
the following scenario.

\paragraph{Bad Scenario.}

Agent~$a$ sends message $\tau_{ab}$ to agent~$b$. Agent~$b$ stores a
subterm~$\tau_b$ of~$\tau_{ab}$ into its belief base. Later, when some
other condition is satisfied, agent~$b$ extracts a subterm~$\tau_{bc}$ of
term~$\tau_b$ and sends it to agent~$c$. Agent~$c$ extracts a
subterm~$\tau_c$ of term~$\tau_{bc}$ and stores it in its belief base.
Later, when some other condition is satisfied, agent~$c$ extracts a
subterm~$\tau$ of~$\tau_c$. Agent~$c$ expects $\tau$ to be an integer
literal, but instead $\tau$ is a string literal. Agent~$c$ notices the
problem, but many things happened since agent~$a$ sent an invalid message.

The goal of our dynamic typing is to notice such errors early.

For this purpose, each agent~$a$ has two attached types, the type
$\delta_b(a)$ of $a$'s beliefs and the type $\delta_m(a)$ of messages that
$a$ can process. These types are selected when the agent is created.
Whenever a belief $\tau$ is added to $a$'s belief base, Agent~Factory
checks whether the belief is in the set of terms defined by the belief type
of~$a$; that is, the expression $\tau:\delta_b(a)$ is evaluated. Whenever a
message~$\tau$ is sent to agent~$a$, the expression $\tau:\delta_m(a)$ is
evaluated. In order to send a message to agent~$a$, other agent need the
agent identifier of agent~$a$. Agent identifiers typically consist of a
name and an IP address. To support the check for messages we add
$\delta_m(a)$ to the agent identifier of agent~$a$.

\subsection{Type-Checking Algorithm}

By \emph{type-checking} we mean deciding whether a given ground term~$\tau$
belongs to a given type~$\delta$. This is the operation performed, for
example, on the content of a message before sending it. We include here a
reference implementation for type-checking. We use the language OCaml, for
it is terse.

First, we need data structures for representing terms and types. Terms are
either non-primitive (abstract) or primitive (integers or strings).
\begin{verbatim}
    type term =
      | A of string * term list
      | I of int
      | S of string
\end{verbatim}
A non-primitive type has the shape $A(\nu,\Theta)$, where $\nu$~is a
function name and $\Theta$~is a list of terms. A primitive type simply maps
to the primitive types of the language in which the type-checker is
written.

Similarly, types are either abstract or primitive.
\begin{verbatim}
    type type_ =
      | AT of (string * type_ list) list
      | IT of (int -> bool)
      | ST of (string -> bool)
\end{verbatim}
An abstract type is defined by a list of patterns. Primitive types contain
the user-defined predicates. For example, the type
\[\mathit{integer}[\mathbf{this}\%2==0]\] is represented in the
type-checker by
\begin{verbatim}
    IT (fun x -> x mod 2 = 0)
\end{verbatim}

Type-checking is then straightforward.
\begin{verbatim}
    let rec check v ts = match (v, ts) with
      | A (v, vs), AT ts ->
          let f (t, ts) =
            v = t && all2 check vs ts in
          List.exists f ts
      | I v, IT ts -> ts v
      | S v, ST ts -> ts v
      | _ -> false
\end{verbatim}
This code first checks that the term and the type are of the same kind. For
example, it returns false if the term is non-primitive, but the type is
primitive. The interesting branch is the first one, which is taken when
both the term and the type are abstract. It checks whether there exists a
pattern (in \verb|ts|) that matches the term with the head~\verb|v| and the
sub-terms~\verb|vs|. The function~\verb|f| processes one pattern with the
head~\verb|t| and the sub-types~\verb|ts|. The heads must match and the
sub-terms must match, respectively, the sub-types. The function \verb|all2|
wraps the standard function \verb|List.for_all2| so that it returns
\verb|false| when the number of sub-terms is different from the number of
sub-types.
\begin{verbatim}
    let all2 f xs ys =
      try List.for_all2 f xs ys
      with Invalid_argument _ -> false
\end{verbatim}

We exhibited an algorithm, hence type-checking is decidable. However, note
that in the worst case the running time is exponential in the depth of the
term. Consider the types
\begin{align}
&\mathbf{type}\,t = x(s)\mid x(t) \\
&\mathbf{type}\,s = x(s)\mid x(t)\mid y()
\end{align}
and the term
\begin{align}
x(x(\ldots x(z())\ldots)).
\end{align}
The running time in this case is~$\Theta(2^n)$, where $n$~is the number of
times the function name~$x$ appears in the term.

% >>>
\section{Autonomy and Debugging} \label{sec:bugs} % <<<

Agents are supposed to be autonomous. In particular, illegal operations
should be caught and turned into beliefs that allow the agent to inspect
itself and adjust. An attempt to divide by zero or to apply numerical
addition to strings should not result in a run-time error that terminates
the agent program. Similarly, the failure of a dynamic type-check stops
only the execution of a small part of the agent program and introduces a
new belief that enables introspection. The failure of a message type-check
$\tau:\delta_m(a)$ stops the message~$\tau$ from being sent and adds the
belief \[\mathit{typeError}(\mathit{messageTo}(a), \tau, \tau(\delta_m(a)))
\] to the belief base of the sender agent. Here,
$\tau(\delta_m(a))$~denotes a term representation of the
type~$\delta_m(a)$.  The failure of a belief type-check $\tau:\delta_m$
stops the belief from being adopted and adds the belief
\[\mathit{typeError}(\mathit{belief}(), \tau, \tau(\delta_b(a))) \] to the
belief base of agent~$a$. The addition of \textit{typeError} beliefs is not
itself subjected to type-checks, otherwise it is easy to enter infinite
loops.

The term representation~$\tau(\delta)$ of the type~$\delta$ is is simply a
constant whose name is the same name as the name of~$\delta$. For example,
$\tau(\mathit{nat})$ is~$\mathit{nat}()$.

In the bad scenario of the previous section, agent~$a$ will notice that it
is trying to send an ill-formed message, will stop sending~$\tau_{ab}$, and
will get the chance to take corrective measures. In contrast, without
typing, agent~$c$ notices that there is a problem and, with difficulty, may
be able to determine that agent~$a$ is the source of the problem and inform
it.

In fact, while an agent program is being debugged we want illegal
operations pause the execution. This way, developers see which illegal
operation they need to worry about. On case-by-case basis, they may decide
to prevent the illegal operation from taking place, or to add code that
handles the exceptional circumstances in which the illegal operation takes
place.

% >>>
%\section{Related Work} \label{sec:related} % <<<
%
%Very few approaches were made to establish a type systems in the area of
%agent systems. AMETAS~\cite{} incorporates preliminary work on a such type
%system, where a type is represented by a textual description which declares input and output messages. To structure interactions they adopt the notion of state to which associate a set of acceptable messages and the possible output messages. The state transition is represented using a non-deterministic finite automaton.
%
%There were also a few type systems proposed in the context of mobile agents to
%tackle some of the issues of distributed programming like controlling the
%use of resources (channels, locations, values)~\cite{}KLAIM
% >>>
\section{Conclusions and Future Work} \label{sec:conclusions} % <<<

We briefly described AF-Raf~\cite{site:af-raf}, an agent-oriented
programming language implemented in the Agent Factory framework. AF-Raf
features algebraic data-types. The dynamic type-checks are done by the
Agent Factory framework itself, so that it is possible for other languages
to take advantage of these checks. However, Agent Factory performs no
type-check for existing untyped languages.

The current design and implementation attaches a type~$\delta_m(a)$ to each
agent~$a$, which is rather coarse grained. We would like to express more
specific requirements, such as ``the reply to message~$m$ must be a message
with the type~$\delta$.''

% >>>
%\section{To Move} % <<<
%
%In Agent~Factory, beliefs and messages are logic terms. A \emph{term} is a
%variable or a function applied to other terms.
%\begin{align}
%\mathit{Term}\quad\tau
%  &::= \omega
%  \mid \phi(\tau_1,\ldots,\tau_n) \\
%\mathit{Variable}\quad\omega
%  &::= x, y, z, \ldots \\
%\mathit{Function}\quad\phi
%  &::= f, g, h, \ldots
%\end{align}
%During development, it often happens that agents fail with run-time errors
%caused by terms not having the expected form. For example, an agent might
%expect to receive a message with a constant integer, but gets a string.  (A
%\emph{constant} is a function of arity~$0$.) Often the bug is in the code
%that constructed the message, rather than in the code that parses the
%message and fails at run-time. In order to get better diagnostic support
%during development, we added sorts to the Agent~Factory framework.
%
%Agents should be autonomous, so they should not crash. Instead, illegal
%operations (such as division by~$0$) should be caught and turned into
%beliefs that allow the agent to subsequently adjust. However, this is true
%for a release build. During development we want to ensure that the agent
%will seldom need to take exceptional corrective measures. Hence, during
%development we \emph{do} want agents to fail close to the buggy code.
%
%Traditionally, in a multi-sorted logic each term has a sort~$\sigma$ and
%each function symbol~$\phi$ has a \emph{signature}
%$\sigma_1\times\cdots\times\sigma_n\to\sigma$: The term it constructs has
%sort~$\sigma$, and the $k$th argument must be a term of sort~$\sigma_k$. We
%opt, however, for a more flexible approach. We define a \emph{sort} to be a
%set of terms.
%
% >>>% ending <<<
\acks

The authors thank the Irish Research Council for Science, Engineering and
Technology for funding.

\bibliographystyle{abbrvnat}
\bibliography{types}

% >>>
\end{document} % >>>

% vim:tw=75:fmr=<<<,>>>:fo+=t:nosi:spl=en_gb:

\documentclass[conference,compsoc]{IEEEtran} % <<<
\usepackage[cmex10]{amsmath}\interdisplaylinepenalty=2500
\usepackage{amssymb}
\usepackage[british]{babel}
\usepackage[nocompress]{cite}
\usepackage{listings}
\usepackage{microtype}
\usepackage{xcolor}

\usepackage{hyperref}

\newcommand{\todo}[1]{{\small \textcolor{gray}{[\textcolor{red}{TODO}: #1]}}}
\newenvironment{notes}{\medskip\hrule\nobreak\smallskip\narrower}{\smallskip\hrule\medskip}

\lstset{columns=fullflexible}
\lstset{identifierstyle=\itshape,commentstyle=\rm}
\lstset{literate={->}{{$\to\;$}}1 {<-}{{$\gets\;$}}1 {=>}{{$\Rightarrow\;$}}1}
\lstdefinestyle{hs}{language=haskell,deletekeywords={elem,all,Eq,Num,Int}}
\lstdefinestyle{me}{language=haskell,keywords={role,agent,plays,true,this,PG,R,rules,send,session,repeat}}

% Format notes for [conference,compsoc]
%  - do NOT use \paragraph
%  - for figures, use \centering and put captions after
%  - refer to figures with "Figure", not "Fig" (you can use \figurename)
%  - put algorithms in figures, not other floats
%  - for tables, the caption comes *before*
%  - \section*{Acknowledgment}

\title{Roles and Modules}
\IEEEspecialpapernotice{(Position Paper)}
\author{
  \IEEEauthorblockN{Claudia Grigore and Rem Collier} 
  \IEEEauthorblockA{
    School of Computer Science and Informatics\\
    University College Dublin\\
    Belfield Campus, Dublin~4, Ireland\\
    Email: claudia.grigore@ucdconnect.ie, rem.collier@ucd.ie}}

% >>>
\begin{document} % <<<
\maketitle
\begin{abstract} % <<<

\todo{Say explicitly what is the position that is advocated.}
\todo{The abstract promises something else than is found in the content.
The Background presents 2APL modules, Haskell modules and type classes.
Connections show how type-classes suggest adding roles (and sessions?) to
2APL. It is conceivable that features added to 2APL can be added to other
AOP languages, and there are good reasons for choosing 2APL in the article:
especially with page limits it is good to stay concrete, 2APL has
semantics, there is an article that implies that 2APL does not need roles.}
In Agent-Oriented Programming (AOP), roles and organisational structures
are regarded as important concepts in the analysis and design of agent
systems, but there is a significant lack of support for the implementation
stage of the development process. In response to this, a number of
organisational frameworks have emerged as powerful approaches to
implementing organisations. However, in our mind, this complicates rather
than simplifies the implementation: developers must understand and work
with both the organisational framework and the AOP language.
\todo{Unless there will be a comparison between multiparty session types
and organisations, this seems to promise something that is not in the
article. I think it is fine if the introduction says that we think
methodologies are crap, so we focus only on programming languages. Or,
perhaps, drop the crap.}
 In this paper,
we present an approach that introduces a general notion of a role for AOP
languages, the purpose of which is to reduce the gap between design and
implementation, and to offer run-time support for enforcing role
compliance. 
\todo{I find the world ``general'' bothersome. If it's supposed to mean
``applicable to more than one PL'' then it simply isn't true: The article
only talks about 2APL.} \todo{I think that aiming for runtime enforcement
is idiotic. Let's leave it at this, so I don't get too hot.}
Our approach, which is inspired by Haskell's type classes,
cannot be conveniently implemented using existing AOP language concepts,
such as modules. Conversely, modules cannot be conveniently implemented
with (even our more general) roles. The two concepts are both useful in
organising large agent systems.
\todo{As it is now, the article does not really argue that the concepts are
orthogonal.}

\end{abstract} % >>>
\section{Introduction} % <<<

\begin{notes}
AO is an evolution of OO that draws from social theory. In this paper we
illustrate how another analogy, with functional languages, helps advance 
AO\null. \todo{It may seem strange that the whole article is based on some
analogies. This is why I think it is a good idea that AO in general
profited from analogies. The Introduction is probably the right place, but
I'm not sure what is the right form.}
\end{notes}

% >>>
\section{Background} % <<<

We assume the reader is familiar with Java and with some agent-oriented
programming language, not necessarily
2APL~\cite{DBLP:journals/aamas/Dastani08}. We do \emph{not} assume that the
reader is familiar with Haskell~\cite{web:haskell}. The reader who does
know Haskell will notice that we bend the truth in the interest of
simplicity.

\subsection{Modules in 2APL} % <<<

In (Extended) 2APL, agents are module
instances~\cite{DBLP:conf/prima/DastaniMS08}. At startup, the agent
interpreter creates a set of agents; later, agents may create other agents.
When an agent is created it is given a name.  At this point, an informal
analogy with Java may aid the reader's intuition. Modules in 2APL are like
classes in Java. The 2APL action $\mathit{create}(A,x)$ creates an agent
named~$x$ by instantiating module~$A$; the analogue statement in Java is
$x=\mathbf{new}\,A()$, where $x$~is a field (of the current class).  The
analogy breaks quickly, however. For example, in 2APL it is a fault to
execute $\mathit{create}(A,x)$ twice without executing
$\mathit{release}(x)$ in-between. Hence, the number of co-existing agents
created by an instance of~$B$ is bounded by the number of distinct agent
names that appear in~$B$'s program text. (Note, however, that the total
number of agents is \emph{not} bounded, because an instance of~$A$ may
create an instance of~$B$ which creates an instance of~$A$ which creates an
instance of~$B$ and so on.)

Once an agent is created, its belief base may be queried and updated by the
agent's creator (given the proper access modifier).

Agents do \emph{not} start executing when they are instantiated. Agent~$x$
is executed synchronously by the action $x.\mathit{execute}(t)$ and
asynchronously by the action $x.\mathit{executeasync}(t)$. Coming back to
our Java analogy, these actions are like the statement
$e.\mathit{execute}(x)$, where $e$~is an instance of
$\mathit{java}.\mathit{util}.\mathit{concurrent}.\mathit{Executor}$ and
$x$~is an instance of $\mathit{java}.\mathit{lang}.\mathit{Runnable}$.  In
Java, depending on the executor, a new thread may be created or not; in
2APL, the agent is run asynchronously or not depending on which of
\textit{execute} and \textit{executeasync} is used. But again, the analogy
soon breaks. The test~$t$ has no counterpart in Java. In 2APL, when
$t$~holds, $x$~stops executing.

Other aspects of 2APL modules, such as how they interact with 
events, do not affect the current discussion.

% >>>
\subsection{Modules and Type Classes in Haskell} % <<<

% modules
%  - small example
%  - information hiding and encapsulation
%    (exact impl may change; but it is ONE)
%  - separate compilation
%  - analogy with Java classes

Haskell modules are often used to implement abstract data types such as
sets.  To illustrate the main features of modules in little space, the code
in Figure~\ref{fig:haskell} is contrived.  The module \textit{Set} contains
the type~$T$ and the functions \textit{add}, \textit{has}, and
\textit{sub}. The \textbf{module} line hides \textit{sub} by not mentioning
it. The names and types of the exported functions \textit{add} and
\textit{has} are visible from outside the module, but their
implementations, which are to the right of~$=$, are hidden.  Similarly, the
type name~$T$ is visible from outside, but the value constructor~$V$ is
not. For example, the set $\{1,2\}$ may be represented by the value
$V[2,1,2]$, but this is not known to the users of the module \textit{Set}.
The names and types visible from outside constitute the module's
\emph{interface}.

\begin{figure}\footnotesize % <<<
\begin{center}
\begin{tabular}{c}
\begin{lstlisting}[style=hs]
-- built-in and standard library
class Eq b where
  eq :: b -> b -> Bool
instance Eq Int where ...
elem x [] = False
elem x (y:ys) = eq x y || elem x ys
all p [] = True
all p (x:xs) = p x && all p xs

-- file set.hs
module Set (T, add, has) where
  data T a = V [a]
  add (V s) x = V (x:s)
  has (V s) x = elem x s
  sub (V s) t = all (has t) s

  instance Eq a => Eq (T a) where
    eq s t = sub s t && sub t s
\end{lstlisting}
\end{tabular}
\end{center}
\caption{Haskell type class \textit{Eq} and module \textit{Set}}
\label{fig:haskell}
\end{figure} % >>>

% type classes
%  - continue example
%  - ad-hoc polymorphism: multiple implementations with the same interface
%  - analogy with Java interfaces

The type class \textit{Eq} contains types whose values can be compared for
equality. To make a type belong to the class \textit{Eq} one must write an
instance declaration that provides an implementation for a function
named~$eq$. The \textbf{instance} declaration in module \textit{Set} says
that the type constructor~$T$ transforms members of \textit{Eq} into
members of \textit{Eq}. For example, $T(T\,\mathit{Int})$ is in \textit{Eq}
because \textit{Int} is in \textit{Eq.} Here is \emph{part} of an
evaluation of the comparison between $V[V[1]]$ and $V[V[1,1]]$, both of
which represent~$\{\{1\}\}$.
\begin{align*}
&\phantom{\;\leadsto\;}
  \mathit{eq}\;V[V[1]]\;V[V[1,1]] \\
&\leadsto
  \mathit{sub}\;V[V[1]]\;V[V[1,1]]\\
&\leadsto
  \mathit{all}\;(\mathit{has}\;V[V[1,1]])\;[V[1]]\\
&\leadsto
  \mathit{has}\;V[V[1,1]]\;V[1]\\
&\leadsto
  \mathit{elem}\;V[1]\;[V[1,1]]\\
&\leadsto
  \mathit{eq}\;V[1]\;V[1,1]\\
&\leadsto
  \mathit{sub}\;V[1]\;V[1,1]\\
&\leadsto
  \mathit{all}\;(\mathit{has}\;V[1,1])\;[1]\\
&\leadsto
  \mathit{has}\;V[1,1]\;1 \\
&\leadsto
  \mathit{elem}\;1\;[1,1]\\
&\leadsto
  \mathit{eq}\;1\;1\\
&\leadsto
  \mathit{True}
\end{align*}
Notice, in particular, that \textit{eq} is applied on $T(T\,\mathit{Int})$,
then on $T\,\mathit{Int}$, and then on \textit{Int}.

In general, modules are responsible with information hiding and
encapsulation. Well known advantages come from modularity. For humans,
modular programs are easier to write and to maintain. They are easier to
write because, to a certain degree, it is enough to understand the
interface of a module in order to use it. They are easier to maintain
because changes to a module's implementation remain local as long as the
interface is not affected. For compilers, modular programs enable separate
compilation. Thus, modules are an important device in structuring large
programs. \todo{Perhaps cite some early paper on modularity.}

Haskell modules, like 2APL modules,  are somewhat similar to Java classes.
A Java class hides private parts and introduces \emph{one} type at the same
time. A Haskell module may introduce zero, one, or more types.  But many
Haskell modules do indeed export \emph{one} main type. 

On the other hand, type classes are an elegant mechanism to provide ad-hoc
polymorphism, also known as overloading~\cite{DBLP:conf/popl/WadlerB89}:
The same name refers to different implementations depending on the context.
Type classes are similar to Java interfaces but there are several key
differences~\cite{WEB:PJ-tc}. For example, in Haskell it is easy to declare
a type from one library to be an instance of a class from another library.

% >>> >>>
\section{Connections} % <<<

Haskell and 2APL are very different languages. We do not aim to establish
any sort of formal connection between them, but rather to identify fruitful
high-level similarities. The task is akin trying to draw the Earth's
surface on paper---much easier to do locally than globally. We proceed,
therefore, by finding a contact point, seeing what it tells about its
surroundings, and then repeating.

\subsection{Functions as Messages} % <<<

A function call $f\,x$ is evaluated by `sending'~$x$ to $f$'s body,
evaluating the body, and then receiving the result. The process is
analogous to the exchange of a pair of messages between two agents. For
example, the role \textit{Calculator} could be specified as follows.
\begin{lstlisting}[style=me]
role Num a => Calculator a
  eval :: Expr a -> a
\end{lstlisting}
An agent that plays the role $\mathit{Calculator}\,\mathit{Int}$ knows
how to compute expressions such as $(3+3)\times5$, given another agent
that plays the role $\mathit{Num}\,\mathit{Int}$.
\begin{lstlisting}[style=me]
role Num a
  add :: Pair a -> a
  multiply :: Pair a -> a
\end{lstlisting}
An agent that plays the role $\mathit{Num}\,\mathit{Int}$ knows how to
compute basic operations on integers, such as $3+3$ and $6\times5$. The
types \textit{Expr} and \textit{Pair} constrain the content of messages.
\begin{lstlisting}[style=hs]
data Expr a = Times (Expr a) (Expr a) 
            | Plus (Expr a) (Expr a) 
            | Ct a
data Pair a = MkPair a a
\end{lstlisting}
Given a user agent~$u$, an agent~$c$ that plays
$\mathit{Calculator}\,\mathit{Int}$, and an agent~$n$ that plays
$\mathit{Num}\,\mathit{Int}$, the following is a possible exchange of
messages.
\begin{align*}
u\to c &: 
  \mathit{eval}(\mathit{call}(n),\\
  &\qquad\mathit{Times}(
    \mathit{Plus}(\mathit{Ct}(3),\mathit{Ct}(3)),\mathit{Ct}(5)))\\
c\to n &: \mathit{add}(\mathit{call}(), \mathit{MkPair}(3, 3))\\
n\to c &: \mathit{add}(\mathit{return}(), 6)\\
c\to n &: \mathit{multiply}(\mathit{call}(), \mathit{MkPair}(6, 5))\\
n\to c &: \mathit{multiply}(\mathit{return}(), 30)\\
c\to u &: \mathit{eval}(\mathit{return}(), 30)
\end{align*}

In general, $f::a\to b$ says that the agent understands messages of the
form $f(\mathit{call}(\alpha_1,\ldots,\alpha_n),x)$ and eventually replies
to each of them with a message of the form $f(\mathit{return}(),y)$. Here,
$\alpha_1$, \dots,~$\alpha_n$ are (addresses of) other agents, $x$~is a
value of type~$a$, and $y$~is a value of type~$b$.

The analogy so far is already fruitful. The content of 2APL messages is a
(ground) term or an atomic formula. Since 2APL is built on top
JADE~\cite{DBLP:books/sp/map2005/BellifemineBCP05}, the message content may
also be declared as part of an ontology. However, if we would show the JADE
ontology for arithmetic expressions we would run over the page limit.
Contrast with the three short lines used here to define $\mathit{Expr}\,a$.
The definition is not only short and readable, but also polymorphic in the
type~$a$ of the constants, and rooted in the theory of algebraic data types
(see, for example, \cite{DBLP:conf/ctcs/Hagino87}).

\begin{figure}\footnotesize % <<<
\begin{center}
\begin{tabular}{c}
\begin{lstlisting}[style=me]
agent foo plays Calculator Int(n)
  R-rules:
    eval(Ct(x)) <- x
    eval(Times(x, y)) <-
      n.multiply(MkPair(this.eval(x), this.eval(y)))
    eval(Plus(x, y)) <-
      n.add(MkPair(this.eval(x), this.eval(y)))
\end{lstlisting}
\end{tabular}
\end{center}
\caption{Implementing a role in 2APL}\label{fig:roleimpl2APL}
\end{figure} % >>>

The analogy is not perfect. The earlier declaration for the role
$\mathit{Calculator}\,a$ is superficially similar to the following type
class declaration.
\begin{lstlisting}[style=hs]
class Num a => Calculator a
  eval :: Expr a -> a
\end{lstlisting}
This declaration reads ``a type~$a$ that is a member of the class
\textit{Num} is also a member of class \textit{Calculator} provided there
exist a function \textit{eval} with the proper type.'' In contrast, the
earlier role declaration reads ``an \emph{unnamed} agent plays role
$\mathit{Calculator}\,a$ if it answers to messages
$\mathit{eval}(\mathit{call}(n),\ldots)$ by messages
$\mathit{eval}(\mathit{return}(),\ldots)$, where $n$~is an agent that plays
$\mathit{Num}\,a$.'' Here $a$~is a type variable.  When implementing a role
the agent must be named, as seen in Figure~\ref{fig:roleimpl2APL}. Because
\textit{foo} plays $\mathit{Calculator}\,\mathit{Int}$, the agent
interpreter creates a goal $\mathit{eval}(m,\mathit{call}(n),x)$ for all
messages with shape $\mathit{eval}(\mathit{call}(n),x)$ that come from some
agent~$m$.  One could handle these goals using 2APL's PG-rules.
\begin{lstlisting}[style=me]
eval(m, call(n), Ct(x)) <- true |
  send(m, role, eval(return(), x))
\end{lstlisting}
The first R-rule in Figure~\ref{fig:roleimpl2APL} does exactly the same,
but is more compact. The other two R-rules, however, are much more
cumbersome to simulate with the other kinds of rules. The main reason is
that the notation $n.\mathit{add}(x)$ hides sending a message
$\mathit{add}(\mathit{call}(),x)$ to agent~$n$, waiting for a reply
$\mathit{add}(\mathit{return}(),y)$, and extracting~$y$. The (goal) query
of an R-rule may only be an atom; the right side of an R-rule is an
expression that is evaluated as described and whose result is sent as a
message.  This is a rough and informal sketch of the intended semantics
that needs to be made precise.

Note that two agents \textit{foo} and \textit{bar} may be instances of the
same 2APL module, yet only \textit{foo} plays the role
$\mathit{Calculator}\,\mathit{Int}$.

In summary, the vague and informal intuition that a function is like a pair
of messages, one carrying the arguments and one carrying the result, led
us to two interesting observations. First, algebraic data types are
convenient for describing the content of messages. We expect to see fewer
runtime errors once messages are typed. Second, we developed a notion of
role in the context of the 2APL language that is independent of that of
module.

% >>>
\subsection{Types as Agents} % <<<

In the previous section, Haskell types do not have a clear analogue in
2APL. On one hand, we proposed typing message content, so it would seem
that the analogue of Haskell types are 2APL types for message content. On
another hand, agents play roles and we wrote roles much like Haskell type
classes, so it would seem that the analogue of Haskell types are 2APL
agents. A Haskell type class is a set of Haskell types; a 2APL role is a
set of 2APL agents that play the role. In this section we explore where
does the intuition ``types as agents'' lead.

{\def\l#1->#2:#3<#4>{\mathtt{#1}\to\mathtt{#2}:#3\langle\mathsf{#4}\rangle}
We read $f::a\to b$ as ``message $f$ is sent by agent~$a$ to agent~$b$.'' A
type class lists several function signatures, so its natural analogue is a
list of messages together with their endpoints. It turns out that such a
list is very similar to the global types that describe multiparty sessions
in the context of $\pi$-calculus. Here is an example of such a type from
Honda et al.~\cite{dblp:conf/popl/hondayc08}:
\begin{align*}
\mu\mathbf{t}. 
  &\l DP->K:d<bool>. \\
  &\l KP->K:k<bool>. \\
  &\l K->C:c<bool>.\mathbf{t}
\end{align*}
This type means that process \texttt{K} receives two booleans, one from
\texttt{DP} through channel~$d$ and one from \texttt{KP} through
channel~$k$, then sends a boolean to~\texttt{C} through channel~$c$, and
the whole process repeats. In 2APL we have agents, rather than processes,
and there are no named channels. We would therefore like to write the
following.
\begin{lstlisting}[style=me]
session ComputeBasicOperation(a, b)
  a -> b: Pair Int
  b -> a: Int
session ComputeExpression(a, b, c)
  c -> a: Expr Int
  repeat ComputeBasicOperation(a, b)
  a -> c: Int
\end{lstlisting}
\todo{Continue. Explain how to project sessions as roles.}
}


\begin{notes}
\todo{Talk about multiparty session types.}

\todo{If the goal is to explain what multiparty session types are, then
move to background. The point of \emph{this} section is to say how to
modify 2APL given the insight from `types as agents' and `session types'.}
A session type~\cite{dblp:conf/popl/hondayc08} is an abstraction
representing the structure of conversations, where a session denotes a
natural unit of conversation.  Multiparty session types consist of global
types, that act like a shared agreement among communicating peers, and
local types, that represent the end-point projections of the global types.
The programming methodology for multiparty interactions can be summarised
in two steps:

\begin{itemize}
   \item{ describing the
interaction scenario as a global type, checking its linearity (to
determine its coherence)} \todo{Linearity? Coherence? Am I supposed to know
what these are?}
   \item{coding each participant, incrementally validating its conformance
   to the projection of the global type (the local type), by type-checking.}
   \todo{Why `incremental'? Never heard of incremental type-checking.}
\end{itemize}
\end{notes}
% >>> >>>

\section{Conclusion and Future Work} % <<<

\todo{Make the notion of role more precise, but keep it abstract enough so
that we can implement it in a multi-language platform like AgentFactory.}
\todo{We want language support for roles and organizations.}
\todo{We aim for ease of use, succinctness, and static checking.}
\todo{Where static checking is not possible we'll do runtime checking (in
the framework) and we'll generate tests.}

% >>>
% ending <<<
\section*{Acknowledgment}

The authors thank IRCSET for the financial support of this research.

\bibliographystyle{IEEEtran}
\bibliography{IEEEabrv,rm}

% >>>
\end{document} % >>>

% vim:tw=75:fmr=<<<,>>>:fo+=t:

\documentclass[conference,compsoc]{IEEEtran} % <<<
\usepackage[cmex10]{amsmath}\interdisplaylinepenalty=2500
\usepackage{amssymb}
\usepackage[british]{babel}
\usepackage[nocompress]{cite}
\usepackage{listings}
\usepackage{microtype}
\usepackage{xcolor}

\usepackage{hyperref}

\newcommand{\todo}[1]{{\small \textcolor{gray}{[\textcolor{red}{TODO}: #1]}}}
\newenvironment{notes}{\medskip\hrule\nobreak\smallskip\narrower}{\smallskip\hrule\medskip}

\lstset{columns=fullflexible}
\lstset{identifierstyle=\itshape,commentstyle=\rm}
\lstset{literate={->}{{$\to\;$}}1 {<-}{{$\gets\;$}}1 {=>}{{$\Rightarrow\;$}}1}
\lstdefinestyle{hs}{language=haskell,deletekeywords={elem,all,Eq,Num,Int}}
\lstdefinestyle{me}{language=haskell,keywords={role,agent,plays,true,this,PG,R,rules,send,session,repeat}}

% Format notes for [conference,compsoc]
%  - do NOT use \paragraph
%  - for figures, use \centering and put captions after
%  - refer to figures with "Figure", not "Fig" (you can use \figurename)
%  - put algorithms in figures, not other floats
%  - for tables, the caption comes *before*
%  - \section*{Acknowledgment}

\title{Supporting Agent Systems in the Programming Language}
\IEEEspecialpapernotice{(Position Paper)}
\author{
  \IEEEauthorblockN{Claudia Grigore and Rem Collier} 
  \IEEEauthorblockA{
    School of Computer Science and Informatics\\
    University College Dublin\\
    Belfield Campus, Dublin~4, Ireland\\
    Email: claudia.grigore@ucdconnect.ie, rem.collier@ucd.ie}}

% >>>
% reviewer comments <<<
% Q: results are very preliminary
% A: Yes, we say that  precise solution (that is, a non-preliminary one)
%   cannot possibly fit in the workshop format. We basically outline the
%   path for future conference papers.
% Q: approach is bottom-up, it would be nicer to also look at the issue
%   top-down: abstract from the particular languages, generalize concepts;
%   e.g., "role" is not defined as a high-level concept
% A: [Introduction] We'll add a short history of related research to
%   clarify how we see our relation with sociology, organizational
%   methodologies and frameworks, and other languages (powerJade and
%   powerJava)
% Q: ignores powerJade and powerJava
% A: See above.
% Q: focuses too much on sets (wtf?)
% A: Say once again that the point is to illustrate type classes and
%   modules in Haskell?
% Q: make Background more accessible for those not knowing Haskell
% A: We're skeptical that we can do a much better job in so little space,
%    but we'll rewrite this section. (Ask Luci to read it and say what he
%    can't understand, or finds hard to understand. We expect all of it to
%    be hard. :P)
% Q: typing for goals and beliefs could be future work
% A: Nice idea! We should say this in future work.
% Q: functions as messages: say explicitly that
%   - first message is an "ask" message ??
%   - we aim for roles to be a bit like interfaces ??
% A: We don't understand the first issue. For the second, we'll update
%    conclusions (we added todo note there). Also, say something at the and
%    of ``fun as msg'' section.
% Q: organizational frameworks put a burden on the programmer but bla-bla
%   ... reuse; why do something that can't be reused (huh?)
% A: We'll make it clear in the introduction that our features can coexist
%   with organisational frameworks and the point is to give more choice to
%   developers and see what they like in the long term. Type classes are
%   quite good with reuse, because they can be retro-fitted on already
%   existing types. Perhaps we can explain this in the section on roles.
% Q: can you interact with agents not developed by you (open MAS)?
% A: Again, adding features to a language does not make it able to do less;
%    quite the contrary. Say this in the conclusions?
% Q; related work: a whole survey on organisations, it seems
% A: We'll give some pointer in the history we add in the intro.
% Q: related work: what features of agent-oriented programming languages
%   are now used to implement organisational concepts.
% A: We know of modules in 2APL and some old roles in AFAPL. We also know
%    a few approaches where roles are a type of agent (e.g., powerJade),
%    but this is not a language feature, strictly speaking.
%    We might need to do a bit of literature review to find others. In any
%    case, we'll add this in the intro.
% >>>
\begin{document} % <<<
\maketitle
\begin{abstract} % <<<

Agent-oriented methodologies emphasize organisational concepts, which give
structure to large agent systems. Organisational frameworks, however, put
an extra burden on developers, who need to master both an agent-oriented
programming language and the framework itself. We believe that, instead,
the organisation of agent systems should be directly supported by features
of the programming language. In this paper we propose three such language
features: algebraic data types, roles, and sessions.  They are inspired by
functional programming languages and by session types.  How they fit
together in the context of agent-oriented programming is new.

\end{abstract} % >>>
\section{Introduction} % <<<

% prior work
% - pb: structuring large systems
% - inspiration from sociology
%   - organizations
%   - roles
% - led to
%   - methodologies (Gaia, Moise, AGR)
%   - organizational frameworks (J-Moise, MadKit)
%   - Java libraries (Jade, powerJava, powerJade)
%   - how to use APL lang features

Researchers working on agents recognized for some time the need for better
mechanisms to structure large systems~\cite{todo} and they looked for
inspiration elsewhere. Economists have a well developed theory of
organizations, which studies how \emph{formal} rules of interaction between
people help groups achieve common goals. Sociologists developed a theory of
roles, which focuses on how individuals fit in multiple groups. The
concepts of organization and role influenced AOP (Agent-Oriented
Programming) in a few ways. First, several AO methodologies (for example,
Gaia, Moise+, AGR) are centered around the concept of organisation and led
to the development of organisational frameworks (for example, J-Moise+ and
MadKit). Second, existing AO frameworks (such as Jade) were enhanced to
incorporate organisational concepts (see, for example, powerJade). Third,
researchers identified how existing AOPL (Agent-Oriented Programming
Language) features support organisational concepts (for example, 2APL
modules could be used to implement a certain type of role). \todo{Rewrite.
At least say `organisational concepts' less often.}

However, there is little work on designing language features with the
specific purpose of supporting organisational concepts.
% Rem's roles in AFAPL
% Anything else? RoleEP? ROPE? (see survey)

\todo{Add, perhaps towards the end, a short history of prior research, so
that we explicitly say how we view our position in a community sense. For
example, we want to make clear that we \emph{do not} aim to replace
organisational frameworks, but simply want to explore how extra features of
the programming language could be useful. We give extra choice to
programmers and we want to get their feedback, rather than decreeing what
they should use.  We also want to say something like ``in sociology a role
is the set of rights and obligations of a person within an organisation.''
We should mention (a)~old roles in AFAPL, (b)~implementing roles with 2APL
modules, (c)~viewing roles as a type of agents in powerJade (and others
IIRC).}

Many developers ignore methodologies, but none can ignore the programming
language she uses. Thus, concepts with support in programming languages
have better chances of wide adoption, compared to concepts that appear only
in methodologies. We feel that discussion of how to structure large agent
systems needs to change focus from
methodologies~\cite{dblp:conf/aose/ferbergm03} to programming languages.
In this article, we illustrate the kind of issues on which we intend to
focus in the future, and in which we hope other researchers will become
more interested. By necessity, the style is informal---a precise solution
would require much more than four pages.

The general approach is to get inspiration from analogies between
agent-oriented programming languages and functional languages. Analogies
are relatively common in research involving agents, but usually they are
made with theories in sociology. To avoid the temptation to be overly
abstract, we decided in the beginning to stick to one particular
agent-oriented programming language,
2APL~\cite{DBLP:journals/aamas/Dastani08}, and one particular functional
programming language, Haskell~\cite{web:haskell}. We chose 2APL mainly
because it has formal syntax and semantics, and is therefore easy to
understand. Also, it was suggested that 2APL with modules does not need
roles~\cite{dblp:conf/prima/dastanims08}, a view we disagree with.

% >>>
\section{Background} % <<<

\todo{Rewrite this section.}

We assume the reader is familiar with 2APL. We assume the reader is
familiar with some functional programming language, not necessarily Haskell.
The rest of this section reviews Haskell type classes, because they later
inspire our notion of `role', and Haskell modules, so that it is clear they
are very different from type classes. \todo{Add something like ``We use
sets and equality as examples because we assume all readers are familiar
with these basic mathematical concepts.''} \todo{Add a warning: We expect
those who know Haskell to read this section quickly and to notice a few
shortcuts we took in the interest of sticking to basic Haskell; we expect
those who don't know Haskell to find this section quite difficult because
of the space, but it is important to understand it in order to fully
appreciate the rest.}

% modules
%  - small example
%  - information hiding and encapsulation
%    (exact impl may change; but it is ONE)
%  - separate compilation
%  - analogy with Java classes

Haskell modules are often used to implement abstract data types such as
sets.  To illustrate the main features of modules in little space, the code
in Figure~\ref{fig:haskell} is contrived.  The module \textit{Set} contains
the type~$T$ and the functions \textit{add}, \textit{has}, and
\textit{sub}. The \textbf{module} line hides \textit{sub} by not mentioning
it. The names and types of the exported functions \textit{add} and
\textit{has} are visible from outside the module, but their
implementations, which are to the right of~$=$, are hidden.  Similarly, the
type name~$T$ is visible from outside, but the value constructor~$V$ is
not. For example, the set $\{1,2\}$ may be represented by the value
$V[2,1,2]$, but this is not known to the users of the module \textit{Set}.
The names and types visible from outside constitute the module's
\emph{interface}.

\begin{figure}\footnotesize % <<<
\begin{center}
\begin{tabular}{c}
\begin{lstlisting}[style=hs]
-- built-in and standard library
class Eq b where
  eq :: b -> b -> Bool
instance Eq Int where ...
elem x [] = False
elem x (y:ys) = eq x y || elem x ys
all p [] = True
all p (x:xs) = p x && all p xs

-- file set.hs
module Set (T, add, has) where
  data T a = V [a]
  add (V s) x = V (x:s)
  has (V s) x = elem x s
  sub (V s) t = all (has t) s

  instance Eq a => Eq (T a) where
    eq s t = sub s t && sub t s
\end{lstlisting}
\end{tabular}
\end{center}
\caption{Haskell type class \textit{Eq} and module \textit{Set}}
\label{fig:haskell}
\end{figure} % >>>

% type classes
%  - continue example
%  - ad-hoc polymorphism: multiple implementations with the same interface
%  - analogy with Java interfaces

The type class \textit{Eq} contains types whose values can be compared for
equality. To make a type belong to the class \textit{Eq} one must write an
instance declaration that provides an implementation for a function
named~$eq$. The \textbf{instance} declaration in module \textit{Set} says
that the type constructor~$T$ transforms members of \textit{Eq} into
members of \textit{Eq}. For example, $T(T\,\mathit{Int})$ is in \textit{Eq}
because \textit{Int} is in \textit{Eq.} Here is \emph{part} of an
evaluation of the comparison between $V[V[1]]$ and $V[V[1,1]]$, both of
which represent~$\{\{1\}\}$.
\begin{align*}
&\phantom{\;\leadsto\;}
  \mathit{eq}\;V[V[1]]\;V[V[1,1]] \\
&\leadsto
  \mathit{sub}\;V[V[1]]\;V[V[1,1]]\\
&\leadsto
  \mathit{all}\;(\mathit{has}\;V[V[1,1]])\;[V[1]]\\
&\leadsto
  \mathit{has}\;V[V[1,1]]\;V[1]\\
&\leadsto
  \mathit{elem}\;V[1]\;[V[1,1]]\\
&\leadsto
  \mathit{eq}\;V[1]\;V[1,1]\\
&\leadsto
  \mathit{sub}\;V[1]\;V[1,1]\\
&\leadsto
  \mathit{all}\;(\mathit{has}\;V[1,1])\;[1]\\
&\leadsto
  \mathit{has}\;V[1,1]\;1 \\
&\leadsto
  \mathit{elem}\;1\;[1,1]\\
&\leadsto
  \mathit{eq}\;1\;1\\
&\leadsto
  \mathit{True}
\end{align*}
Notice, in particular, that \textit{eq} is applied on $T(T\,\mathit{Int})$,
then on $T\,\mathit{Int}$, and then on \textit{Int}.

In general, modules are responsible with information
hiding~\cite{DBLP:journals/cacm/Parnas72a} and encapsulation. Well known
advantages come from modularity. For humans, modular programs are easier to
write and to maintain. They are easier to write because, to a certain
degree, it is enough to understand the interface of a module in order to
use it. They are easier to maintain because changes to a module's
implementation remain local as long as the interface is not affected. For
compilers, modular programs enable separate compilation. Thus, modules are
an important device in structuring large programs. Exactly the same is true
of 2APL modules.

On the other hand, type classes are an elegant mechanism to provide ad-hoc
polymorphism, also known as overloading~\cite{DBLP:conf/popl/WadlerB89}:
The same name refers to different implementations depending on the context.
Type classes are similar to Java interfaces but there are several key
differences~\cite{WEB:PJ-tc}. For example, in Haskell it is easy to declare
a type from one library to be an instance of a class from another library.
The notion of role we propose later has similar characteristics.

% >>>
\section{Connections} % <<<

Haskell and 2APL are very different languages. We do not aim to establish
any sort of formal connection between them, but rather to identify fruitful
high-level similarities. The task is akin trying to draw the Earth's
surface on paper---much easier to do locally than globally. We proceed,
therefore, by finding a contact point, seeing what it tells about its
surroundings, and then repeating.

\subsection{Functions as Messages} % <<<

A function call $f\,x$ is evaluated by `sending'~$x$ to $f$'s body,
evaluating the body, and then receiving the result. The process is
analogous to the exchange of a pair of messages between two agents. For
example, the role \textit{Calculator} could be specified as follows.
\begin{lstlisting}[style=me]
role Num a => Calculator a
  eval :: Expr a -> a
\end{lstlisting}
An agent that plays the role $\mathit{Calculator}\,\mathit{Int}$ knows
how to compute expressions such as $(3+3)\times5$, given another agent
that plays the role $\mathit{Num}\,\mathit{Int}$.
\begin{lstlisting}[style=me]
role Num a
  add :: Pair a -> a
  multiply :: Pair a -> a
\end{lstlisting}
An agent that plays the role $\mathit{Num}\,\mathit{Int}$ knows how to
compute basic operations on integers, such as $3+3$ and $6\times5$. The
types \textit{Expr} and \textit{Pair} constrain the content of messages.
\begin{lstlisting}[style=hs]
data Expr a = Times (Expr a) (Expr a) 
            | Plus (Expr a) (Expr a) 
            | Ct a
data Pair a = MkPair a a
\end{lstlisting}
Given a user agent~$u$, an agent~$c$ that plays
$\mathit{Calculator}\,\mathit{Int}$, and an agent~$n$ that plays
$\mathit{Num}\,\mathit{Int}$, the following is a possible exchange of
messages.
\begin{align*}
u\to c &: 
  \mathit{eval}(\mathit{call}(n),\\
  &\qquad\mathit{Times}(
    \mathit{Plus}(\mathit{Ct}(3),\mathit{Ct}(3)),\mathit{Ct}(5)))\\
c\to n &: \mathit{add}(\mathit{call}(), \mathit{MkPair}(3, 3))\\
n\to c &: \mathit{add}(\mathit{return}(), 6)\\
c\to n &: \mathit{multiply}(\mathit{call}(), \mathit{MkPair}(6, 5))\\
n\to c &: \mathit{multiply}(\mathit{return}(), 30)\\
c\to u &: \mathit{eval}(\mathit{return}(), 30)
\end{align*}

In general, $f::a\to b$ says that the agent understands messages of the
form $f(\mathit{call}(\alpha_1,\ldots,\alpha_n),x)$ and eventually replies
to each of them with a message of the form $f(\mathit{return}(),y)$. Here,
$\alpha_1$, \dots,~$\alpha_n$ are (addresses of) other agents, $x$~is a
value of type~$a$, and $y$~is a value of type~$b$.

The analogy so far is already fruitful. The content of 2APL messages is a
(ground) term or an atomic formula. Since 2APL is built on top
JADE~\cite{DBLP:books/sp/map2005/BellifemineBCP05}, the message content may
also be declared as part of an ontology. However, if we would show the JADE
ontology for arithmetic expressions we would run over the page limit.
Contrast with the three short lines used here to define $\mathit{Expr}\,a$.
The definition is not only short and readable, but also polymorphic in the
type~$a$ of the constants, and rooted in the theory of algebraic data types
(see, for example, \cite{DBLP:conf/ctcs/Hagino87}).

\begin{figure}\footnotesize % <<<
\begin{center}
\begin{tabular}{c}
\begin{lstlisting}[style=me]
agent foo plays Calculator Int(n)
  R-rules:
    eval(Ct(x)) <- x
    eval(Times(x, y)) <-
      n.multiply(MkPair(this.eval(x), this.eval(y)))
    eval(Plus(x, y)) <-
      n.add(MkPair(this.eval(x), this.eval(y)))
\end{lstlisting}
\end{tabular}
\end{center}
\caption{Implementing a role in 2APL}\label{fig:roleimpl2APL}
\end{figure} % >>>

The analogy is not perfect. The earlier declaration for the role
$\mathit{Calculator}\,a$ is superficially similar to the following type
class declaration.
\begin{lstlisting}[style=hs]
class Num a => Calculator a
  eval :: Expr a -> a
\end{lstlisting}
This declaration reads ``a type~$a$ that is a member of the class
\textit{Num} is also a member of class \textit{Calculator} provided there
exist a function \textit{eval} with the proper type.'' In contrast, the
earlier role declaration reads ``an \emph{unnamed} agent plays role
$\mathit{Calculator}\,a$ if it answers to messages
$\mathit{eval}(\mathit{call}(n),\ldots)$ by messages
$\mathit{eval}(\mathit{return}(),\ldots)$, where $n$~is an agent that plays
$\mathit{Num}\,a$.'' Here $a$~is a type variable.  When implementing a role
the agent must be named, as seen in Figure~\ref{fig:roleimpl2APL}. Because
\textit{foo} plays $\mathit{Calculator}\,\mathit{Int}$, the agent
interpreter creates a goal $\mathit{eval}(m,\mathit{call}(n),x)$ for all
messages with shape $\mathit{eval}(\mathit{call}(n),x)$ that come from some
agent~$m$.  One could handle these goals using 2APL's PG-rules.
\begin{lstlisting}[style=me]
eval(m, call(n), Ct(x)) <- true |
  send(m, role, eval(return(), x))
\end{lstlisting}
The first R-rule in Figure~\ref{fig:roleimpl2APL} does exactly the same,
but is more compact. The other two R-rules, however, are much more
cumbersome to simulate with the other kinds of rules. The main reason is
that the notation $n.\mathit{add}(x)$ hides sending a message
$\mathit{add}(\mathit{call}(),x)$ to agent~$n$, waiting for a reply
$\mathit{add}(\mathit{return}(),y)$, and extracting~$y$. The (goal) query
of an R-rule may only be an atom; the right side of an R-rule is an
expression that is evaluated as described and whose result is sent as a
message.  This is a rough and informal sketch of the intended semantics
that needs to be made precise.

Note that two agents \textit{foo} and \textit{bar} may be instances of the
same 2APL module, yet only \textit{foo} plays the role
$\mathit{Calculator}\,\mathit{Int}$.

In summary, the vague and informal intuition that a function is like a pair
of messages, one carrying the arguments and one carrying the result, led
us to two interesting observations. First, algebraic data types are
convenient for describing the content of messages. We expect to see fewer
runtime errors once messages are typed. Second, we developed a notion of
role in the context of the 2APL language that is independent of that of
module. \todo{Add something like ``roles act like typed interfaces between
agents.''}

\todo{Explain how roles help reuse: You can make an existing agent play an
extra role, \emph{without having access to the implementation of the
agent}.}

% >>>
\subsection{Types as Agents} % <<<

In the previous section, Haskell types do not have a clear analogue in
2APL. On one hand, we proposed typing message content, so it would seem
that the analogue of Haskell types are 2APL types for message content. On
another hand, agents play roles and we wrote roles much like Haskell type
classes, so it would seem that the analogue of Haskell types are 2APL
agents. A Haskell type class is a set of Haskell types; a 2APL role is a
set of 2APL agents that play the role. In this section we explore where
does the intuition ``types as agents'' lead.

\begin{figure}\footnotesize % <<<
\begin{center}
\begin{tabular}{c}
\begin{lstlisting}[style=me]
session ComputeBasicOperation(a, b)
  a -> b: Pair Int
  b -> a: Int
session ComputeExpression(a, b, c)
  c -> a: Expr Int
  repeat ComputeBasicOperation(a, b)
  a -> c: Int
\end{lstlisting}
\end{tabular}
\end{center}
\caption{Sessions for 2APL}\label{fig:sessions}
\end{figure} % >>>

{\def\l#1->#2:#3<#4>{\mathtt{#1}\to\mathtt{#2}:#3\langle\mathsf{#4}\rangle}
We read $f::a\to b$ as ``message $f$ is sent by agent~$a$ to agent~$b$.'' A
type class lists several function signatures, so its natural analogue is a
list of messages together with their endpoints. It turns out that such a
list is very similar to the global types that describe multiparty sessions
in the context of $\pi$-calculus. Here is an example of such a type from
Honda et al.~\cite{dblp:conf/popl/hondayc08}:
\begin{align*}
\mu\mathbf{t}. 
  &\l DP->K:d<bool>. \\
  &\l KP->K:k<bool>. \\
  &\l K->C:c<bool>.\mathbf{t}
\end{align*}
This type means that process \texttt{K} receives two booleans, one from
\texttt{DP} through channel~$d$ and one from \texttt{KP} through
channel~$k$, then sends a boolean to~\texttt{C} through channel~$c$, and
the whole process repeats. In 2APL we have agents, rather than processes,
and there are no named channels. We would therefore like to write code like
that in Figure~\ref{fig:sessions}.  These sessions are a global description
of the messages that should flow within an agent system. When we project
$\mathit{ComputeExpression}(a,b,c)$ on agent~$a$ we obtain the role
$\mathit{Calculator}\,\mathit{Int}$ from the previous section.}

In agent-oriented methodologies it is standard to say that ``an agent plays
a role within an organisation,'' and therefore organisations are somehow
collections of interacting roles, just as the sessions above are in a way
putting together interacting roles. Similarly, in multiparty session types
there is a notion of projecting global types onto local types. The
essential advantage of session types is that the projection can be done
automatically. By imitating session types, we hope that it will be possible
to at least check automatically that the projection of a certain session on
a certain agent matches a certain role, which the agent implements.

In summary, the vague and informal intuition that a Haskell type is
sometimes like a 2APL agent led us to the proposal of specifying global
interactions in agent-oriented programming languages in terms of sessions.
Moreover, it seems reasonable to expect that a precise link between these
session and the roles proposed in the previous section can be found.

% >>> >>>
\section{Conclusions and Future Work} % <<<

Starting from high-level similarities between certain aspects of an
agent-oriented programming language (2APL) and a functional programming
language (Haskell), we described three new features for the former:
(1)~algebraic data types, which constrain the content of messages,
(2)~roles, which complement modules \todo{Be more specific, something like
``roles at as a typed interface between agents''}, and (3)~sessions, which
describe slices of the global interactions in the agent system. Together,
these \emph{language} features support organisational concepts, which are
so far discussed mostly in the literature on agent-oriented methodologies.

\todo{Say explicitly that adding features to a language does not preclude
one from doing whatever one could do before: interact with agents written
in other languages, use organizations, whatever you could do before. You
also can't do \emph{more} because we're assuming Turing completeness.
However, you can do certain things nicer.}

Our description is informal but, we hope, with enough detail to convince
the reader that pursuing their implementation is a feasible and worthwhile
endeavor. Our position is that many agent-oriented programming languages
would benefit from implementing these three features. In the future, we
will focus on implementing them, which will force us to be precise about
their semantics. We wish to implement them in AgentFactory~\cite{phd:rem},
a framework that supports multiple agent-oriented programming languages,
such that our notion of role is not overly tied into one language.

% >>>
% ending <<<
\section*{Acknowledgment}

The authors thank the Irish Research Council for Science, Engineering and
Technology for funding.

\bibliographystyle{IEEEtran}
\bibliography{IEEEabrv,rm}

% >>>
\end{document} % >>>

% vim:tw=75:fmr=<<<,>>>:fo+=t:

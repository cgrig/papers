\documentclass[conference,compsoc]{IEEEtran}
\usepackage[cmex10]{amsmath}\interdisplaylinepenalty=2500
\usepackage[nocompress]{cite}
\usepackage{microtype}

% TODO: After submitting abstract, switch peerreiewca to conference,
%       and remove \IEEEpeerreviewmaketitle

% Format notes for [conference,compsoc]
%  - do NOT use \paragraph
%  - for figures, use \centering and put captions after
%  - refer to figures with "Figure", not "Fig" (you can use \figurename)
%  - put algorithms in figures, not other floats
%  - for tables, the caption comes *before*
%  - \section*{Acknowledgment}


\title{Roles and Modules}
\IEEEspecialpapernotice{(Position Paper)}
\author{
  \IEEEauthorblockN{Rem Collier and Claudia Grigore} 
  \IEEEauthorblockA{
    School of Computer Science and Informatics\\
    University College Dublin\\
    Belfield Campus, Dublin~4, Ireland\\
    Email: rem.collier@ucd.ie, claudia.grigore@ucdconnect.ie}}

\begin{document}
\maketitle

% problem
% why important (?)
% solution
% consequences

\begin{abstract}
Roles are in danger of being supplanted by modules in agent-oriented
programming languages.  We argue that a more general concept of role,
inspired by Haskell's type classes, cannot be conveniently implemented with
modules. Conversely, modules cannot be conveniently implemented with (even
our more general) roles. The two concepts are both useful in organising
large agent systems.
\end{abstract}

\section*{Notes}

By ``module'', we mean whatever~\cite{DBLP:conf/prima/DastaniMS08} means.

Ideally, we should have an example program that uses both modules and
roles, and is cumbersome to write using only modules or only roles.

Roles (like type classes) can have more than one implementation: Each
agent plays a role in its own way.

We should try to sneak in AgentFactory, if not too hard.


\bibliographystyle{IEEEtran}
\bibliography{IEEEabrv,rm}

\end{document}

% vim:tw=75:

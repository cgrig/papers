\documentclass{beamer}
\usepackage[T1]{fontenc}
\usepackage[english]{babel}
\usepackage{amsmath}
\usepackage{concmath}
\usepackage{graphics}
\usepackage{listings}
\usepackage{microtype}
\usepackage{tikz}

\usefonttheme{serif}
\usetheme{Rochester}
\setbeamertemplate{navigation symbols}{}
\setbeamertemplate{background}{\includegraphics[width=\paperwidth]{back.jpg}}
\setbeamerfont{note page}{size*={8}{10pt}}
%\setbeameroption{show notes}

\lstset{language=Haskell}
\lstset{columns=flexible}
\lstset{keywordstyle=\rmfamily\underbar}
\lstset{morekeywords={role,session}}
\lstset{deletekeywords={Eq,Num}}

\title{Supporting Agent Systems in the Programming Language}
\author{Claudia Grigore \and Rem W.~Collier}
\institute{
  School of Computer Science\\
  University College Dublin\\
  IRCSET}
\date[COIN~2011]{Coordination, Organizations, Institutions and Norms
  in Agent Systems, 2011}

\begin{document}

\begin{frame}
  \titlepage
\note[item]{Agent systems need to be organized. There are many interesting
  and useful ways being explored, such as organizational frameworks. It seems,
  however, that providing some support directly in the language is an direction
  not very well explored.}
\note[item]{First-class support in the language should have a big impact on
  software developers. On the other hand, it is not realistic to expect the
  language to have all the tools necessary to organize all agent systems:
  Such a language would be too complicated, like C++.}
\note[item]{This work is a first step in taking ideas from functional
  programming to improve agent-oriented programming languages.}
\end{frame}

\begin{frame}[fragile]
\frametitle{Modules and Type Classes in Haskell}
\begin{lstlisting}
module Set (SetType, add, has) where
  data SetType a = SetConstructor [a]
  add (SetConstructor s) x = SetConstructor (x:s)
  has (SetConstructor s) x = elem x s
  sub (SetConstructor s) t = all (has t) s

  instance Eq a => Eq (SetType a) where
    eq s t = sub s t && sub t s
\end{lstlisting}
\medskip
\begin{lstlisting}
class Eq b where
  eq :: b -> b -> Bool
\end{lstlisting}
\note[item]{Nice and compact notation.}
\note[item]{In Java code is grouped into modules; in Haskell code is grouped into modules.}
\note[item]{In Java each class is a type; in Haskell a module usually defines one type (here, Set defines SetType), but may define more.}
\note[item]{The element type~$a$ is variable, like with Java generics.}
\note[item]{To construct a set, say SetConstructor followed by a list.
  (Lists are built in.)}
\note[item]{This representation, however, is \emph{not} visible from outside.
  The exposed parts are listed on the first line.}
\note[item]{Warning: Eq is in Haskell's prelude and is slightly different.}
\note[item]{(Eq t) says that type t is in the class Eq. A type is in the
  class Eq if there is a function eq with the proper type.}
\note[item]{The last two lines of the module say that if some type~$a$
  is in the class Eq, then the type (SetType a) is also in the class Eq.}
\end{frame}

\begin{frame}[fragile]
\frametitle{Functions as Messages}
\begin{align*}
f :: a \to b
\end{align*}
means that we send pairs of messages like
\begin{align*}
&f(\mathrm{call}(\alpha_1,\ldots,\alpha_n), x) &&\text{where $x::a$}\\
&f(\mathrm{return}(),y) &&\text{where $y::b$}
\end{align*}
\note[item]{Approach: Let's look at some pieces of notation and try to give
  them sensible semantics in the AOP setting.}
\note[item]{Agents communicate thru messages. We could introduce functions
  as a more abstract mechanism build on top of messages. The call would
  be a message; the return would be another message.}
\note[item]{Then $f::a\to b$ would denote such exchanges of messages---a
  query followed by a reply.}
\note[item]{Here I assume that the whole payload is some first-order structure.
  With a standard like FIPA, one might like to encode call/return using
  performatives rather than a special function-term.}
\note[item]{Here $\alpha_1,\ldots,\alpha_n$ are agent addresses. Their role
  will become clear a little later.}
\note[item]{Once we agree on a structure for the message, it is natural to
  want for the payload to be typed.}
\end{frame}

\begin{frame}[fragile]
\frametitle{Algebraic Data Types}
\begin{block}{Type Declarations}
\begin{lstlisting}
data Expr a =
    Times (Expr a) (Expr a) | Plus (Expr a) (Expr a) | Ct a
data Pair a = MkPair a a
\end{lstlisting}
\end{block}
\begin{block}{Examples}
{ \def\.#1({\mathit{#1}(}
  \def\ei{\mathit{Expr}\,\mathit{Int}}
\begin{align*}
\.Ct(2) &::\ei\\
\.Plus(\.Ct(2),\.Ct(3)) &::\ei \\
\.Times(\.Plus(\.Ct(2),\.Ct(3)),\.Ct(4)) &::\ei \\
\.MkPair(\text{``foo''}, \text{``bar''}) &:: \mathit{Pair}\,\mathit{String}
\end{align*}}
\end{block}
\note[item]{Algebraic data types are a very good way of typing first-order
  structures used in messages by many agent-platforms!}
\note[item]{Haskell values essentially \emph{are} first-order structures.}
\end{frame}


\begin{frame}[fragile]
\frametitle{Roles}
\begin{lstlisting}
role Num a
  add :: Pair a -> a
  multiply :: Pair a -> a

role Num a => Calculator a
  evaluate :: Expr a -> a
\end{lstlisting}
\note[item]{By analogy with type-classes, we \emph{define} a role to be
  a set of function signatures. But we restrict the types to have exactly
  one arrow, because that's the kind of types for which we have semantics.}
\note[item]{An agent that plays the role Num knows how to add and how to
  multiply terms of the type~$a$.}
\note[item]{An agent may play the role Calculator only if it is given
  the address of another agent that plays the role Num.}
\note[item]{A Calculator can evaluate expressions.}
\end{frame}

\begin{frame}[fragile]
\frametitle{Exchange of Messages\,---\,Example}
\begin{center}
\begin{tikzpicture}
\tikzset{agent/.style={rectangle,draw,minimum width=5em,minimum height=4ex}}
\tikzset{msg/.style={draw,->,>=stealth,red}}
\node[agent] (u) at (0,4) {$\mathit{user}$};
\node[agent] (n) at (4,0) {$\mathit{num}$};
\node[agent] (c) at (8,4) {$\mathit{calculator}$};
\uncover<2>{\node[anchor=north west,red] at (u.south west) {$\mathit{evaluate}(\mathit{Times}(\mathit{Plus}(\mathit{Ct}(2),\mathit{Ct}(3)), \mathit{Ct}(4)))$};}
\uncover<3>{\node[anchor=north,red] at (u.south) {$\mathit{evaluate}((2+3)\times4)$};}
\uncover<4-9>{\node[anchor=north] at (u.south) {$\mathit{evaluate}((2+3)\times4)$};}
\uncover<4>{\draw[msg] (u) -- node[below]{$\mathit{evaluate}(\mathrm{call}(\mathit{num}), (2+3)\times4$)} (c);}
\uncover<5>{\draw[msg] (c) -- node[left]{$\mathit{add}(\mathrm{call}(), \mathit{MkPair}(2,3))$}(n);}
\uncover<6>{\draw[msg] (n) -- node[left]{$\mathit{add}(\mathrm{return}(),5)$}(c);}
\uncover<7>{\draw[msg] (c) -- node[left]{$\mathit{multiply}(\mathrm{call}(),\mathit{MkPair}(5,4))$} (n);}
\uncover<8>{\draw[msg] (n) -- node[left]{$\mathit{multiply}(\mathrm{return}(),20)$}(c);}
\uncover<9>{\draw[msg] (c) -- node[below]{$\mathit{evaluate}(\mathrm{return}(), 20)$} (u);}
\uncover<10>{\node[anchor=north,red] at (u.south) {$20$};}
\uncover<11>{\node[anchor=north,red] at (u.south) {$\mathit{Ct}(20)$};}
\end{tikzpicture}
\end{center}
\end{frame}

\note{(1)~Might need an ID to link \texttt{return} to corresponding \texttt{call}.\\
(2)~It's not clear that we want \textit{user} to choose \textit{num}.
Perhaps it's better if the choice is made one when \textit{calculator}
starts playing the role \textit{Calculator}.}

\begin{frame}[fragile]
\frametitle{Types as Agents\,---\,Intuition}
\begin{block}{Idea}
We read \[f :: a \to b\] as
``message~$f$ is sent by agent~$a$ to agent~$b$.''
\end{block}
\begin{block}{Like Session Types!}
\begin{align*}
\mu\mathbf{t}&.\mathtt{DP}\to\mathtt{K}:d\langle\mathsf{bool}\rangle \\
  &.\mathtt{KP}\to\mathtt{K}:k\langle\mathsf{bool}\rangle\\
  &.\mathtt{K}\to\mathtt{C}:c\langle\mathsf{bool}\rangle\\
  &.\mathbf{t}
\end{align*}
\end{block}
\note[item]{}
\end{frame}

\begin{frame}[fragile]
\frametitle{Types as Agents\,---\,Example}
\begin{lstlisting}
session ComputeBasicOperation(a, b)
  a -> b: Pair Int
  b -> a: Int

session ComputeExpression(a, b, c)
  c -> a: Expr Int
  repeat ComputeBasicOperation(a, b)
  a -> c: Int
\end{lstlisting}
\note[item]{}
\end{frame}

\begin{frame}
\frametitle{Conclusions}
\begin{itemize}
\item \alert{algebraic data types}
\item roles
\item sessions
\end{itemize}
\note[item]{I am now implementing support for ADTs in AgentFactory.}
\end{frame}

\begin{frame}
\centerline{Thank you!}
\note[item]{}
\end{frame}

\end{document}

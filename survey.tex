\documentclass{article}
\title{Roles in Agent Oriented Programming\\\normalsize{a survey}}
\author{Claudia Grigore, Rem Collier, Radu Grigore}

\begin{document}
\maketitle

\section{Introduction}

Compared to OOP (object oriented programming), AOP (agent oriented
programming) is done at a higher level of abstraction. The state of
an agent is a set of beliefs, which are described in a logic-like
language. The actions of the agent are dictated by declarative rules.
In AOP style it is easy to simulate societies and to code distributed
algorithms.

Clearly, AOP is inspired by sociology. Recently there has been
much interest in importing other concepts from sociology, roles
and organisations, in order to make it easier to write large scale
programs. In this paper we briefly review the concept of roles as it
is used in sociology and then survey attempts by computer scientists
to integrate those concepts in AOP languages and frameworks. We assume
that the reader is familiar with the basic concepts of AOP (see, for
example, \cite{todo}).

\section{Roles in Sociology}

Roles are sets of connected behaviours, rights, and obligations
that correspond to the social position of a person (agent) in a
group (organisation). This definition is used both by functionalists
(for example, Auguste Comte) and by modern and contemporary
interactionists.


\section{Roles in Software Engineering}

AOP (agent oriented programming) is supported by languages and by
frameworks.

\subsection{Roles in AOP languages}

\subsubsection{AGENT0}
\subsubsection{AFAPL2}
\subsubsection{PLACA}
\subsubsection{SPLAW}
\subsubsection{SLABSp}
\subsubsection{Jason}
\subsubsection{3APL, 2APL}
\subsubsection{GOAL}
\subsubsection{MetateM}
\subsubsection{J-Moise+}
\subsubsection{Normative multi-agent programming language}

\subsection{Roles in AOP frameworks}

\subsubsection{AALAADIN} The AALAADIN meta-model, AGR, is described
as being a model for organisation centered multiagent systems.
AGR is based on three main concepts: agents, groups and roles.

An organisation is constituted of agents that manifest a behaviour.
The overall organisation can be partitioned into groups that may
overlap. 

Agents play roles within groups. A group can be viewed as a context
for activities. Two agents can interact only if they are member of
the same group, but usually an agent is member of several groups.
Groups are sets of agents that are sharing common characteristics, and
represent either sets of similar agents or function based systems.

A role describes an abstract behaviour of agents and represents the
abstraction of a functional position of an agent in a group.


\subsubsection{Jade} 
\subsubsection{Brain} 
\subsubsection{S-Moise+}
\subsubsection{MadKit}

\section{Conclusion}

\bibliographystyle{plain}
\bibliography{survey}

\end{document}

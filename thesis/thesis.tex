\documentclass[a4paper,12pt,oneside,fleqn]{book} %<<<
\usepackage{amsmath}
\usepackage{amsthm}
\usepackage{arev}
\usepackage[T1]{fontenc}
\usepackage{microtype}
\usepackage{thmtools}
\usepackage{xcolor}

\usepackage{hyperref}

\newcommand{\pmap}{\rightharpoonup}
\newcommand{\rg}[1]{\marginpar{\tiny\raggedright\textcolor{blue}{\bf rg:} #1}}
\newcommand{\todo}[1]{[\textcolor{red}{TODO}: #1]}

\renewcommand{\chapterautorefname}{Chapter}

\renewcommand{\qed}{% amsthm doesn't know to go on next line if needed
  \ifmmode\mathqed%
  \else\unskip\nobreak\hfil\penalty50%
    \hskip2em\hbox{}\nobreak\hfill\hbox{\qedsymbol}}
\declaretheorem[qed={(end of example)},style=definition]{example}
\declaretheorem{proposition}
\declaretheorem[qed={(end of remark)},style=remark]{remark}

\setlength{\marginparwidth}{95pt}
\setlength{\mathindent}{3em}

% PDF settings <<<
\definecolor{darkblue}{rgb}{0,0,0.4}
\definecolor{verylightgray}{rgb}{0.95,0.95,0.95}
\hypersetup{colorlinks,linkcolor=darkblue,citecolor=darkblue,urlcolor=darkblue}
%\hypersetup{colorlinks=false}
\hypersetup{
  pdfauthor={Claudia V. Grigore},
  pdftitle={Supporting Agent Systems in the Programming Language}}
% >>>
% >>>
\begin{document} % <<<
\overfullrule=5pt \pretolerance=400 \tolerance=200 % temporary
\title{Supporting Agent Systems in the Programming Language} % <<<
\author{Claudia V. Grigore}
\date{January 2014}
\maketitle

\pagenumbering{roman}
\tableofcontents
\listoffigures
\listoftables
% >>>
\baselineskip=21.75pt
\parskip=3pt plus 3pt minus 3pt
\chapter*{Acknowledgements} % <<<

% >>>

\chapter*{Abstract} % <<<

This thesis defines a novel agent-oriented programming language, called
AF-RAF\null.  The agent-oriented paradigm is just beginning to make strides
into the practice of programming distributed, asynchronous systems.
Several methodologies identified the need for organising principles, which
tend to be inspired by theories of sociology and economy.  In particular,
several methodologies make use of the concepts of role and organisation.
The exact meaning of these concepts varies from one methodology to the
other, and lacks a formal definition.  The primary goal of the AF-RAF
design is to concretise the concepts of role and organisation at the level
of a programming language, thus giving them rigorous meaning. The
development is done in the context of the AgentFactory framework.  Unlike
most agent-oriented programming languages, AF-RAF is mostly a functional
language, with few imperative constructs.  Numerous examples illustrate the
elegance of AF-RAF's design.
\rg{Perhaps you could emphasize something else than the language AF-Raf: the
fusion of ideas from sociology and from type theory, used to give
scalability to an agent-oriented programming language.}


% >>>

\chapter*{Overview} % <<<

The introduction (\autoref{ch:intro}) places agent-oriented programming
within computer science and software engineering. It is meant to provide a
gentle entry point for the reader who might be a software engineer or a
researcher in another area of computer science.

The relevant parts of the research literature on agents is surveyed next
(\autoref{ch:aop}). Since this dissertation is about an agent-oriented
programming language, other agent-oriented programming languages are
clearly relevant, as are agent-oriented methodologies. The concept of
agent, on which these languages and methodologies are based, is inspired by
sociological theories. These theories are surveyed, because they inform the
design of AF-Raf.

The next two chapters present the language AF-Raf. The first of these
(\autoref{ch:concepts}) emphasizes conceptual understanding, makes use of
plenty of examples, and could serve as a first introduction to the language
for a programmer. The second of these (\autoref{ch:langdef}) presents
AF-Raf in a systematic way, and could serve as a reference formal
definition of the language.
\rg{Should say that relevant background on type theory goes here?}

Next comes a bigger case study (\autoref{ch:casestudy}) and a more thorough
discussion of related work (\autoref{ch:related}).


\begin{enumerate}
\item introduction
\item preliminaries
  \begin{enumerate}
  \item role theory
  \item organisational theory
  \item logic
    \begin{itemize}
    \item terms, formulas
    \item sorts
    \end{itemize}
  \item functional languages
    \begin{itemize}
    \item algebraic data types and type-checking
    \item modules
    \item type classes
    \end{itemize}
  \item session types
  \item agentfactory
  \end{enumerate}
\item usage scenarios
  \begin{itemize}
  \item give many examples of use
  \end{itemize}
\item language definition
  \begin{itemize}
  \item syntax
  \item semantics
  \end{itemize}
\item related work
  \begin{itemize}
  \item review other languages
  \end{itemize}
\item conclusions
\end{enumerate}

% >>>

\chapter{Introduction}\label{ch:intro} % <<<
\pagenumbering{arabic}

Research in multi-agent systems thrived since $\sim2000$.
Very recently, it began to impact the practice of programming.

Complex systems are often viewed as being hierarchically decomposed into layers
of subsystems that interact with one another in an inherently decentralized
manner. It is increasingly accepted that this natural lack of an upper layer of
control, together with the tendency within such systems for interaction to
occur between components that are situated within the same layer has a close
synergy with the concept of a multi-agent
system~\cite{Jennings00agent-orientedsoftware}. Multi-agent systems promote a
view of distributed systems as a collection of intelligent (agent) components
that are autonomous and which interact in a way that is highly disciplined and
well defined. Further, through this interaction, agents are able to cooperate
as necessary, allowing competing system objectives to be realized in a context
sensitive fashion.

While the above argument presents a strong case for the use of a multi-agent
systems approach in the design and implementation of complex distributed
systems, it is not enough. It must be supplemented through the provision of
appropriate programming languages, toolkits, and software engineering
methodologies that support and facilitate the adoption of an agent-oriented
perspective. Further, it is paramount that the use and value of these artifacts
be demonstrated and validated through their extensive use within a range of
real world application domains. In response to this, the multi-agent systems
research community has created a range of implementation environments, such as
JADE~\cite{DBLP:books/sp/map2005/BellifemineBCP05}, Agent
Factory~\cite{collier1999agent}, ZEUS~\cite{DBLP:conf/agents/NwanaNLC99},
together with a nascent set of standards developed by the Foundation for
Intelligent Physical Agents (FIPA), and a diverse set of software engineering
methodologies, such as GAIA~\cite{DBLP:journals/aamas/WooldridgeJK00}, Agent
UML~\cite{bauer2001agent}, MaSE~\cite{deloach2001analysis}, and
Prometheus~\cite{DBLP:conf/atal/PadghamW02}.

One approach to implement agent systems is Agent Oriented Programming.
This relatively new paradigm ``promotes a societal view of computation, in
which multiple agents interact with one
another''~\cite{DBLP:journals/ai/Shoham93}.  The agent is the fundamental
unit of computation, analysed and controlled using mental terms.  The state
of an agent is restricted to mental components, such as beliefs,
commitments, capabilities. Agent Oriented Programming tries to match the
programmer intuition to the formal concepts in the same way the Object
Oriented Programming paradigm did before.

Agent Oriented Programming can be viewed as a specialisation of Object
Oriented Programming\null. Object Oriented Programming sees the
computational system as made of units, named objects, that are able to
communicate using messages. Agent Oriented Programming makes this framework
more specific by restricting the state of the unit to consist only of
mental components, and by restricting the types of valid messages to those
specified in an underlying agent communication language.

A number of Agent Oriented Programming languages have been developed to
date, such as Agent-0~\cite{DBLP:journals/ai/Shoham93},
AgentSpeak(L)~\cite{DBLP:conf/maamaw/Rao96},
2APL~\cite{DBLP:journals/aamas/Dastani08},
3APL~\cite{DBLP:conf/promas/DastaniRDM03}, and
AFAPL~\cite{DBLP:conf/seke/CollierOR04}. The research in this area has
focused on clearly defining the reasoning process, linking these processes
to the agents environment, and generally trying to improve the usability of
the languages via better tool support.

Agent Oriented Software Engineering aims to offer methodologies and
toolkits for structuring agent development. A new trend in Agent Oriented
Software Engineering methodologies is to support organisational design for
building dynamic agent organisations.

This approach will be crucial for domains like grid and ubiquitous
computing~\cite{luck2005agent}. The concept of organisation is studied in
several disciplines including sociology, economy and social psychology.
Agent Oriented Software Engineering tries to integrate theories that were
developed in other disciplines, such as the Role
Theory~\cite{biddle1986recent}, and their associated concepts, to model
agent organisation and to structure interactions between the agents inside
of the organisation. One example in this direction is AALAADIN, a
meta-model of multi-agent systems~\cite{ferber1998meta}.

Implementation-level support for these concepts is a research area. One
approach is to design organisational models and frameworks such as
BRAIN~\cite{DBLP:conf/coopis/CabriLZ03},
Moise+~\cite{DBLP:conf/atal/HubnerSB02},
GAIA~\cite{DBLP:journals/aamas/WooldridgeJK00}, and
MaSE~\cite{deloach2001analysis}.

A different approach to support the implementation of organisational
concepts would be to develop custom programming languages, in the context
of Agent Oriented Programming. The majority of Agent Oriented Programming
languages are based on theories from mid-nineties and do not reflect the
increased importance of organisations in multi-agent system design.

\subsubsection{Hypotheses and Objectives}
Agent Oriented Programming is a relatively new programming paradigm that
adopts a social metaphor for the design and implementation of software
systems.  Specifically, software systems are viewed as communities of
software entities, known as agents, that interact with one another in order
to solve problems that are beyond their individual
capabilities~\cite{DBLP:journals/ai/Shoham93}. It is widely accepted that
this approach is well suited to problem domains in which there is no global
system control, data is decentralised and computation is asynchronous.

Agent-oriented methodologies emphasize organisational concepts, which give
structure to large agent systems. Organisational frameworks, however, put
an extra burden on developers, who need to master both an agent-oriented
programming language and the framework itself. We believe that the
organisation of agent systems should be directly supported by features of
the programming language.

Even though organisations are increasingly seen as an important concept in
agent oriented design, little work has been done on applying organisational
concepts to Agent Oriented Programming languages. Integrating the concept
of roles offers significant advantages. They are a natural metaphor for
describing the overall system behaviour, and they increase the adaptability
and flexibility of agent systems by offering the appropriate level of
granularity. Roles define a set of related behaviours, encapsulating them
realises a separation of concerns and promotes information hiding.

The primary hypothesis of my research is that the integration of roles,
based on sociological Role Theory, will improve the readability and
usability of Agent Oriented Programming languages. The objective is to
develop a novel Agent Oriented Programming language that employs roles,
based on the existing Agent Factory framework, in order to show how the
concept of role maps to a programming language construct in the specific
context of Agent Oriented Programming.  

% \rg{Maybe promise only that you show how roles, a concept from sociology,
%     maps to a programming language construct.}

% >>>

\chapter{Agent Oriented Programming}\label{ch:aop} % <<<
\section{Overview} % <<<

Using an anthropomorphic metaphor to design computer programs is not a new
idea. By anthropomorphic metaphor meaning the attribution of human
characteristics and behaviour to a software program. This idea grew within
Artificial Intelligence research which focused on developing computational
models for human thinking. Later, Distributed Artificial Intelligence added
communication and interaction to the processes studied by Artificial
Intelligence.  Within Distributed Artificial Intelligence appeared the
actor model as a way of performing concurrent computation based on message
passing. From the actor model emerged the idea of open systems, which are
systems in continuous interaction with an external environment. Next step
was to build a society of interacting autonomous agents.

The Agent Oriented Programming paradigm was introduced in early '90 by Yoav
Shoham along with the first Agent Oriented Programming language, Agent0.
This relatively new programming paradigm adopts a social metaphor for the
design and implementation of software systems. Specifically, software
systems are seen as communities of software entities, known as agents, that
interact with one another in order to solve problems that are beyond their
individual capabilities ~\cite{DBLP:journals/ai/Shoham93}. An agent
is an entity whose state is viewed as consisting of mental components such
as beliefs, capabilities, choices and commitments. In this context, the role
of an agent program is to control the evolution of the agent's mental
state.

According to Shoham, an agent oriented programming system, in order to be
complete, needs each of the following components:
\begin{enumerate}
   \item A formal language with a clear syntax for describing the mental state.
   \item An interpreted programming language in which to define and program the agents.
   \item A method for converting neutral applications into agents.
\end{enumerate}

The core components of the mental state selected by Shoham to be
implemented in the Agent0 programming language are beliefs, commitments and
capabilities. This three concepts were identified as the minimum necessary
to illustrate a software agent. Shoham argues that more complex mental
notions such as desires, intentions, goals or plans can be included, but
those three are the most important ones.

The generic agent interpreter's basic loop contains two steps: (1). read
the current messages and update the mental state; (2). execute commitments
(possibly resulting in further belief changes).

Agent0 is a prototype programming language used to illustrate the Agent
Oriented Programming paradigm. It includes only constructs related to the
agent level of a multiagent system. In his proposal, Shoham mentioned that
a society of agents would need further organisation in order to work
properly. He suggested that the roles and norms could be appropriate
options to consider, but there was no concrete implementation for any of
the social level constructs.

A rich profusion of agent oriented programming languages followed Agent0,
focusing, as well, on the mental notions and implying a societal view of
computation.  Many of them are heavily influenced by the Belief, Desire,
Intention agent architecture and by the reactive planning theory.  The
Belief, Desire, Intention agent architecture~\cite{DBLP:conf/icmas/RaoG95},
based on the human practical reasoning theory developed by Bratman in the
mid-1980s~\cite{Bratman:1999}, provides a mechanism to separate the plans
selection from the current plans execution.  The reactive
planning~\cite{DBLP:conf/aaai/GeorgeffL87} represents a group of techniques
used in the process of selecting their actions by autonomous agents.
Notable agent oriented programming languages, from a historical point of
view, are AgentSpeak~\cite{DBLP:conf/maamaw/Rao96}, based on speech acts,
MetateM~\cite{DBLP:conf/promas/Fisher05}, based on temporal logic,
3APL~\cite{DBLP:conf/promas/DastaniRDM03}, based on practical reasoning
rules, Jason~\cite{DBLP:books/sp/map2005/BordiniHV05}, which extends
AgentSpeak with communication, and
GOAL~\cite{DBLP:journals/corr/cs-AI-0207008}, based on declarative goals.

A diversity of language constructs are employed to model the abstract
concepts manipulated in the Agent Oriented Programming paradigm. These
abstract concepts include beliefs, goals, intentions, plans, groups, norms,
interactions, roles, artifacts, percepts, actions, and describe the
individual level, the social level and the environmental level of a
multiagent systems. Not all abstract concepts have a one to one
correspondence with the programming language constructs, and not all of the
language constructs are first-class constructs. The individual (agent) level
had most of the attention so far, with environment coming after, and social
level lagging behind. Only recently the social level of multiagent systems
has started to receive more attention.

% >>>
\section{Role Theory} % <<<

The Role Theory elaborated in Sociology and Social Psychology tries to
explain the predictability of individual behaviour by linking it to the
social structures. Role Theory uses the theatre metaphor to describe human
behaviour and patterns of interaction, in this context individuals are
viewed as actors enacting a role in a play. There are several approaches to
Role Theory, each with slightly different definitions of roles, but the
common idea is that each role has associated some behavioral expectations.
Conformity or nonconformity to these expectations trigger rewards or
punishments. These behavioral expectations can be expressed as rights and
duties, and are guided by social norms.

According to B.J. Biddle~\cite{biddle1986recent} there are five major
approaches to Role Theory: the Functional Role Theory have a static
understanding of roles, which are seen as a set of expectations from the
society, inflexible and universally agreed upon; In the Interactionist
Perspective interpersonal interaction is of major importance, roles being
negotiated between individuals on a constant basis; the Structural
Perspective focus more on social structures like the one of status as the
key concept in role definition; Organisational Role Theory looks at the
development of roles in the context of preplanned and task oriented social
systems; The Cognitive Role Theory emphasizes the relationship between
expectations and behaviour.

Role Theory bridges individual behaviour and social structure. Roles as
social constructs influence individual behaviour on different levels. They
have associated characteristic beliefs and attitudes, meaning that whenever
roles are changing beliefs and attitudes change along. Also roles specify
goals, tasks and performance standards required in specific social
situations, acting as plans or blueprints to guide behaviour.

An individual plays multiple roles over time, and at the same time. As a
result roles interact not only at the social level, but at the individual
level as well. A number of issues arise such as role conflict, role strain,
role overload, on the negative side, and role accumulation on the positive
side.

In summary, Role Theory could be synthesised as follows: Social situations
are governed by social norms, which determine behavioral expectations,
which are in turn described in term of duties and rights or obligations and
permissions. The expectations are both internal and external to the
individual. Conformity results in rewards while nonconformity triggers
social sanctions. Role prescriptions are subject to change through social
pressure.

During the last 35 years the role metaphor has been applied to many fields
of computer science including to programming languages, software
engineering, coordination languages, databases, multiagent systems,
knowledge representation, formal ontology, computational linguistics,
security, and conceptual modelling~\cite{DBLP:journals/ao/BoellaTV07}.
Unfortunately, there is little consensus between different approaches,
which makes very hard the transfer of results from one area to another.

Relatively recently the idea of role oriented programming appeared mainly
related to object oriented programming.
ObjectTeams~\cite{DBLP:journals/ao/Herrmann07},
PowerJava~\cite{DBLP:journals/entcs/BaldoniBT06}, and
EpsilonJ~\cite{DBLP:conf/snpd/MonpratarnchaiT08} are examples of such
object oriented programming languages that have explicit support for roles.
A role taxonomy was developed~\cite{graversen06nature} to help role
researchers to understand, communicate and spawn of new ideas, in the
context of object oriented programming.

In the multiagent systems domain, there have been a number of successful
endeavours to incorporate the concept of role. Examples include PowerJava,
an extension of Java, and ROPE~\cite{DBLP:conf/coopis/BechtGKM99}, a
language independent framework.  Sill, to the moment there is no agent
oriented programming language to have explicit support for roles.

%Rope: role oriented programming environment for multiagent systems. Becht
%1999

%Roles survey. Ekinci
% >>>
\section{Organisational Theory}  % <<<

Organisational Theory is a discipline of social sciences for more than 100
years. Organisational Theory studies organisational design, organisational
structures, the relationship of organisations with their environment, and
the behaviour of individuals in organisational settings.

An organisation could be regarded as a coordinated collective action
towards a common goal. Organisations are governed by rules and placed
within an environment.

The study of formal organisations accompanied the industrial
revolution, following the need to organise labour. There are a number of
organisational theories that can be categorised in three groups or stages
depending on their main focus: classical theories focus on the product,
neoclassic theories focus on the employee, and contemporary theories focus
on the environment~\cite{DohertySD01}.

Each of the Organisational Theories can be described in terms of the four
components of a division of labour: the hierarchy of authority, the span of
control, the level of centralisation versus decentralisation in decision
making, and the specialisation of functions or tasks.

The hierarchy of authority refers to a system of rules regarding the line
of communication and control, which determine the organisational structure.
In this context focusing on the specialisation of tasks results in
functional organisations, while focusing on the end product results in
product organisations. Focusing on both function and product results in a
matrix organisation with a dual authority.

The span of control can be narrow resulting in a high organisation, with
multiple levels or wide resulting in a flat organisation, with fewer levels.

The centralisation and decentralisation refers to the number of people and
the grade in which people are involved in decision making. In centralised
organisation the number of people involved in decision making is small,
representing the upper-level in the authority hierarchy. In decentralised
organisations the decision making is delegated to all levels.

Regarding the specialisation, a high degree of specialisation means that
each individual need to perform fewer smaller and simpler tasks, which have
the advantage of promoting proficiency and the need of less transfer time
between tasks, but also the disadvantage of dissatisfaction associated to
boredom.

The organisational rules (constraints, conventions), structures (role
structure) and patterns
(interaction) have been identified
as the most significant organisational components to the multiagent system
development~\cite{DBLP:conf/aose/ZambonelliJW00}.

Within the multiagent systems development settings, a number of
organisational frameworks have been developed to facilitate the definition
of a social or organisational level on top of the agent level. Examples
include Moise+~\cite{DBLP:conf/atal/HubnerSB02},
Gaia~\cite{DBLP:journals/aamas/WooldridgeJK00},
MASE~\cite{deloach2001analysis}, AALAADDIN~\cite{ferber1998meta}.  However,
there is no full support at the language level for the organisation of
agents. It was previously acknowledged that, in programming agent
organisations, developing new organisational abstractions at the
programming language level is the preferred approach in order to reduce the
effort needed for programming and maintaining agent
systems~\cite{DBLP:conf/esaw/RiemsdijkHJ09}.
% >>>

\section{Agent Oriented Programming Languages} % <<<
Role Theory and Organisational Theory have not been exploited yet at the
programming language level. Jason~\cite{DBLP:books/sp/map2005/BordiniHV05},
2APL~\cite{DBLP:journals/aamas/Dastani08}, and
GOAL~\cite{DBLP:journals/corr/cs-AI-0207008} are three representative agent
programming languages, being among the most widely used. What are the similarities between them, and what does
Af-Raf bring new?

\subsection{Jason} % <<<
Jason is an interpreter written in Java for an extended version of the
AgentSpeak programming language. AgentSpeak is logic based and supports
Belief Desire Intention agent architecture. Moreover, the AgentSpeak agent
is a reactive planning system. 

The core concepts defining an AgentSpeak agent are the belief base, the
plan library, the goals and the triggering events. The belief base consists
of a set of ground atomic formulae. There are two types of goals in
AgentSpeak: achievement goals and test goals, both of them written as
atomic formulae prefixed with either "!" or "?". The first one describes a
state of the world to be achieved, while the other specifies a check for
conformity with the agent's belief base. The triggering events can be
either addition or deletion of beliefs and goals. The plan consists of a
head, a context and a body. The head is represented by the triggering
event, the context comprises a conjunction of beliefs which determine if a
plan is applicable, and the body, which is a sequence of basic actions or
(sub)goals. In order to be executed plans have to be designated as
intentions. Actually, intentions represent partially instantiated plans,
that were selected for execution.

Additionally, the AgentSpeak interpreter relies on three selection
functions. First, the event selection selects a single event from the set
of events. Second, the option function selects a plan from the set of
applicable plans. Third, the intentions selection function selects an
intention from the set of intentions.

% >>>

\subsection{2APL} % <<<
2APL is also a Belief Desire Intention architecture-based programming
language. 2APL integrates both the declarative and the imperatives style of
programming. Also, 2APL incorporates two sets of programming constructs:
the ones that implement multiagent concepts and the ones that implement
individual agent concepts. The multiagent level constructs include
functionality like creating individual agents, external environments,
assigning unique names to agents, and specifying access relations. The
individual agent concepts include beliefs, goals, actions, plans, events
and three types of practical reasoning rules. The beliefs and goals are
implemented in a declarative way to support the reasoning and updating
mechanisms. Plans, external environments, events, flow of control are
implemented in an imperative way.

Environments are implemented as Java objects. Information about changes in
the environment can be accessed by the agents in two different ways,
actively through sensors or passively through events. Another distinction
is made between goals and events. Even though both of them trigger agent
action, goals correspond to a desirable state, while the events correspond
to environmental changes.

Different types of actions are available to 2APL agents, such as: updating
beliefs, testing beliefs and goals, managing dynamics of goals, external
actions and communication actions.

The three types of reasoning rules mentioned earlier are: rules that
generate plans to achieve goals, rules that process internal events and
received messages, and rules to handle and repair failed plans.
% >>>

\subsection{GOAL} % <<<
GOAL is a rule-based programming language, its focus is on agent reasoning.
GOAL agents consist of four components: a belief base, a goal base, action
rules, and action specifications. An actual GOAL agent program is a set of
modules each consisting of various sections including knowledge, beliefs,
goals, action specifications, and a program section containing action
rules. Each of the sections is represented in a knowledge representation
language such as Prolog~\cite{DBLP:books/daglib/0076175}.

The init module defines the agent's initial mental state. It includes the
knowledge, the initial beliefs, the goals, and the action specifications.
The main module specifies a strategy for selecting actions using action
rules which, in turn, rely on the agent's mental state. The event module
provides rules for processing information received from the environment.
Actions are specified using preconditions and postconditions.

GOAL agents use a blind commitment strategy, meaning that an agent commits
to achieving a goal and only drops it after it was achieved. GOAL
distinguishes different types of goals: primitive, achievement or achieved
goals~\cite{DBLP:conf/jelia/indriksH08}.

% >>>

\subsection{simpAL} % <<<

% >>>

\subsection{Discussion} % <<<

All of the languages presented above, Jason, 2APL, and GOAL, have a number
of constructs to model the agent level concepts: a belief base to model
agent knowledge, goals that model the agent desires, some sort of plans or
strategies for achieving agent's goals, actions that result in changes,
mechanisms for triggering actions based on the mental state of the agent,
sensing mechanisms to get information about the changes in the environment.
However, even though Jason and 2APL have specific proposal in this
direction, none of them incorporate any social concept to organise agent
interaction. The social level of multiagent systems is where Af-Raf, and
this research work brings its contribution.

Table x synthesises the main features of 2APL, Jason, GOAL and AF-Raf:
\\

\vbox{\tabskip=0pt \offinterlineskip
\def\tablerule{\noalign{\hrule}}
\halign to250pt {
\hfil#&\vrule#\tabskip=1em plus2em&
\hfil#& \vrule#& \hfil#\hfil& \vrule#&
\hfil#& \vrule#\tabskip=0pt\cr\tablerule
&&\omit\hidewidth \textit{Features}\hidewidth&&
\omit\hidewidth \bf{Jason}\hidewidth&&
%\omit\hidewidth \bf{2APL}\hidewidth&&
%\omit\hidewidth \bf{GOAL}\hidewidth&&
\omit\hidewidth \bf{AF-Raf}\hidewidth&\cr\tablerule
&&belief base&&+&&+&\cr\tablerule
&&goals&&+&&+&\cr\tablerule
&&plans&&+&&+&\cr\tablerule
&&actions&&+&&+&\cr\tablerule
&&triggers&&+&&+&\cr\tablerule
&&rules&&+&&+&\cr\tablerule
&&sensing&&+&&+&\cr\tablerule
&&roles&&-&&+&\cr\tablerule
&&organisation&&-&&+&\cr\tablerule \noalign{\smallskip}
\hfil\cr}}

% >>>
% >>>

% >>>

\chapter{Main AF-Raf Concepts}\label{ch:concepts} % <<<
\section{An overview of the AF-Raf programming Language} % <<<

AF-Raf is a new agent oriented programming language specifically developed
to incorporate the support for organising multiagent systems at the
language level. AF-Raf programming language is integrated with the
Agent Factory development framework, which provides support for the
development and deployment of agent-based systems.

The main particularity of AF-Raf, that makes it unique among agent oriented
programming languages is that, even though it is built upon the agent
oriented conceptual framework, it also draws inspiration from functional
languages. Figure~\ref{fig:AF-Raf} illustrates an AF-Raf agent.

\begin{figure}\footnotesize % <<<
\begin{verbatim}
include stdio

rule State(initialized()) & Name(n) {
      println("hello from " + n);
}

rule Monitoring(name, addr) {
      println("ask " + name + " for status");
      send(agentID(name, addr), request(status()));
}

rule Message(other, status(alive())) {
      println("OK, ask again.");
      send(other, request(status()));
}

rule Message(other, status()) {
      println("Oh, someone wants me alive!");
      send(other, inform(status(alive())));
}
\end{verbatim}
\caption{The code of an AF-Raf agent}
\label{fig:AF-Raf}
\end{figure} % >>>

The AF-Raf core concepts include: beliefs, rules, actions, sensors, roles,
sessions, and types. They are related to one another and the unique
features of AF-Raf are built upon them. Among these concepts the ones of
roles, sessions and types constitute new approaches, and they will be
described in more depth in the next sections. 

The first two fundamental components of AF-Raf are the \textit{belief base}
and the \textit{rule base}. The belief base models the agent's view of the
current state of its environment and the rule base models the agent's
behaviour.

Beliefs in AF-Raf are regarded as logical terms and represented in the form
of trees of strings. Beliefs are grouped together in a belief base that can
be accessed by, and updated by the agent. On the other hand, rules
represent a means for accessing the belief base and also a way of
triggering agent behaviour. Figure~\ref{fig:rule} illustrates a simple
AF-Raf rule, where $status()$ and $request(status())$ represent AF-Raf
beliefs. 

A rule has two main components besides a name, which is optional. These
components are a query, and an action. The query is like a condition that
has to be met and can be seen as a question sent to the belief base, if the
belief base is meeting the condition then some action is triggered. The
rules are evaluated at each time-step. The first stage of evaluating a rule
is to evaluate the query on the current belief base. The result is a set of
query results. If the set is not empty, the action is then evaluated for
every query result. A particular query result says with what term to
substitute each free variable in the action. In other words, a query result
is a set of bindings that covers all the free variable in the action (of
the rule). The first step of evaluating the action is to apply these
substitutions. The next step is to execute it.

Executing an action means executing a piece of associated Java code.
%Examples of actions include sending a message (\textit{send}), logging a
%string (\textit{println}), and adopting a belief (\textit{adopt}).  
An action represents the actual agent's behaviour, which can
range from printing something to the console to sending a message to
another agent, creating another agent or even updating the belief base.
Another way to update the belief base is through sensors.

Sensors are pieces of Java code that are run at each time-step. They are
typically used to update the belief base according to the changes in the
environment. For example, there is a standard sensor (defined in
\texttt{stdio}) that adds a belief $\mathit{Message}(s,c)$ when a message
with content~$c$ is received from sender~$s$. 

%Sensors can range from the monitoring of the agent's message inbox to
%monitoring changes in the agent's environment.  

% modules
%  - small example
%  - information hiding and encapsulation
%    (exact impl may change; but it is ONE)
%  - separate compilation
%  - analogy with Java classes

\begin{figure}\footnotesize % <<<
\begin{verbatim}
rule Monitoring(name, addr) {
      println("ask " + name + " for status");
      send(agentID(name, addr), request(status()));
}
\end{verbatim}
\caption{AF-Raf rules example}
\label{fig:rule}
\end{figure} % >>>

Actions and sensors are pieces of Java code used to extend the AF-Raf
programs. Actions are invoked within rules. There is a number of actions
included in the standard library, which any agent can use, there is also
the option to create new actions. Sensors are run automatically every time
step. Figure~\ref{fig:actions-sensors} presents some of the AF-Raf standard
actions and sensors. Both println and send actions have been used in the
rule Monitoring illustrated in figure~\ref{fig:rule}.

 
\begin{figure}\footnotesize % <<<
\begin{verbatim}
action println(s) =
          com.agentfactory.raf.RafStandardAction$Println;

action send(agentID(name, address), message) =
             com.agentfactory.raf.RafStandardAction$Send;

sensor com.agentfactory.raf.RafStandardSensor$Message;
\end{verbatim}
\caption{Standard actions and sensors}
\label{fig:actions-sensors}
\end{figure} % >>>


An operation can be defined as a specific kind of rule stating that upon
receiving a message of a particular type the agent should reply with a
message of a specified type. The message type can be any type including the
type unit(), and in this case there is no actual message returned. Even
though operations can be written as rules, using syntactic sugar to
automatise the process of message passing makes answering messages much
easier.

A role's purpose is to group together related behaviour. As stated before
AF-Raf agents interact through message passing. AF-Raf's role concept
reflects this by considering only those kind of behaviours (rules) that
impact other agents through message passing, namely operations. A role is a
set of operation types, acting in a similar way to interfaces in object
oriented programming. When an agent plays a role it needs to give an
implementation to all the operations defined within that role.

Sessions organise the interaction between agents stating the exact order in
which the messages should succeed. A session brings together related roles
and maps the set of operations defined in those roles to a particular
protocol of conversation. A session specifies the agents involved in the
conversation, as well as the roles they play, then provides a list
containing entries of the form: message type followed by the sender agent
and the receiver agent.

AF-Raf statically checks that the type of each call message in a session
corresponds to an operation defined in the receiver's role, and each return
message corresponds to an operation defined in the sender's role.

AF-Raf checks dynamically that the agent playing a role actually implements
the specified types of the operations defined in that role.  For each call
message there's an associated call id which must appear in the return
message. This is a way of keeping track of the message pair associated to
an operation and enable the dynamic check that the message types correspond
to the ones defined inside the role. If the types do not correspond an
error is risen.

Further checks are performed at runtime. Each session has a session id.
Each message has to contain the session id so that it can be associated to
an specific session. This enables the interpreter to monitor that the
messages succeed in the right order.




\subsection{Agent Factory} % <<<

AF-Raf uses Agent Factory as a platform. Agent
Factory~\cite{collier2002agent} is an open-source Java-based development
framework that provides support for the development and deployment of
agent-oriented applications.

Agent Factory provides a generic run-time environment for deploying
agent-based systems that is based on the FIPA
standards~\cite{poslad2000fipa}.  Central to this environment is one or
more configurable agent platforms that support the concurrent deployment of
heterogeneous agent types that make up an application, and which can employ
a range of agent architectures and interpreters. In other words the agents
can be categorised in two subgroups: the Application Agents and the System
Agents. The Application Agents implement the application logic. The System
Agents provide the services infrastructure needed to support the deployment
of Application Agents.

Platform-level resources in the form of platform services are shared
amongst agents. The default platform services are: Agent management
services, which provides support for creating, terminating, suspending and
resuming agents; and local message transport service, which provide an
intra-platform communication channel. Other platform services include the
HTTP message transport service and the directory facilitator service.

Agent Factory also offers monitoring and inspection tools that aid the
developer in debugging their implementations.

Another key feature of Agent Factory is the Common Language Framework,
which represents a collection of pre-written components that facilitates
the design and implementation of diverse Agent Programming Languages in
Agent Factory\null. The Common Language Framework includes a generic logic
framework, a framework for planning and for executing plans, a common API
model based on sensors, actions and modules, an outline grammar and
template compiler implementation based on JavaCC, and a configurable
debugging tool.

The Common Language Framework supports the building of agent interpreters
on top of the Agent Factory core. These interpreters operate within an
agent platform to control the execution of the agent according to an
interpreted Agent Oriented Programming language. This way agents written in
different languages share the same actions, sensors and modules (libraries
or APIs). Actions, sensors and modules are written in Java, and they extend
the agent program so that it can get information about its environment and
act out its intentions. Actions are used to produce changes to the
environment, while sensors generate beliefs, and change the knowledge of
the agent. Each action has an interface that define how it may be invoked
from within the agent program. If a set of actions and sensors are linked
together they can be defined within a module. A module may also provide
resources that are shared amongst several sensors and actions.

Currently there are four Agent Programming Languages that have been
integrated with Agent Factory using the Common Language Framework:
\begin{enumerate}

\item \textit{AFAPL}, a reimplementation of the original Agent Factory
agent programming language that is based on commitment rules;

\item \textit{AF-AgentSpeak}, an implementation of the AgentSpeak language based on Jason;

\item \textit{AF-TeleoReactive}, an implementation of Nilsson's teleo-reactive
programming model;

\item and the new \textit{AF-Raf} programming language, that draws
inspiration from functional languages .
\end{enumerate}

Agent Factory is fully integrated with Eclipse in a way that simplifies
the task of providing support for new languages and architectures.

For further details on Agent Factory the reader is directed
to~\cite{collier2009modeling}. The Common Language Framework is described
in~\cite{russell2011af}. Also, a discussion on the evolution of Agent Factory since it
was created in the early 1990s can be found in~\cite{muldoon2009towards}.

% >>>

% >>>

\section{The Belief Base} % <<< 

In most agent-oriented programming languages, agents maintain a
\emph{belief base}.  The operations available to the agent are
\textit{query} and \textit{adopt}.  A query asks the belief base which
beliefs that follow a certain pattern it holds.  The adopt operation
changes the state of the belief base by adding and possibly removing
beliefs.  Sometimes, the {\it adopt\/} operation comes in two flavours:
{\it update\/} and {\it revise}.  An update corresponds to a change in the
environment, while a revision is exclusively a mental state change.

\subsubsection{Beliefs}

In general, beliefs are sentences in some logic.  However, the focus of
this work is on roles and organisations, not on beliefs.  Thus, AF-Raf uses
a very simple language for beliefs\,---\,they are essentially trees whose
nodes are labeled by strings: \begin{align} \tau &::= \nu(\tau,\ldots,\tau)
&& \text{function term} \\ \nu  &::= {\rm string} &&\text{function symbol}
\end{align} The number of arguments can be any non-negative integer.

\begin{example}
The following are AF-Raf \emph{beliefs}:
\begin{align}
&{\it hasColor}({\it my}({\it car}()), {\it blue}()) \\
&{\it iHave}({\it cat}(), {\it int}({\it 10}()))
\end{align}
Note that any belief must contain a function with $0$~arguments, such as
{\it car}, {\it blue}, {\it cat}, and {\it 10\/} in this example;
otherwise, the belief would not have a finite size.  Note also that
function symbols can represent both values (such as~{\it 10\/}) and typing
information (such as~{\it int\/}).
\end{example}

Each function symbol is used with a fixed number of arguments.  For
example, if the function symbol ${\it car}$ is used in one belief with
$0$~arguments in the function term ${\it car}()$, then we expect {\it
car\/} to never have arguments.  We say that each function symbol has an
\emph{arity}, which is~$0$ for {\it car}.  The function terms of arity~$0$,
such as ${\it 10}()$ and ${\it car}()$, are said to be \emph{constants}.

\begin{remark}
Of course, it is important what beliefs mean, not only how they are
represented syntactically.  The meaning of beliefs arises from how beliefs
are used by agents.  We will see later how beliefs trigger actions via
rules, and how beliefs spread through messages.
\end{remark}

\subsubsection{Queries}

In general, a query $q$ is a sentence in some logic that determines a
subset of beliefs.  Given a belief base and a query, it is important to be
able to efficiently find the subset of beliefs from the belief base that
satisfy the query.  Again, AF-Raf takes a simple approach, because its
focus is not the belief base.  Each query is term that may contain
variables, which are interpreted as patterns.  \begin{align} \tau &::=
\nu(\tau,\ldots,\tau) && \text{function term} \\ \tau &::= x &&
\text{variable term} \\ \nu  &::= {\rm string} &&\text{function symbol} \\
x &::= {\rm string} && \text{variable symbol} \end{align} The last two
grammar rules are the same as for beliefs.  Thus, any belief can be used as
a query:  It stands for asking whether the belief base contains that
particular belief.

\begin{example} The following are AF-Raf \emph{queries}: \begin{align}
&{\it hasColor}({\it my}({\it car}()), x) \label{eq:query.1} \\ &{\it
hasColor}(y, z) \label{eq:query.2} \end{align} The variables $x$,~$y$,
and~$z$ are easily identifiable, because they are not followed by
parentheses.  Suppose the belief base contains exactly one belief, namely
\begin{align} &{\it hasColor}({\it sky}(), {\it blue}()) \label{eq:skyblue}
\end{align} Belief~\eqref{eq:skyblue} does not match
query~\eqref{eq:query.1} but does match~\eqref{eq:query.2}.  In the latter
case, the match is justified by setting $y$ to be ${\it sky}()$, and
setting $z$ to be ${\it blue}()$.
\end{example}

The previous example shows how a belief matching a query can be described
by a binding of terms, such as ${\it sky}()$ and ${\it blue}()$, to
variables, such as $y$~and~$z$.  More formally, a \emph{match} is a finite
partial map from the set of variables to the set of ground terms.  We write
$[\tau_1/x_1,\ldots,\tau_n/x_n]$ for the match that binds $x_k$
to~$\tau_k$, for all $1\le k\le n$.  Also, we write
$\tau[\tau_1/x_1,\ldots,\tau_n/x_n]$ for the result of substituting in
parallel $x_k$ with~$\tau_k$, for all $1\le k\le n$.

\begin{example} Consider the following belief base \begin{align} &{\it
knows}({\it john}(), {\it john}()) \\ &{\it knows}({\it john}(), {\it
laura}()) \end{align} and the following queries \begin{align} &{\it
knows}(x, y) \\ &{\it knows}(x, x) \\ &{\it knows}({\it laura}(), {\it
john}()) \\ &{\it knows}({\it john}(), {\it laura}()) \end{align} The
results of these queries are, respectively, the following sets of
matchings: \begin{align} &\bigl\{ [{\it john}()/x, {\it john}()/y],\; [{\it
john}()/x,{\it laura}()/y] \bigr\} \\ &\bigl\{ [{\it john}()/x] \bigr\} \\
&\bigl\{\bigr\} \\ &\bigl\{[]\bigr\} \end{align} Note that a variable may
appear multiple times in a query.  In such a case, it implicitly requires
the corresponding subterms to be equal.  Also note that the \emph{empty
match} differs from \emph{no match}.
\end{example}

\todo{Conjunctive queries.}

\subsubsection{Updates and Revisions}

In general, belief bases have internal and external consistency
requirements.  Internally, beliefs should not contradict each-other.
Externally, beliefs should correspond to reality.

In AF-Raf, the situation is simpler.  There is no other meaning for beliefs
apart from how they affect the behaviour of the agent, through rules.
Thus, it is the agent's responsibility to detect inconsistencies.  In other
words, inconsistencies are detected by programs, not by the programming
language.

\begin{example} Suppose that a particular agent-oriented program makes the
convention that ${\it not}(x)$ means that `I do not hold belief~$x$'.  In
this case, the query ${\it not}(x)\&x$ detects inconsistencies.  The action
that the agent takes when it discovers an inconsistency could be to report
an error, but may be any action at all.  \end{example}

Similarly, the agent (and not the programming language) is responsible with
ensuring that its beliefs are consistent with the environment.

Once semantic concerns are left to the user program, the programming
language AF-Raf is free to adopt simple mechanisms for updating the belief
base.  The belief base is a set of beliefs, and its operations are those of
a set:  ${\bf adopt}(x)$ inserts belief~$x$ into the belief base, and ${\bf
forget}(x)$ removes the belief~$x$ from the belief base.  In case $x$~is
not a belief, ${\bf forget}(x)$ does not modify the belief base.

% >>>
\section{Types} % <<<

AF-Raf contributes to the AOP languages domain by being the first agent
oriented programming language to introduce algebraic data types and
type-checking. From a practical point of view, it tries to improve
upon the following scenario.

\paragraph{Bad Scenario.}

\rg{Needs a diagram; otherwise, hard to follow.}
Agent~$a$ sends message $\tau_{ab}$ to agent~$b$. Agent~$b$ stores a
subterm~$\tau_b$ of~$\tau_{ab}$ into its belief base. Later, when some
other condition is satisfied, agent~$b$ extracts a subterm~$\tau_{bc}$ of
term~$\tau_b$ and sends it to agent~$c$. Agent~$c$ extracts a
subterm~$\tau_c$ of term~$\tau_{bc}$ and stores it in its belief base.
Later, when some other condition is satisfied, agent~$c$ extracts a
subterm~$\tau$ of~$\tau_c$. Agent~$c$ expects $\tau$ to be an integer
literal, but instead $\tau$ is a string literal. Agent~$c$ notices the
problem, but many things happened since agent~$a$ sent an invalid message.

The goal of AF-Raf's dynamic typing is to notice such errors early.

For this purpose, each agent~$a$ has two attached types. The type $\delta_b(a)$
represents the type of $a$'s beliefs and the type $\delta_m(a)$ represents the
type of messages that $a$ can process. These types are selected when the agent
is created.

Whenever a belief $\tau$ is added to $a$'s belief base, Agent~Factory checks
whether that belief is in the set of terms defined by the $a$'s belief type;
that is, the expression $\tau:\delta_b(a)$ is evaluated.  Whenever a
message~$\tau$ is sent to agent~$a$, the expression $\tau:\delta_m(a)$ is
evaluated, to check whether that message complies to the type of messages that
agent~$a$ can process. In order to send a message to agent~$a$, the other agent
needs the agent identifier of agent~$a$. Agent identifiers typically consist of
a name and an IP address. To support the check for messages we add
$\delta_m(a)$ to the agent identifier of agent~$a$.

\subsection{Multi-Sorted Predicate Logic} \label{sec:multi-sorted} % <<<

The starting point in designing a type system is the observation that most
agent oriented programming languages use first order logic, more
specifically predicate logic terms, to represent beliefs and messages. As a
consequence we should look at how the introduction of types is managed in
mathematical logic, and recall basic notions necessary to our
understanding. And, also, introduce notational conventions used in AF-Raf.
Moreover, although these notions are simple, their definitions in
literature tend to have subtle but important differences.

A \emph{term} is a variable, or a function applied to other terms.
\begin{align}
\mathit{Term}\quad\tau &::= \omega \mid \phi \\
\mathit{Variable}\quad\omega &::= \nu \\
\mathit{Function}\quad\phi &::= \nu(\tau_1,\ldots,\tau_n) \\
\mathit{Name}\quad\nu
\end{align}
In AF-Raf, term names are strings.  Because term names uniquely identify a
variable or a function we will say ``the function~$\nu$'' rather than ``the
function with name~$\nu$.'' A term not containing variables is said to be
\emph{ground}.

\begin{remark}
Ground terms are essentially trees of strings.
\end{remark}

A multi-sorted logic has a set~$S$ of sorts.  We use the letter~$\sigma$ to
denote sorts.
\begin{align}
\sigma, \sigma_1, \sigma_2, \sigma_3, \ldots &\in S
\end{align}
Each function~$\nu$ has a signature
$(\sigma_1\times\cdots\times\sigma_n)\to\sigma$.  Function~$\nu$ is always
applied to $n$~terms whose sorts must be, respectively,
$\sigma_1$,~$\sigma_2$, \dots,~$\sigma_n$, and the resulting term has
sort~$\sigma$. We say that $\nu$~has \emph{arity}~$n$. \emph{Constants} are
functions with arity~$0$.

Typically, in mathematical logic, the set~$S$ of sorts and the function
signatures are required to satisfy further constraints. \textit{Bool} must
be one of the sorts. \emph{Formulas} are terms with sort \textit{Bool}.
The argument sorts ($\sigma_1$,~$\sigma_2$, \dots,~$\sigma_n$) either are
all \textit{Bool}, or none is \textit{Bool}.  If the argument sorts are
\textit{Bool}, then the result sort~$\sigma$ must also be \textit{Bool},
and $\nu$ is said to be a \emph{boolean connective}.  If the argument sorts
are not \textit{Bool} and the result sort~$\sigma$ is \textit{Bool}, then
$\nu$ is said to be a \emph{predicate}.  Agent~Factory has these
constraints and AF-Raf inherits them. In addition, AF-Raf uses an infix
notation (described later) for boolean connectives, letter strings starting
with uppercase for predicate names, and letter strings starting with
lowercase for other function names and for variable names.

\begin{remark}
\rg{I'm not sure about this remark. I think \emph{traditional} logicians
behave as described here.}
The definitions given here are in-between what applied computer scientists
tend to prefer and what pure logicians tend to prefer.  Computer scientists
work with \emph{expressions} (rather than terms and formulas) and do not
have constraints on function signatures with respect to booleans.
Logicians, on the other hand, do have these constraints and, moreover,
define terms in such a way that formulas are not terms, and predicates are
not functions. Moreover, logicians single out equality between terms as
being a special predicate.
\end{remark}

\rg{Maybe add some simple examples early on.}
In AF-Raf, a \emph{belief} is a boolean ground term; in AF-Raf, a
\emph{message} is a non-boolean ground term.

Algebraic data types offer an alternative way of defining sets of ground
terms. In multi-sorted predicate logic, the sort~$\sigma$ determines a set
of ground terms, namely those of the form $\nu(*)$, where $\nu$ has a
signature of the form $*\to\sigma$.  For example, the sort \textit{Bool}
determines the set of terms that are formulas.  With algebraic data types,
a type~$\delta$ is defined by a sequence of patterns, each of the form
$\nu(\delta_1,\ldots,\delta_n)$. For example, one could define the types
\textit{nat}, $e$, and~$o$ as follows.
\begin{align}
&\mathbf{type}\,\mathit{nat} =
      \mathit{zero}()
  \mid\mathit{succ}(\mathit{nat})
  \mid\mathit{add}(\mathit{nat},\mathit{nat}) \\
&\mathbf{type}\,e =
      \mathit{zero}()
  \mid\mathit{succ}(o)
  \mid\mathit{add}(e,e)
  \mid\mathit{add}(o,o) \\
&\mathbf{type}\,o =
      \mathit{succ}(e)
  \mid\mathit{add}(o,e)
  \mid\mathit{add}(e,o)
\end{align}
\rg{These things are presented as if they are completely standard. They're
not. I think they should be presented as part of the design of AF-RAF.}
We make the following observations.
\begin{enumerate}
\item
  Function names may appear in multiple definitions. For example,
  \textit{zero} appears in the definition of~$\mathit{nat}$ and also in the
  definition of~$e$. In particular, the term $\mathit{zero}()$ belongs both
  to the set of ground terms defined by the type~$\mathit{nat}$ and to the
  one defined by the type~$e$.
\item
  Function names may appear multiple times in the same definition. For
  example, \textit{add} appears twice in the definition of~$e$. (This
  property distinguishes our types from polymorphic
  variants~\cite{garrigue1998}.)
\item
  Type definitions may be recursive or mutually recursive. For example,
  $\mathit{nat}$~appears within the definition of~$\mathit{nat}$.
\end{enumerate}
The set of ground terms corresponding to type~$\mathit{nat}$ is the following.
\begin{equation}
\begin{aligned}
\{\,&\mathit{zero}(), \\
    &\mathit{succ}(\mathit{zero}()),
        \mathit{succ}(\mathit{succ}(\mathit{zero}())), \ldots \\
    &\mathit{add}(\mathit{zero}(), \mathit{zero}()),
        \mathit{add}(\mathit{zero}(), \mathit{succ}(\mathit{zero}())),
        \ldots \\
    &\mathit{succ}(\mathit{add}(\mathit{zero}(), \mathit{zero}())),
        \ldots \\
    &\ldots\, \}
\end{aligned}
\end{equation}
Moreover, $e$~and~$o$ partition~$\mathit{nat}$.

\begin{proposition}
Algebraic data types (as defined above) are strictly more expressive than
sorts. More precisely, (1)~all sets of ground terms that can be defined by
sorts can also be defined by algebraic data types, and (2)~there are pairs
of sets of ground terms that can be defined by algebraic data types but
cannot be defined by sorts.
\end{proposition}

\begin{proof}
For~(1), we define a type~$\delta(\sigma)$ for each sort~$\sigma$ as
follows: For each signature $\nu:(\sigma_1,\ldots,\sigma_n)\to\sigma$ we
add a pattern $\nu(\delta(\sigma_1),\ldots,\delta(\sigma_n))$. It is easy
to see, by structural induction, that the sort~$\sigma$ and the
type~$\delta(\sigma)$ define the same sets of ground terms.

For~(2), note that, given a fixed set of function signatures, the sets of
ground terms defined by distinct sorts are disjoint. In the previous
example, however, the sets of ground terms defined by the types
$\mathit{nat}$~and~$e$ have (at least) a common element, namely
$\mathit{zero}()$.
\end{proof}

Finally, let us note that there are sets of ground terms for which it is
undecidable whether a given term is an element.  (A classic result of
computability theory is that there are undecidable sets of bit-strings, and
sorted ground terms can encode bit-strings.) In particular, there exist
sets of ground terms that cannot be defined using algebraic data types.
\rg{These things are probably too cryptic, and should be explained more.}

% >>>
\subsection{Algebraic Data Types} % <<<
Algebraic data types are the natural target of pattern-matching functions.
When the type is parametric then the pattern-matching functions typically
support parametric polymorphism~\cite{AlgebraicDT09}, meaning that the
values are handled identically regardless of their actual type. 

In functional programming an algebraic data type is a kind of compound type
formed by combining other types. In this context an algebra of data types
denotes a set of types along with a collection of operations on that set of
types. Those operations are: the Cartesian product, also known as
multiplication or conjunction, and the discriminated union, also known as
sum or disjunction. The operations are used to create new types.


\paragraph{AF-Raf's Algebraic Data Types, Primitive Types, and Aliases.}

In AF-Raf, a \emph{belief} is a boolean ground term; in AF-Raf, a
\emph{message} is a non-boolean ground term.

Algebraic data types offer an alternative way of defining sets of ground
terms. In multi-sorted predicate logic, the sort~$\sigma$ determines a set
of ground terms, namely those of the form $\nu(*)$, where $\nu$ has a
signature of the form $*\to\sigma$.  For example, the sort \textit{Bool}
determines the set of terms that are formulas.  With algebraic data types,
a type~$\delta$ is defined by a sequence of patterns, each of the form
$\nu(\delta_1,\ldots,\delta_n)$. For example, one could define the types
\textit{nat}, $e$, and~$o$ as follows.
\[\mathbf{type}\,\mathit{nat} =
  \mathit{zero}()
  \mid\mathit{succ}(\mathit{nat})
  \mid\mathit{add}(\mathit{nat},\mathit{nat})\]
\[\mathbf{type}\,e =
  \mathit{zero}()
  \mid\mathit{succ}(o)
  \mid\mathit{add}(e,e)
  \mid\mathit{add}(o,o)\]
\[\mathbf{type}\,o =
  \mathit{succ}(e)
  \mid\mathit{add}(o,e)
  \mid\mathit{add}(e,o)\]

In Agent~Factory, string literals and integer literals are
also terms, and AF-Raf inherits this decision. AF-Raf has
predefined corresponding primitive types \textit{string} and
\textit{integer}. Moreover, any Java type is suitable to be an AF-Raf
primitive type.

Binary operators such as $+$ are also inherited from Agent~Factory. From
the point of view of type-checking, an expressions like $2+3$ is equivalent
to $+(2,3)$. Standard conventions on precedence and associativity are
obeyed.

More interestingly, AF-Raf has built-in the type $\mathit{integer}[\tau]$,
where $\tau$ is a formula that may use the special variable \textbf{this}.
For example, the type \[\mathit{integer}[\mathbf{this}\%2==0]\] defines the
set of ground terms $\{\,\ldots,-4,-2,0,2,4,\ldots\,\}$. Similarly, AF-Raf
has built-in the type $\mathit{string}[\tau]$. For example, the type
\[\mathit{string}[\mathbf{this}\;\mathbf{matches}\;\verb|"[a-z]+"|]\]
defines the set of string literals that match the given regular expression.

Enhanced with algebraic data types, the AF-Raf programming language has
virtually endless possibilities to define new types. The primitive types
are combined using discriminated union and cartesian product in the manner
described above.

To support the interaction with untyped languages, AF-Raf also has built-in
the type \textit{Any}, which defines the set of all ground terms.

Finally, one may define type aliases.
$\mathbf{type}\,\delta=\delta'$

Such aliases are especially useful when $\delta'$ is of the form
$\mathit{integer}[\tau]$ or $\mathit{string}[\tau]$, and $\tau$~is long.

\todo{Example with something like {\tt type t=int*t list}.}

% >>>
\subsection{Type Checking} % <<<
Type-checking is a program analysis that verifies the type safety of a
program. In the strict sense it verifies that the analysed program
will not have any type errors when executed. In the weak sense only
provides some amount of type safety. Type checking can be static or
dynamic. The static type checking can be based on explicit type decoration
or based on implicit type inferences.


\paragraph{Type Checking in AF-Raf.}
By \emph{type-checking} we mean deciding whether a given ground term~$\tau$
belongs to a given type~$\delta$. This is the operation performed, for
example, on the content of a message before sending it. We include here a
reference implementation for type-checking. We use the language OCaml, for
it is terse.

First, we need data structures for representing terms and types. Terms are
either non-primitive (abstract) or primitive (integers or strings).
\begin{verbatim}
    type term =
          | A of string * term list
          | I of int
          | S of string
\end{verbatim}
A non-primitive term has the shape $A(\nu,\Theta)$, where $\nu$~is a
function name and $\Theta$~is a list of terms. A primitive term simply maps
to the primitive types of the language in which the type-checker is
written.

Similarly, types are either abstract or primitive.
\begin{verbatim}
    type type_ =
          | AT of (string * type_ list) list
          | IT of (int -> bool)
          | ST of (string -> bool)
\end{verbatim}
An abstract type is defined by a list of patterns. Primitive types contain the
user-defined predicates. For example, the type
\[\mathit{integer}[\mathbf{this}\%2==0]\] is represented in the type-checker by
\begin{verbatim}
   IT (fun x -> x mod 2 = 0)
\end{verbatim}

Type-checking is then straightforward.
\begin{verbatim}
   let rec check v ts = match v, ts with
     | A (v, vs), AT ts ->
         let f (t, ts) =
           v = t && all2 check vs ts in
         List.exists f ts
     | I v, IT ts -> ts v
     | S v, ST ts -> ts v
     | _ -> false
\end{verbatim}
This code first checks that the term and the type are of the same kind. For
example, it returns false if the term is non-primitive but the type is
primitive. The interesting branch is the first one, which is taken when
both the term and the type are abstract. It checks whether there exists a
pattern (in \verb|ts|) that matches the term with the head~\verb|v| and the
sub-terms~\verb|vs|. The function~\verb|f| processes one pattern with the
head~\verb|t| and the sub-types~\verb|ts|. The heads must match and the
sub-terms must match, respectively, the sub-types. The function \verb|all2|
wraps the standard function \verb|List.for_all2| so that it returns
\verb|false| when the number of sub-terms is different from the number of
sub-types.
\begin{verbatim}
    let all2 f xs ys =
      try List.for_all2 f xs ys
      with Invalid_argument _ -> false
\end{verbatim}

We exhibited an algorithm, hence type-checking is decidable. As it is, the
function \verb|check| takes exponential time in the size of the term.
However, with memoization its complexity is~$O(mn)$, where $m$~is the
size of the term and $n$~is the size of all type definitions.
\rg{Maybe expand a bit here, with some example?}

% >>>
% >>>
\section{Roles} % <<<

The theory of organisations, from social sciences, studies how formal rules
of interaction enable groups of people to achieve common goals, on the
other hand, the theory of roles studies how individuals fit in multiple
informal groups. Both theories influenced prior attempts to structure agent
systems. Several agent-oriented methodologies and libraries incorporate the
concept of organisation and the concept of role.  However, there is little
work on designing language features with the specific purpose of supporting
large agent systems by organising agent
interaction~\cite{collier2005,DBLP:journals/entcs/BaldoniBT06,DBLP:conf/oopsla/RicciS11}.
The approach taken, regarding this language design problem, was to draw
inspiration from analogies between agent-oriented programming languages and
functional languages. For concreteness, we focused on
2APL~\cite{DBLP:journals/aamas/Dastani08} and Haskell~\cite{web:haskell}.

Haskell and 2APL are very different languages. We do not try to establish
any sort of formal connection between them, but rather to identify fruitful
high-level similarities. The task is akin trying to draw the Earth's
surface on paper---much easier to do locally than globally. We proceed,
therefore, by finding a contact point, seeing what it tells about its
surroundings, and then repeating. But, in order to do this, let's review
first the Haskell type classes and modules.

\subsubsection{Haskell Type Classes} % <<<

This section reviews Haskell type classes, because they later inspire our
notion of `role', and Haskell modules, so that it is clear they are very
different from type classes. We use sets and equality as examples because
we assume all readers are familiar with these basic mathematical concepts.
\rg{Well, in a thesis you can afford the space to explain even basic concepts.}

% modules
%  - small example
%  - information hiding and encapsulation
%    (exact impl may change; but it is ONE)
%  - separate compilation
%  - analogy with Java classes

The modular approach to programming implies separating the programs into
independent modules that perform logically distinct functions. Modules are
integrated into programs through interfaces. An interface define the
elements required and provided by the module. The advantage of modular
programming is that it allows one to write a module having little knowledge
of the code in another module, and, also, it allows modules to be
reassembled and replaced without reassembly of the whole
system~\cite{DBLP:journals/cacm/Parnas72a}.

In Haskell, modules are used to control name-spaces and to create abstract
data types. Moreover, a Haskell program can be seen as a collection of
modules, where the main module loads up the other modules. A module
contains functions, types and type classes that are related and serve a
common purpose.


Haskell modules are often used to implement abstract data types such as
sets.  To illustrate the main features of modules in little space, the code
in Figure~\ref{fig:haskell} is contrived.  The module \textit{Set} contains
the type~$T$ and the functions \textit{add}, \textit{has}, and
\textit{sub}. The \textbf{module} line hides \textit{sub} by not mentioning
it. The names and types of the exported functions \textit{add} and
\textit{has} are visible from outside the module, but their
implementations, which are to the right of~$=$, are hidden.  Similarly, the
type name~$T$ is visible from outside, but the value constructor~$V$ is
not. For example, the set $\{1,2\}$ may be represented by the value
$V[2,1,2]$, but this is not known to the users of the module \textit{Set}.
The names and types visible from outside constitute the module's
\emph{interface}.

\begin{figure}\footnotesize % <<<
\begin{verbatim}
-- built-in and standard library
class Eq b where
  eq :: b -> b -> Bool
instance Eq Int where ...
elem x [] = False
elem x (y:ys) = eq x y || elem x ys
all p [] = True
all p (x:xs) = p x && all p xs

-- file set.hs
module Set (T, add, has) where
  data T a = V [a]
  add (V s) x = V (x:s)
  has (V s) x = elem x s
  sub (V s) t = all (has t) s

  instance Eq a => Eq (T a) where
    eq s t = sub s t && sub t s
\end{verbatim}
\caption{Haskell type class \textit{Eq} and module \textit{Set}}
\label{fig:haskell}
\end{figure} % >>>

% type classes
%  - continue example
%  - ad-hoc polymorphism: multiple implementations with the same interface
%  - analogy with Java interfaces

The type class \textit{Eq} contains types whose values can be compared for
equality. To make a type belong to the class \textit{Eq} one must write an
instance declaration that provides an implementation for a function
named~$eq$. The \textbf{instance} declaration in module \textit{Set} says
that the type constructor~$T$ transforms members of \textit{Eq} into
members of \textit{Eq}. For example, $T(T\,\mathit{Int})$ is in \textit{Eq}
because \textit{Int} is in~\textit{Eq}.

In general, modules are responsible with information
hiding~\cite{DBLP:journals/cacm/Parnas72a} and encapsulation.  On the other
hand, type classes are an elegant mechanism to provide ad-hoc polymorphism,
also known as overloading~\cite{DBLP:conf/popl/WadlerB89}: The same name
refers to different implementations depending on the context.  Just as
Haskell type classes are different from Haskell modules, the roles we
introduce later are different from existing 2APL
modules~\cite{dblp:conf/prima/dastanims08}.
% >>>

\paragraph{The Functions as Messages Analogy} % <<<

A function call $f\,x$ is evaluated by `sending'~$x$ to $f$'s body,
evaluating the body, and then receiving the result. The process is
analogous to the exchange of a pair of messages between two agents. For
example, the role \textit{Calculator} could be specified as follows.
\begin{verbatim}
role Num a => Calculator a
  eval :: Expr a -> a
\end{verbatim}
An agent that plays the role $\mathit{Calculator}\,\mathit{Int}$ knows
how to compute expressions such as $(3+3)\times5$, given another agent
that plays the role $\mathit{Num}\,\mathit{Int}$.
\begin{verbatim}
role Num a
  add :: Pair a -> a
  multiply :: Pair a -> a
\end{verbatim}
An agent that plays the role $\mathit{Num}\,\mathit{Int}$ knows how to
compute basic operations on integers, such as $3+3$ and $6\times5$. The
types \textit{Expr} and \textit{Pair} constrain the content of messages.
\begin{verbatim}
data Expr a = Times (Expr a) (Expr a)
            | Plus (Expr a) (Expr a)
            | Ct a
data Pair a = MkPair a a
\end{verbatim}
Given a user agent~$u$, an agent~$c$ that plays
$\mathit{Calculator}\,\mathit{Int}$, and an agent~$n$ that plays
$\mathit{Num}\,\mathit{Int}$, the following is a possible exchange of
messages.\\ \\
$u\to c :
  \mathit{eval}(\mathit{call}(n), \mathit{Times}(
    \mathit{Plus}(\mathit{Ct}(3),\mathit{Ct}(3)),\mathit{Ct}(5)))\\
c\to n : \mathit{add}(\mathit{call}(), \mathit{MkPair}(3, 3))\\
n\to c : \mathit{add}(\mathit{return}(), 6)\\
c\to n : \mathit{multiply}(\mathit{call}(), \mathit{MkPair}(6, 5))\\
n\to c : \mathit{multiply}(\mathit{return}(), 30)\\
c\to u : \mathit{eval}(\mathit{return}(), 30)$\\

In general, $f::a\to b$ says that the agent understands messages of the
form $f(\mathit{call}(\alpha_1,\ldots,\alpha_n),x)$ and eventually replies
to each of them with a message of the form $f(\mathit{return}(),y)$. Here,
$\alpha_1$, \dots,~$\alpha_n$ are (addresses of) other agents, $x$~is a
value of type~$a$, and $y$~is a value of type~$b$.

The analogy so far is already fruitful. The content of 2APL messages is a
(ground) term or an atomic formula. Since 2APL is built on top
JADE~\cite{DBLP:books/sp/map2005/BellifemineBCP05}, the message content may
also be declared as part of an ontology. However, if we would show the JADE
ontology for arithmetic expressions we would run over the page limit.
Contrast with the three short lines used here to define $\mathit{Expr}\,a$.
The definition is not only short and readable, but also polymorphic in the
type~$a$ of the constants, and rooted in the theory of algebraic data types
(see, for example, \cite{DBLP:conf/ctcs/Hagino87}).

\begin{figure}\footnotesize % <<<
\begin{verbatim}
agent foo plays Calculator Int(n)
  R-rules:
    eval(Ct(x)) <- x
    eval(Times(x, y)) <-
      n.multiply(MkPair(this.eval(x), this.eval(y)))
    eval(Plus(x, y)) <-
      n.add(MkPair(this.eval(x), this.eval(y)))
\end{verbatim}
\caption{Implementing a role in 2APL}\label{fig:roleimpl2APL}
\end{figure} % >>>

The analogy is not perfect. The earlier declaration for the role
$\mathit{Calculator}\,a$ is superficially similar to the following type
class declaration.
\begin{verbatim}
class Num a => Calculator a
  eval :: Expr a -> a
\end{verbatim}
This declaration reads ``a type~$a$ that is a member of the class
\textit{Num} is also a member of class \textit{Calculator} provided there
exist a function \textit{eval} with the proper type.'' In contrast, the
earlier role declaration reads ``an \emph{unnamed} agent plays role
$\mathit{Calculator}\,a$ if it answers to messages
$\mathit{eval}(\mathit{call}(n),\ldots)$ by messages
$\mathit{eval}(\mathit{return}(),\ldots)$, where $n$~is an agent that plays
$\mathit{Num}\,a$.'' Here $a$~is a type variable.  When implementing a role
the agent must be named, as seen in Figure~\ref{fig:roleimpl2APL}. Because
\textit{foo} plays $\mathit{Calculator}\,\mathit{Int}$, the agent
interpreter creates a goal $\mathit{eval}(m,\mathit{call}(n),x)$ for all
messages with shape $\mathit{eval}(\mathit{call}(n),x)$ that come from some
agent~$m$.  One could handle these goals using 2APL's PG-rules.
\begin{verbatim}
eval(m, call(n), Ct(x)) <- true |
  send(m, role, eval(return(), x))
\end{verbatim}
The first R-rule in Figure~\ref{fig:roleimpl2APL} does exactly the same,
but is more compact. The other two R-rules, however, are much more
cumbersome to simulate with the other kinds of rules. The main reason is
that the notation $n.\mathit{add}(x)$ hides sending a message
$\mathit{add}(\mathit{call}(),x)$ to agent~$n$, waiting for a reply
$\mathit{add}(\mathit{return}(),y)$, and extracting~$y$. The (goal) query
of an R-rule may only be an atom; the right side of an R-rule is an
expression that is evaluated as described and whose result is sent as a
message.  This is a rough and informal sketch of the intended semantics
that needs to be made precise.

Note that two agents \textit{foo} and \textit{bar} may be instances of the
same 2APL module, yet only \textit{foo} plays the role
$\mathit{Calculator}\,\mathit{Int}$. Also, note that an agent may be
declared as playing a role without having access to its implementation. In
fact, it may be that the basic behaviour of the agent is programmed in a
different language than 2APL. Such flexibility helps code reuse.

In summary, the vague and informal intuition that a function is like a pair
of messages, one carrying the arguments and one carrying the result, led us
to two interesting observations. First, algebraic data types are convenient
for describing the content of messages. We expect to see fewer runtime
errors once messages are typed. Second, we developed a notion of role in
the context of the 2APL language. These roles have certain similarities
with existing 2APL modules and with Java interfaces, but are nevertheless
distinct concepts.

% >>>
% >>>
\section{Sessions} % <<<

Agents play roles and we wrote roles much like Haskell type
classes, so it would seem that the analogue of Haskell types are
agents. A Haskell type class is a set of Haskell types; a role is a
set of agents that play the role. I would like to explore where
does the intuition ``types as agents'' lead.

\begin{figure}\footnotesize % <<<
\begin{verbatim}
session ComputeBasicOperation(a, b)
  a -> b: Pair Int
  b -> a: Int
session ComputeExpression(a, b, c)
  c -> a: Expr Int
  repeat ComputeBasicOperation(a, b)
  a -> c: Int
\end{verbatim}
\caption{Sessions for AF-Raf}\label{fig:sessions}
\end{figure} % >>>

{\def\l#1->#2:#3<#4>{\mathtt{#1}\to\mathtt{#2}:#3\langle\mathsf{#4}\rangle}
We read $f::a\to b$ as ``message $f$ is sent by agent~$a$ to agent~$b$.'' A
type class lists several function signatures, so its natural analogue is a
list of messages together with their endpoints. It turns out that such a
list is very similar to the global types that describe multiparty sessions
in the context of $\pi$-calculus.

Session types represent a type foundation for structured communication
centered programming, that facilitate communication safety, progress and
fidelity \cite{dblp:conf/popl/hondayc08}. The research in session types
initiated in process calculi and programming languages with binary session,
and it was extended to multiparty asynchronous session types. A session is
a series of interactions which serve as a unit of communication. The
structure of a session is abstracted as a type. Such a global type is like
a shared agreement among communication peers, and represents the basis for
efficient type checking through its projection onto individual peers.

 

Here is an example of such a type from 
Honda et al.~\cite{dblp:conf/popl/hondayc08}:\\
$\mu\mathbf{t}.$
  $\l DP->K:d<bool>. $\\
  $\l KP->K:k<bool>. $\\
  $\l K->C:c<bool>.\mathbf{t}$

This type means that process \texttt{K} receives two booleans, one from
\texttt{DP} through channel~$d$ and one from \texttt{KP} through channel~$k$,
then sends a boolean to~\texttt{C} through channel~$c$, and the whole process
repeats. The main differences are that we have agents, rather than processes,
and there are no named channels. We would therefore like to write code like
that in Figure~\ref{fig:sessions}.  These sessions are a global description of
the messages that should flow within an agent system. When we project
$\mathit{ComputeExpression}(a,b,c)$ on agent~$a$ we obtain the role
$\mathit{Calculator}\,\mathit{Int}$ from the previous section.}

In agent-oriented methodologies it is standard to say that ``an agent plays
a role within an organisation,'' and therefore organisations are somehow
collections of interacting roles, just as the sessions above are in a way
putting together interacting roles. Similarly, in multiparty session types
there is a notion of projecting global types onto local types. The
essential advantage of session types is that the projection can be done
automatically. By imitating session types, we check automatically that the
projection of a certain session on a certain agent matches a certain role,
which the agent implements.

In summary, the vague and informal intuition that a Haskell type is
sometimes like an agent led us to the proposal of specifying global
interactions in agent-oriented programming languages in terms of sessions.
%Moreover, it seems reasonable to expect that a precise link between these
%session and the roles proposed in the previous section can be found.
% >>>

\section{The Calculator Example} % <<<

This section introduces a number of example programs that were contrived to
illustrate AF-Raf's functionality.

The Calculator example demonstrates the power of AF-Raf operations, as well
as being a good opportunity to understand how roles are implemented.
Figure~\ref{fig:calc-roles} presents the interrelated types, roles and
sessions associated with the Calculator example. First the types
\textit{pair} and \textit{expression} are defined. The type \textit{pair}
<x> is defined such that a value of \textit{pair} of type x is constructed
by pairing two values of type x. The type \textit{expression} <value> is
defined using discriminated union meaning that a value of
\textit{expression} <value> can be either one of the following: a value of
type <value>, an addition of two \textit{expression} <value>, a subtraction
of two \textit{expression} <value>, a multiplication of two
\textit{expression} <value>, or a division of two \textit{expression}
<value>. There are two roles defined: the \textit{Calculator} and the
\textit{BinaryCalc}. The \textit{Calculator} is responsible with evaluating
expressions to a value of type <value>, while the \textit{BinaryCalc} is
responsible with executing binary operations on values of type <value>. The
session \textit{EvaluateExpression} <value> articulates the interaction
between three types of agents that work together in order to evaluate
expressions of type <value>. These agent types are: any type,
\textit{Calculator} <value>, and \textit{BinaryCalc} <value>. The
interaction pattern is described in a recursive manner. The interaction is
initiated by an agent of any type that asks the \textit{Calculator} agent
to evaluate an \textit{expression} <value>. The \textit{EvaluateExpression}
session is run recursively by the \textit{Calculator} agent until a pair of
<value> can be passed over to the \textit{BinaryCalc} or an expression is
evaluated to a value.

Figure~\ref{fig:bcalc-rules} presents the implementation of a binary
calculator agent, an instance of the BinaryCalc role.
Figure~\ref{fig:calc-rules} is a partial implementation of the expression
calculator agent, an instance of the Calculator role, more exactly it
presents the implementation of addition expression evaluation. Both binary
calculator and calculator implementations are using only rules.

Let's look first at the binary calculator~\ref{fig:bcalc-rules}. There are
four actions defined, for each basic binary operation: intadd(x, y),
intsub(x, y), intmul(x, y), intdiv(x, y). Also, each binary operation has
associated a rule saying that upon receiving a message asking to perform
that operation on a pair of integers, the agent should first delete the
message, compute the result, and send the result back to the requester.
Each message contains the sender's details, the details regarding the
operation to be executed, and the pair of integers on which the operation
is performed. The operation details contain the status of the operation,
the call id, and the session id. The status of the operation could be
either call or return, and in this particular case the message contains a
call.

The calculator agent implementation in figure~\ref{fig:calc-rules} is not
complete, only one of the four cases of evaluation is illustrated, namely
the addition. The addition is implemented using four rules. The first rule
states that upon receiving a message to evaluate an expression that matches
add(x, y), the calculator should recursively evaluate expression x, and save
the other information from the message to be able to continue from there when
the result from evaluating expression x is ready. The second rule states
that upon receiving a message with the result of the evaluation of
expression x, the calculator should evaluate expression y, and continue
from there. The third rule states that upon receiving a message with the
result of evaluation of expression y, the calculator should send a message
to the binary calculator agent to perform the binary operation on the
argument send, in this case a pair of integers, and then continue from
there. The forth rule states that upon receiving the result from the
binary calculator, the calculator agent should send back to the first agent
the result received from the binary calculator as the result for the
evaluation of expression add(x, y).

Figure~\ref{fig:calc-op} presents a much simpler and shorter implementation
of agents BinaryCalc and Calculator, using operations. In the Binary
calculator implementation each rule used to model a basic binary operations
is replaced by a shorter and more concise operation. In the calculator
implementation one single operation replaces four rules, so the effect is
much more visible.

\begin{figure}\footnotesize % <<<
\begin{verbatim}
type pair<x> = pair(x, x)

type expression<value> =
  | val(value)
  | add(expression<value>, expression<value>)
  | sub(expression<value>, expression<value>)
  | mul(expression<value>, expression<value>)
  | div(expression<value>, expression<value>)

role Calculator<value> {
  eval : expression<value> -> value;
}

role BinaryCalc<value> {
  add : pair<value> -> value;
  sub : pair<value> -> value;
  mul : pair<value> -> value;
  div : pair<value> -> value;
}

session EvaluateExpression<value>(A : any,
                                  B : Calculator<value>,
                                  C : BinaryCalc<value>) {
  expression<value> : A -> B;
  repeat EvaluateExpression(B, B, C); // actually, 0 or 2 times
  pair<value> : B -> C;
  value : C -> B;
  value : B -> A;
}
\end{verbatim}
\caption{Calculator: types, roles and session}
\label{fig:calc-roles}
\end{figure} % >>>

\begin{figure}\footnotesize % <<<
\begin{verbatim}
include stdio
include CalcTypes

action intadd(x, y) = raftest.IntegerAddAction;
action intsub(x, y) = raftest.IntegerSubAction;
action intmul(x, y) = raftest.IntegerMulAction;
action intdiv(x, y) = raftest.IntegerDivAction;

agent BinaryCalcInt : BinaryCalc<integer> {
  rule Message(other, op(call(add(), cid, sid), pair(x, y))) {
    -Message(other, op(call(add(), cid, sid), pair(x, y)));
            z := intadd(x, y);
    send(other, op(return(add(), cid, sid), z));
  }
  rule Message(other, op(call(sub(), cid, sid), pair(x, y))) {
    -Message(other, op(call(sub(), cid, sid), pair(x, y)));
            z := intsub(x, y);
    send(other, op(return(sub(), cid, sid), z));
  }
  rule Message(other, op(call(mul(), cid, sid), pair(x, y))) {
    -Message(other, op(call(mul(), cid, sid), pair(x, y)));
            z := intmul(x, y);
    send(other, op(return(mul(), cid, sid), z));
  }
  rule Message(other, op(call(div(), cid, sid), pair(x, y))) {
    -Message(other, op(call(div(), cid, sid), pair(x, y)));
            z := intdiv(x, y);
    send(other, op(return(div(), cid, sid), z));
  }
}
\end{verbatim}
\caption{BinaryCalc agent implementation}
\label{fig:bcalc-rules}
\end{figure} % >>>

\begin{figure}\footnotesize % <<<
\begin{verbatim}
include stdio
include CalcTypes

action intadd(x, y) = raftest.IntegerAddAction;
action intsub(x, y) = raftest.IntegerSubAction;
action intmul(x, y) = raftest.IntegerMulAction;
action intdiv(x, y) = raftest.IntegerDivAction;

agent ExprCalc(binaryCalc : BinaryCalc<value>) : Calculator<value> {
  rule Message(other, op(call(eval(), cid, sid), add(x, y))) & Iam(self) {
    -Message(other, op(call(eval(), cid, sid), add(x, y)));
                xcid := newcall;
    +Continue(other, cid, xcid, add1(y));
    send(self, op(call(eval(), xcid, sid)), x);
  }

  rule Message(self, op(return(eval(), xcid, sid), ctx))
                & Iam(self)
                & Continue(other, cid, xcid, add1(y))
  {
    -Continue(other, cid, xcid, add1(y));
    -Message(self, op(return(eval(), xcid, sid), ctx));
    ycid := newcall;
    +Continue(other, cid, ycid, add2(ctx));
    send(self, op(call(eval(), ycid, sid), y));
  }

  rule Message(self, op(return(eval(), cid, sid), cty))
                & Iam(self)
                & Continue(other, cid, ycid, add2(ctx))
  {
    -Message(self, op(return(eval(), cid, sid), cty));
    -Continue(other, cid, ycid, add2(ctx));
    zcid := newcall;
    +Continue(other, cid, zcid, add3());
    send(binaryCalc, op(call(add(), zcid, sid), operands(ctx, cty)));
  }

  rule Message(binaryCalc, op(return(add(), zcid, sid), z))
                & Continue(other, cid, zcid, add3())
  {
    -Message(binaryCalc, op(return(add(), zcid, sid), z));
    -Continue(other, cid, zcid, add3());
     send(other, op(return(eval(), cid, sid), z));
  }

...

}

\end{verbatim}
\caption{Calculator: addition expression evaluation}
\label{fig:calc-rules}
\end{figure} % >>>

\begin{figure}\footnotesize % <<<
\begin{verbatim}
include stdio
include CalcTypes

action intadd(x, y) = raftest.IntegerAddAction;
action intsub(x, y) = raftest.IntegerSubAction;
action intmul(x, y) = raftest.IntegerMulAction;
action intdiv(x, y) = raftest.IntegerDivAction;

agent BinaryCalcInt : BinaryCalc<integer> {
  operation add(pair(x, y)) = { return intadd(x, y); }
  operation sub(pair(x, y)) = {return intsub(x, y); }
  operation mul(pair(x, y)) = {return intmul(x, y); }
  operation div(pair(x, y)) = {return intdiv(x, y); }
}

agent ExprCalc(binaryCalc : BinaryCalc<value>) : Calculator<value> {
  operation eval(add(x, y)) {
    return binaryCalc.add(self.eval(x), self.eval(y));
  }
  operation eval(sub(x, y)) {
    return binaryCalc.sub(self.eval(x), self.eval(y));
  }
  operation eval(mul(x, y)) {
    return binaryCalc.mul(self.eval(x), self.eval(y));
  }
  operation eval(div(x, y)) {
    return binaryCalc.div(self.eval(x), self.eval(y));
  }
}
\end{verbatim}
\caption{BinaryCalc and Calculator implementation with operations}
\label{fig:calc-op}
\end{figure} % >>>


% >>>
\section{Related Work}\label{ch:related} % <<<


Ricci  and Santi developed a novel programming language, called simpAL, that
supports types and type checking~\cite{DBLP:conf/promas/RicciS12}. SimpAl
draws inspiration from the Agent and Artifacts
model~\cite{DBLP:conf/atal/RicciVO07} and from the Jason programming
language, which is based on the Belief, Desire, Intention model. SimpAL is
conceived as an extension to the Java object oriented programming language
with a separate agent abstraction layer.

Even though this language is very different from AF-Raf, there are a
number of overlapping concepts. SimpAL imports a social view of computation
where agents playing different roles constitute an organisation, and work
within a shared environment. The main building blocks of this model are
agents, the autonomous components, and artifacts, the environmental
components that provide functionality or services. The agent has a belief
base, some tasks to perform and a set of plans to utilise in order to
acomplish the tasks. SimpAl roles are very similar to AF-Raf roles in that
they constitute a set of tasks types, where tasks are the analogous to the
AF-Raf operations.

The roles are organised in a slightly different manner compared to the
AF-Raf's session type model, which can be seen as a kind of protocol
prescribing the agent communication. In simpAL organisation models are used
to define organisation types, that bring together related roles and
artifact in coherent units named workspaces. Organisation types are
implemented by concrete organisations. SimpAL organisation types do not put
any constraint on the order in which interactions between agents occur, as
AF-Raf session types do, instead they simply group together related roles
and artifacts working in a similar way to the object oriented programming
interfaces.

A number of static type checks are performed. At the role level it is
checked that for each task type there exists at least one plan, and for
each plan it's checked that the type of the sent messages are only the ones
defined in the task type. Also it's checked that the assignment of tasks
corresponds to the task types defined in the role, and that the types of
the messages sent to an agent correspond to that the agent can understand.

Errors about beliefs that are checked include finding, inside the plans,
beliefs that are not declared, and beliefs that are assigned with
expressions of the wrong type.

Similar checks are performed at the environment level. The usage interface
represents the type of artifact, and contains the observable properties and
the operations provided by the artifact. Artifact template is the actual
artifact implementation. On the agent side are checked errors regarding
artifact operations and its observable state. For each action in the plan
there must be a matching operation in the artifact's interface.  Likewise,
for each update belief statement regarding an artifact property there must
exist a matching observable property in the usage interface. On the
environment side is checked that the artifact templates match the
corresponding usage interfaces. First, checks are performed to ascertain
that for each operation declared in the interface there exists an
implementation in the template, also for any of the observable properties
in the template there exists a declaration in the corresponding interface
with a compatible type.

At the organisation level it is checked that each workspace, artifact, and
agent types declared in an organisational model is actually implemented in
the implementation of the organisation.
\\~\\

Subsequent to my proposal to integrate session types with agent programming
languages, global session types were exploited further in the context of
multiagent systems~\cite{DBLP:conf/dalt/AnconaDM12} to specify multiparty
interactions and verify their correctness. They were used for automatic
generation of self-monitoring agent systems implemented in Jason. The
solution suggested was to generate a monitor agent that checks at runtime
that the ongoing conversation is correct conform to the global session
type. In this context, global session types are viewed as states, on which
several transition steps are possible. Transitions occur with each
successfully sent message.

In order to be able to generate the agent monitor, the developer needs to
provide Prolog rules to define global session types, to specify possible
transitions to different states upon message sending, and to specify the
message content types. As well, a number of changes to the code of the
participating agents need to be done to support the integration with the
monitor agent.

There's only one monitor agent that keeps track of all messages and checks
if there exists a new state that can be reached from the current state. If
the transition is possible the monitor agent allows the sending action by
acknowledging it to the sending agent. If such a transition is not
possible the protocol fails and there's no possibility of self recovery.
The developer has to change either the agent code or the global session
type's specification in order to fix the error.
\\~\\

A proposed extension to the ALPHA programming language that employs roles
as a run-time construct dates back to 2005\cite{Collier_arole-based}. An
ALPHA agent is able to play multiple roles and be aware of them at
run-time. The role construct, also called a role template, consists of a
variable named role identifier, a set of commitment rules, that defines the
behaviour, and a set of trigger conditions, that cause the activation of
the role. Each instance of a role template has to have a unique role
identifier. A mechanism for inheritance of roles is provided in the form of
a possibility to  extend the agent behaviour by adding more commitment
rules and trigger conditions.  Also, a mechanism for composition and
aggregation of roles is outlined.

However, there is no infrastructure to support the location and enactment
of roles at run-time, and also there is no mechanism to model the
interaction between roles. \\~\\

The 3APL programming language\cite{DBLP:conf/aose/DastaniRHDM04} has a
formal proposal to include roles as first order constructs. An agent role
is seen as a set of behaviour rules, expected objectives, and information
resources. An agent can assume multiple roles at a time, but only one of
them can actually be active, from here the distinction between role
enactment and role activation. A subset of consistent role, which do not
contradict each other, are grouped together in agent types. At the end role
have not been introduced in 3APL, and the decision was carried further to
2APL\cite{DBLP:journals/aamas/Dastani08}, 3APL's successor, where modules
are used as a substitute for roles. We argued that modules and roles have
different functionalities within a programming language, and both are
useful construct that cannot be reduced one to another.\\~\\

%Roles and norms for programming agent organisations. Tinnemeier 2009

% >>>
% >>>

\chapter{Language Definition}\label{ch:langdef} % <<<

\section{Overview}\label{sec:langdef.overview} % <<<

This chapter presents the AF-Raf language in detail, covering its syntax,
typing, and semantics.

% >>>
\section{AF-Raf Syntax} % <<<

AF-Raf, like most languages, has an abstract grammar and a concrete
grammar. The abstract grammar describes the structure that programs must
have, without much regard to how they are written down. The concrete
grammar takes into account that the program must be represented as a
unidimensional string of characters. For example, the abstract grammar
would not say how should items in a list be delimited. Thus, the abstract
grammar is essentially a concrete grammar from which the non-essential
features have been elided.

This section describes the abstract grammar and briefly gives the concrete
grammar.

The abstract grammar of AF-Raf~(\autoref{fig:Abstract-Grammar}) is taken
directly from the reference implementation of AF-Raf.  In general, there
are two equivalent ways to understand abstract grammars. First, an abstract
grammar is a concise way to specify a set of trees, each of which
represents a program. Second, an abstract grammar is simply a set of data
types. The relation between the two ways to view abstract grammars is
simple --- the trees are instances of the data types. Since the reference
implementation of AF-Raf is in Java, the abstract grammar is a set of Java
classes. \autoref{fig:Abstract-Grammar} uses a notation less verbose than
Java code. For example,
\begin{align}
\texttt{AgentRole = String name, Type role;}
\end{align}
means that the class {\it AgentRole\/} has two fields, {\it name\/} and
{\it role}; of course, it also has a constructor, getter methods, and other
paraphernalia required by Java.

\begin{figure}\footnotesize % <<<
\begin{verbatim}
AgentDeclaration =
  String design,
  ImmutableList<AgentRole> arguments,
  ImmutableList<Type> roles,
  ImmutableList<Rule> rules;

RoleDeclaration =
  String role,
  ImmutableList<String> formals,
  ImmutableList<Operation> operations;

Operation =
  String name,
  Type input,
  Type output;

AgentRole =
  String name,
  Type role;

Type =
  String name,
  ImmutableList<Type> actuals;

TypeDeclaration =
  String name,
  ImmutableList<String> formals,
  ImmutableList<TypeBranch> branches;

TypeBranch :> Type, ValueConstructor;

ValueConstructor =
  String name,
  ImmutableList<TypeBranch> arguments;

ActionDeclaration =
  Function function,
  QualifiedId javaClass;

QualifiedId =
  QualifiedId path,
  String id;

Rule =
  IFormula condition,
  IPlanStep statement;
\end{verbatim}
\caption{AF-Raf Abstract Grammar}
\label{fig:Abstract-Grammar}
\end{figure}
% >>>

Each {\it AgentDeclaration\/} introduces a new agent design, by specifying
a design name, a list of arguments, a list of roles, and a list of rules.
Each argument ({\it AgentRole\/} in the abstract syntax) consists of a name
and a role. Suppose the design arguments have names $n_1$,~$n_2$, $\ldots$
with corresponding roles $r_1$,~$r_2$,~$\ldots$ Then, in order to
instantiate the design, one will need to provide the addresses of agents
that play the roles $r_1$,~$r_2$,~\dots; the rules of the design may refer
to these addresses using the names $n_1$,~$n_2$,~$\ldots$ (To a first
approximation, the design name is analogous to a Java class name, and the
arguments are analogous to the formal arguments of a Java constructor.)

\begin{example}
The following declares a design named {\it ExprCalc}.
\begin{verbatim}
  agent ExprCalc(binaryCalc : BinaryCalc<value>)
    : Calculator<value> { /* ... */ }
\end{verbatim}
The rules, which are omitted above, may use the identifier {\it
binaryCalc\/} in places where the address of an agent is expected.
Instances of the design {\it ExprCalc\/} are agents that play the role $\it
Calculator\langle value\rangle$, given that the address {\it binaryCalc\/}
belongs to an agent that plays the role $\it BinaryCalc\langle
value\rangle$. The {\it value\/} type could be integers, reals, or
something else.

In this example the name of the design is {\it ExprCalc\/}; the list of
arguments contains one argument, {\it binaryCalc\/} and its role; the list
of roles contains one role, $\it Calculator\langle value\rangle$; finally,
the list of rules is omitted.
\end{example}

There are two kinds of types in AF-Raf: types of values and types of
agents.  For brevity, we may refer to them as types and roles. Thus, `type'
is used often as a shorthand for `type of value'. From the point of view of
the syntax, AF-Raf types and roles are sometimes similar and sometimes
different. In general, in a programming language, one needs a way to
declare types, and a way to refer to types. In AF-Raf, the syntax for
referring to a type is the same as the syntax for referring to a role.

References to roles and references to types are represented in the abstract
grammar by one class, {\it Type}, because of their similarity. Both roles
and types have names, and both may be parametrized by types. Thus, a
reference to a role/type consists of the name of the role/type together
with the types of the~parameters.

\begin{example}
In the previous example, $\it BinaryCalc\langle value\rangle$ and $\it
ExprCalc\langle value\rangle$ refer to roles.  The syntax for referring to
types is very similar: for example, $\it List\langle int\rangle$ and {\it
int}. (The type {\it int\/} is built-in.) Type variables, like {\it value}
above, always refer to types, never to roles.
\end{example}

On the other hand, declarations of roles are quite different from
declarations of types. The latter are simpler.

A {\it TypeDeclaration\/} consists of the name of the (possibly generic)
type being defined, the generic type variables (if any), and at least one
{\it TypeBranch}. Each {\it TypeBranch\/} identifies a set of terms by
providing a pattern that the terms should match. The pattern looks exactly
like a term (which is a tree of strings), except that leaves may refer back
to types. These references to a type may be to a type defined above, to a
type defined below, to the type defined now, or to the generic type
variables of the type being defined. In other words, type declarations may
be mutually recursive.

\begin{example}
Given the declaration
\begin{align}
&{\bf type}\ {\it expr}\langle v\rangle
  = {\it v}
  \mid {\it add}({\it expr}\langle v\rangle, {\it expr}\langle v\rangle)
\end{align}
the type ${\it expr}\langle{\it int}\rangle$ contains values $0$, $1$, $2$,
${\it add}(1,{\it add}(2,0))$, and so on. Note that the type ${\it
expr}\langle{\it expr}\langle{\it int}\rangle\rangle$ contains exactly the
same values as ${\it expr}\langle{\it int}\rangle$: one could refer to the
same set of values using different (type) names.
\end{example}

Each {\it RoleDecaration\/} introduces a new role, by specifying its name,
a list of arguments, and a list of operations. The arguments are type
variables, which play a similar role as they do in type declarations. The
arguments may be used inside operations in places where a reference to a
type is expected. Each {\it Operation\/} has a name, an input type, and an
output type.

\begin{example}
An array is a data structure that maps integers to some arbitrary data
type. It can be expressed as an AF-Raf role, as follows.
\begin{verbatim}
  role Array<value> {
    get : int -> value;
    set : pair<int, value> -> unit;
  }
\end{verbatim}
The code above assumes the existence of two types, {\it pair\/} and {\it
unit}, which are pre-defined as follows.
\begin{verbatim}
  type pair<a,b> = pair(a,b);
  type unit = unit();
\end{verbatim}
\end{example}

To interact with the environment, Af-Raf uses sensors and actions.
Sensors do not appear explicitly in the abstract syntax, because they
are very simple: Each sensor is a {\it QualifiedId}, which is the name of a
Java class.

\begin{example}
To activate a sensor, one simply has to give the name of the Java class
that implements is.
\begin{verbatim}
sensor com.agentfactory.raf.RafStandardSensor$Message;
\end{verbatim}
Several such declarations appear in the {\tt stdio.raf} file, which users
may include.
\end{example}

The declaration of an action is only marginally more involved.  Each
{\it ActionDeclaration\/} consists of a function and the name of a Java
class. Whenever the function is used as a statement inside a rule, the
corresponding Java class will be used to actually carry out the action.

\begin{example}
The file {\tt stdio.raf} includes the following action declaration, to help
debugging.
\begin{verbatim}
  action println(s) =
    com.agentfactory.raf.RafStandardAction$Println;
\end{verbatim}
If {\tt stdio.raf} is included, then the statement {\tt println("test")}
would print the string `test' to the console.
\end{example}

Each {\it Rule\/} has a query\slash condition and a body, which is usually
a composed statement. The formula is obtained by connecting several
predicates with boolean operators; that is, there are no explicit
quantifiers. Both {\it IFormula\/} and {\it IPlanStep\/} are interfaces
reused from the AgentFactory infrastructure. AF-Raf adds a few new classes
that implement {\it IPlanStep\/}; in other words, AF-Raf has some new
types of statements. These are presented later (\autoref{sec:opsem}).

\begin{example}
In AF-Raf, it is typical to have one agent named that is instantiated at
the start of the execution. Its rules are tasked with instantiating other
agents. An example of such a rule follows.
\begin{verbatim}
  rule Start("pingpong") {
        -Start("pingpong");
        println("Main: Starting pingpong scenario.");
        new PingPong "Alice" {Monitoring("Bob")};
        new PingPong "Bob" {Friend("Alice")};
  }
\end{verbatim}
The query checks whether the belief base contains the fact ${\it
Start}(\text{\tt``pingpong''})$. If so, this fact is removed from the
belief base, a message is printed to the console, and two agents are
instantiated. One agent is named Alice, the other is named Bob. Both have
the design {\it PingPong}. The initial belief of Alice is ${\it
Monitoring}(\text{\tt ``Bob''})$; the initial belief of Bob is ${\it
Friend}(\text{\tt``Alice''})$.
\end{example}

\todo{Abstract grammar for sessions.}

Let us now move from the abstract grammar to the concrete grammar.  The
concrete grammar has three main parts: the top-level structures
(\autoref{fig:grammar-top-level}), expressions
(\autoref{fig:grammar-expr}), and statements
(\autoref{fig:grammar-statements}). The grammar is almost identical to the
one used in the reference implementation. The upside is that it is
extensively tested and likely to be correct. The downside is that the
grammar productions are constrained such that the grammar is (almost)
$LL(k)$, which could make the grammar more complicated than needed in
theory. (The parser generator used is~ANTLRv3.)

An AF-Raf program consists of a set of scripts, each stored in a file with
the extension {\tt raf}.  A {\it script\/} is composed of a number of
include directives and a number of declarations. (`EOF' stands for `end of
file'. The order of declarations is semantically irrelevant.)

\begin{figure}\footnotesize % <<<
\begin{verbatim}
script: directive* EOF;
directive: include | declaration;
declaration:
  java_binding | agent | role | session | type_declaration;
include: 'include' qualified_id;
java_binding: action | sensor;
agent: 'agent' UID ('(' agent_role_list? ')')?
       (':' type_list)? '{' rule_list '}';
rule: 'rule' formula sequence_statement;
role: 'role' id ('<' type_list '>')? '
      {' (operation_type ';')* '}';
session: 'session' id ('<' type_list '>')?
         '(' agent_role_list ')' '{' (session_step ';')* '}';
session_step: simple_session_step
            | nested_session_step;
simple_session_step: type ':' id '->' id;
nested_session_step: ('maybe' | 'repeat')? id '(' id_list ')';
action: 'action' function '=' qualified_id ';';
sensor: 'sensor' qualified_id ';';
type_declaration: 'type' LID ('<' lid_list '>')? '='
                  '|'? type_branch ('|'type_branch)*;
type_branch: type | LID '(' type_branch_list ')';
\end{verbatim}
\caption{AF-Raf grammar for top level structures}
\label{fig:grammar-top-level}
\end{figure} % >>>

There are five kinds of declarations: for agent designs, for roles, for
types, for sessions, and for Java bindings. These where discussed above, in
the presentation of the abstract grammar. On the other hand, include
directives were not discussed above, because they do not produce any nodes
in the parse tree, and hence do not appear in the abstract grammar.  The
{\it include} directive specifies the name of a file that should be
textually included. Recursive includes are ignored.

There is an interesting interaction between how agent designs are found,
and how include directives are handled. Suppose that a design named {\tt A}
is needed, perhaps because name~{\tt A} appears in a statement. The AF-Raf
runtime then looks for a file named {\tt A.raf} in all the directories
mentioned in the environment variable {\tt RAFPATH}. Files named in include
directives are searched based on the same environment variable, {\tt
RAFPATH}. As a result, the following scenario is possible. The agent design
{\tt A} is in file {\tt A.raf}; the agent design {\tt B} is in file {\tt
B.raf}. Design~{\tt A} contains a statement that instantiates design~{\tt
B}; therefore, it includes file {\tt B.raf}. Similarly, design~{\tt B}
contains a statement that instantiates design~{\tt A}; therefore, it
includes file {\tt A.raf}. This creates a cycle of include directives.
But, because of the rule that recursive includes are ignored, the setup is
valid and works as expected.

AF-Raf formulas --- its boolean expressions --- use four boolean operators:
\verb+=>+~(implies), \verb+&+~(and), \verb+|+~(or), \verb+!+~(not).  On the
lowest level of precedence there is the \verb+=>+~operator.  On the next
level of precedence there are the \verb+&+~and~\verb+|+ operators. On the
next level of precedence there is the \verb+!+~operator.  The binary
operators (\verb+=>+,~\verb+&+,~\verb+|+) are right associative.

\begin{figure}\footnotesize % <<<
\begin{verbatim}
formula: formula_a ('=>' formula_a )?;
formula_a: atom_formula (('&' | '|') atom_formula)*;
atom_formula: predicate | '!' atom_formula | '(' formula ')';
predicate: term pred_op term | UID '(' term_list ')' ;
term: term_a (add_op term_a )*;
term_a: atom_term (mul_op atom_term )*;
atom_term: '(' term ')' | literal | LID | function ;
function: LID '(' term_list ')';
pred_op: (eq_op | cmp_op) ;
eq_op: ('==' | '!=');
cmp_op: ('<' | '>'); 
add_op: ('+' | '-');
mul_op: ('*' | '/' | '%');
unary_op: ('!' | '-'); 
\end{verbatim}
\caption{AF-Raf grammar for formulas and terms}
\label{fig:grammar-expr}
\end{figure} % >>>

\begin{example}
The associativity of \verb+&+~and~\verb+|+ is important only if they are
mixed without parentheses.  The formula \verb+A()&B()|C()+ is equivalent to
\verb+A()&(B()|C())+.  The formula \verb+A()|B()&C()+ is equivalent to
\verb+A()|(B()&C())+.
\end{example}

The grammar for terms allows several common operators on numbers and
strings: \verb-+-, \verb+-+, \verb+/+, and so on. Most terms in AF-Raf
programs are built using functions. The names of such functions are
arbitrary identifiers, starting with a lowercase letter. Such terms could
be thought of as being trees whose nodes are labelled by strings, the
function names. In particular, functions applied to zero arguments are very
similar to a string literal in a language like Java.

AF-Raf has a small but powerful set of statements
(\autoref{fig:grammar-statements}): assignments, {\bf new} statements,
action invocations, and updates of~beliefs.

\begin{figure}\footnotesize % <<<
\begin{verbatim}
statement: sequence_statement | atom_statement;
sequence_statement:
  '{' (statement  (';' statement )* ';'?)? '}' ;
atom_statement:
    assignment
  | new_statement
  | term_statement
  | update_belief;
assignment: LID ':=' statement;
new_statement: ( new_id | new_agent );
new_id: 'new' ( 'call' | 'session' );
new_agent:
  'new' UID ('(' term_list ')')? STRING? ('{' formula_list '}')?;
term_statement: term;
update_belief: ('+' | '-') predicate;
\end{verbatim}
\caption{AF-Raf Grammar for statements}
\label{fig:grammar-statements}
\end{figure} % >>>

The left-hand side of assignments should be a variable. The variable needs
not be declared separately.  The right-hand side can be any statement.  In
this respect, AF-Raf statements are similar to expressions --- they
evaluate to a value. More precisely, they evaluate to a term. The
convention that everything evaluates to a value is commonplace in
functional languages but a little peculiar in imperative ones.
Nevertheless, this simple convention allows some useful idioms.

\begin{example}
Because assignments are themselves statements, and because the right-hand
side of an assignment should be a statement, it is legal to chain
assignments.
\begin{verbatim}
  x := y := do_something(z);
\end{verbatim}
Action invocations have exactly the same syntax as terms. Above, ${\it
do\_something}(z)$ could be either a literal term or an action invocation,
depending on whether there exists an action named ${\it do\_something}$.
\end{example}

The {\bf new} statements come in two syntactic flavours. In one case the
{\bf new} statements are used to obtain fresh names (also called
identifiers). These names have special meaning, which is presented later
(\autoref{sec:opsem}). Syntactically, in this case, the keyword {\bf new}
must be followed by the category of the name that should be generated. In
the other case, {\bf new} statements are used to instantiate agents. In
this case, the {\bf new} keyword must be followed by the name of the design
to be instantiated. Optionally, the user \emph{may} specify a list of agent
adresses, a name for the new agent, and a set of initial beliefs.

As mentioned earlier, action invocations have the same syntax as terms.
Statements that update beliefs also have a very similar syntax: a term
preceded by either \verb-+-~or~\verb+-+, depending on whether the belief
should be added or removed.

\begin{example}
The core of AF-Raf performs no hidden actions. For example, received
messages must be explicitly removed from the belief base.
\begin{verbatim}
  rule Message(other, someData(x, y)) {
      -Message(other, someData(x, y));
      /* ... rest ... */
  }
\end{verbatim}
There exist, however, several constructs defined as syntactic sugar, that
reduce the verbosity of AF-Raf programs.
\end{example}

The lexical conventions of AF-Raf are rather typical for a Java-like
language (\autoref{fig:Grammar-lexer}). The one interesting feature is that
AF-Raf distinguishes identifiers starting with a lowercase letter
(\verb+LID+) from identifiers starting with an uppercase letter
(\verb+UID+). Again, such a convention is unusual in imperative languages,
but commonplace in functional languages. In AF-Raf, predicates have names
starting with an uppercase letter, while functions have names starting with
a lowercase letter. From the language implementer's perspective, this
convention helps disambiguate the grammar. From the user's perspective,
this convention acts as a helpful mnemonic.

\begin{figure}\footnotesize % <<<
\begin{verbatim}
qualified_id: (id '.')* id;
literal: STRING | INT;
id: LID | UID;

INT: '0'..'9'+;
STRING: '"' (~'"' | '\\"')* '"';
LID: ('a'..'z')('a'..'z'|'A'..'Z'|'0'..'9'|'$')*;
UID: ('A'..'Z')('a'..'z'|'A'..'Z'|'0'..'9'|'$')*;
WS:   (' '|'\t'|'\n'|'\r')+;
BLOCK_COMMENT: '/*' ( options : . )* '*/';
LINE_COMMENT: '//' ~('\n'|'\r')* ('\r'|'\n');
\end{verbatim}
\caption{AF-Raf Grammar Lexer}
\label{fig:Grammar-lexer}
\end{figure} % >>>

% >>>
\section{Type System} % <<<

\todo{Reuse material from \autoref{old-stuff}.}

% >>>
\section{Operational Semantics}\label{sec:opsem} % <<<

There are many alternative ways for describing the semantics of a
programming language. Operational semantics and, more specifically,
small-step operational semantics are one of the more pragmatic methods.

One way to view operational semantics is as a concise way to describe an
interpreter for the language being defined. This interpreter has a standard
structure. It consists of a several procedures, each responsible with
interpreting one type of node in the abstract syntax tree. These procedures
are often mutually recursive, and they pass around a \emph{context}. The
typical information one would find in a context is a set of bindings from
program variables to program values. In this view, the difference between
big-step operational semantics and small-step operational semantics is as
follows. In big-step operational semantics, these procedure have no
side-effects: they all compute a \emph{value} out of a context and an
\emph{expression}. On the other hand, in small-step operational semantics
there is a global \emph{state} of the program --- held in an interpreter
data structure --- which changes while the program is being interpreted.
The effects of the interpreter's procedure exist and are important.

\todo{Continue here. Describe the above more formally, then move on to
AF-Raf.}

The semantics of programming languages is often specified as transitions
from a state to another with each step of individual computation. The
specification takes the form of a set of inference rules that define valid
transitions.
\rg{Technically, these are `small-step operational semantics'.
Perhaps say that, intuitively, giving the operational semantics of a
language amounts to providing a simple (perhaps inefficient) interpreter
for it.}

At each time step, an AF-Raf agent
\begin{enumerate}
\item transforms messages from its inbox into beliefs that it received
those messages,
\item asks each sensor for new beliefs, and
\item executes those rules that are activated by the current belief base.
\end{enumerate}

\rg{This paragraph should be re-written. For example, the first thing you
say is that the some method is called `execute'. First, the reader has no
idea that you are talking about the implementation of the interpreter.
Second, how methods are named in the interpreter is rather unimportant,
certainly not important enough to be the first thing you mention.}
An AF-Raf agent program interpretation starts with the execute() method.
First operation is to copy any messages received into the inbox for the
agent to process further. Only messages received since the last invocation
of this method are copied into the inbox. Next step is to get information
from all the sensors. After that a list is created to hold last fired
rules. Then each rule is handled in the following way. A query is
constructed based on the rule's condition and run on the belief base. The
result will be a list of sets of bindings. In the case the list is not
empty, for each set of bindings a PlanStep is constructed by applying the
bindings to the statement of the rule. If all the variables have a binding
the rule is included in the list of last fired rules and the PlanStep is
executed. A PlanStep is any of the following statements: an assignment, a block,
a call, a foreach, an if, an update, or a while.

TODO: inference rules
% >>>
% >>>

\chapter{Case Study: The Contract Net Interaction Protocol}\label{ch:casestudy} % <<<

This chapter tries to answer the following research question: \textit{Is
the readability and usability of an agent oriented programming language
improved by the addition of the role and session programming constructs as
well as by the algebraic data types and the type checking functionality?}
In order to answer this twofold research question the Af-Raf implementation
of a FIPA interaction protocol is analysed in a case study.

\section{The FIPA Contract Net Interaction Protocol} % <<<
The FIPA Contract Net Interaction Protocol\cite{web:fipa} describes a
pattern of interaction very similar to the Contract Net Interaction
Protocol\cite{DBLP:journals/tc/Smith80a} developed by Smith and Davis in
1980, to which it adds rejection and confirmation communicative acts. One
agent, the initiator, takes the manager role, the other agents are the
participants. The manager has a task that wants to be performed by the
participants in the most effective way possible. The effectiveness is
considered with regard to some specified properties like the price, the
time to completion, or the fair distribution of tasks. The participants
have to respond before a given deadline with either a refuse or a proposal.
Negotiation continues with the participants that proposed, from which one,
several or no agent is chosen to actually perform the task. The chosen
agents have to inform about the execution of the task indicating either the
success, the failure or by giving a result.
% >>>
\section{Af-Raf Implementation} % <<<
The Af-Raf implementation of the FIPA Contract Net Interaction Protocol.
Figure~\ref{fig:contract-roles}

\begin{figure}\footnotesize % <<<
\begin{verbatim}
// assumes primitive types:
//   integer = 0() | 1() | ...
//   agentid = agentId(...)

type conditions = conditions(integer); // say, price

type task = addVAT(integer)
          | increase(integer, integer)
          | reduce(integer, integer);

type request = request(task, conditions, deadline(integer));
type bid = bid(conditions);
type decision = win() | lose();
type result = failure() | success(integer);
// success could have, say, price after adding VAT

role Buyer {
  ask : request -> bid;
  commit : decision -> result;
}

role Seller {
  sell : request -> result;
}

session BidRound(owner, s : Seller, b1 : Buyer, b2 : Buyer) {
  request : owner -> s;
  request : s -> b1;
  request : s -> b2;
  oneof
    { bid : b1 -> s; win() : s -> b1; result : b1 -> s; }
    { maybe bid : b1 -> s; lose() : s -> b1; };
  oneof
    { bid : b2 -> s; win() : s -> b2; result : b2 -> s; }
    { maybe bid : b2 -> s; lose() : s -> b2; };
  result : s -> owner
}
\end{verbatim}
\caption{Contract Net types, roles and sessions}
\label{fig:contract-roles}
\end{figure} % >>>

\begin{figure}\footnotesize % <<<
\begin{verbatim}
include stdio
include ContractNet

              
action vat(x) = raftest.xx;
action dec(x, y) = raftest.xx;
action inc(x, y) = raftest.xx;
action mul(x, y) = raftest.xx;

agent ContractNetBuyer : Buyer {
  operation ask(request(task, conditions(y), deadline(z)))
  when Multiplier(m) {
    +LastTask(sid, task);
    return bid(mul(m, y));
  }
  
  operation commit(win()) 
    when LastTask(sid, decrease(x, y)) { return dec(x, y); }
    when LastTask(sid, increase(x, y)) { return inc(x, y); }
    when LastTask(sid, addVAT(x)) { return vat(x); }
    
}
\end{verbatim}
\caption{Contract Net Buyer agent}
\label{fig:contract-buyer}
\end{figure} % >>>

\begin{figure}\footnotesize % <<<
\begin{verbatim}
include stdio
include ContractNet

agent ContractNetSeller(b1 : BuyerRole, b2 : BuyerRole): Seller {
  operation sell(r) {
    return if b1.ask(r) < b2.ask(r) then b1.commit(win()) 
                                    else b2.commit(win())
  }
}
\end{verbatim}
\caption{Contract Net Seller agent}
\label{fig:contract-seller}
\end{figure} % >>>

% >>>
% >>>

\chapter{Conclusions}\label{ch:conc} % <<<

Starting from high-level similarities between certain aspects of an
agent-oriented programming language (2APL) and a functional programming
language (Haskell), we described three new features for the former:
(1)~algebraic data types, which constrain the content of messages,
(2)~roles, which constrain how particular agents interact, and
(3)~sessions, which describe slices of the global interactions in the agent
system. Together, these \emph{language features} support organisational
concepts, which are so far discussed mostly in the literature on
agent-oriented methodologies and frameworks.

The aim is to provide programmers with a wider set of options at their disposal
for structuring large agent systems. For example, they can use existing
methodologies during the design phase and they can use a mixture of our
proposed language features and existing organisational frameworks during the
implementation phase. First-class language support has at least two
advantages over organisational frameworks. First, implementations tend to be
shorter and more readable. Second, implementations have stronger correctness
guarantees because of the extra type-checking. On the other hand,
organisational frameworks may be preferred when one needs to interact with
legacy code.

The new agent oriented programming language, AF-Raf, based on the existing
Agent Factory framework, implements the innovative features that have been
identified. AF-Raf has types, including algebraic data types, which are
used to constrain the content of the messages. It has type checking, a
feature very useful in identifying programming errors early on. It has
roles, which organise agent behaviour, promoting encapsulation and reuse.
It also have sessions, which further organise agent interaction at a higher
level, namely at the role level. A case study is used to illustrate the
results.  The AF-Raf programming language is properly formalised, with its
grammar specified in EBNF format and its operational semantics is defined
using inference rules.

% >>>
\bibliographystyle{plain}
\bibliography{thesis}
\appendix

\end{document}
% >>>
% vim:fmr=<<<,>>>:tw=75:spell:fo+=t:
